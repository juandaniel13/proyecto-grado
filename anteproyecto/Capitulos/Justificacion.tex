\chapter{Justificación}

El presente proyecto tiene  como  propósito  ofrecer  una  solución  tecnológica
innovadora mediante el desarrollo de una aplicación web que permita  analizar  y
simular  redes  LPWAN  (Low  Power  Wide  Area  Network)  aplicadas  a  entornos
agrícolas, utilizando inteligencia artificial (IA) generativa para  la  toma  de
decisiones  en  contextos  rurales.   Esta   propuesta   busca   contribuir   al
fortalecimiento del sector agrícola colombiano, considerado un pilar estratégico
de  la  economía  nacional  por  su  papel  en  la  generación  de  empleo,   el
abastecimiento alimentario y el desarrollo regional.

En un contexto donde los efectos del cambio climático y la variabilidad  de  las
condiciones ambientales afectan directamente la productividad del campo, se hace
necesario incorporar herramientas tecnológicas que faciliten  la  planificación,
monitoreo y predicción de  variables  agrícolas.  Mediante  el  uso  de  IA,  la
aplicación podrá analizar escenarios específicos —tipo de cultivo, localización,
presupuesto y requerimientos de conectividad— para determinar el protocolo LPWAN
más adecuado  entre  alternativas  como  LoRa,  Sigfox  y  NB-IoT,  considerando
criterios de cobertura, rendimiento, escalabilidad y costo.

\begin{itemize}
      \item \textbf{Justificación Investigativa:} Desde una perspectiva científica
            y tecnológica, este proyecto representa un aporte significativo a  las
            líneas de investigación en agricultura inteligente,  internet  de  las
            cosas (IoT) y comunicaciones de baja potencia, al integrar un  enfoque
            de análisis predictivo impulsado por IA generativa. El  desarrollo  se
            enmarca dentro de los  objetivos  del  grupo  de  investigación  LIDER
            (Laboratorio de Investigación y Desarrollo en Electrónica y Redes)  de
            la Universidad Distrital, a través del semillero SCISEN, articulándose
            con  las  áreas  institucionales  de  ciencias  de   la   computación,
            infraestructura  tecnológica  y   telecomunicaciones.   Asimismo,   la
            aplicación  de  la  metodología  híbrida (modelo de prototipado + Kanban)   permitirá   gestionar   el
            desarrollo del proyecto de manera iterativa y  flexible,  garantizando
            una  evolución  continua  del  producto  y  un  enfoque  orientado   a
            resultados.

      \item \textbf{Justificación Social:} La implementación de esta herramienta
            beneficiará directamente a las comunidades agrícolas, especialmente en
            zonas rurales con limitaciones de conectividad y acceso a  tecnología.
            Al ofrecer una plataforma accesible que asista a los actores  técnicos
            en la selección de la red de comunicación más eficiente,  se  promueve
            la inclusión digital y el empoderamiento tecnológico del sector rural.
            Esto no solo contribuirá a reducir las brechas digitales, sino también
            a mejorar la calidad de vida de los productores  mediante  el  uso  de
            datos inteligentes para optimizar sus operaciones agrícolas.
      \item \textbf{Justificación Económica:} El uso eficiente de las tecnologías
            LPWAN  en  la  agricultura  tiene  un  impacto  económico  directo  al
            optimizar los costos de implementación de redes y mejorar  la  gestión
            de los cultivos. La aplicación propuesta permitirá evaluar la relación
            costo-beneficio de cada tecnología en función del contexto productivo,
            lo  que  reducirá  pérdidas   económicas   derivadas   de   decisiones
            tecnológicas inadecuadas. Además, la posibilidad de simular escenarios
            antes de su implementación real representa un ahorro significativo  en
            infraestructura y mantenimiento, contribuyendo al  fortalecimiento  de
            la economía local y nacional.
      \item \textbf{Justificación Ambiental:} El proyecto aporta al desarrollo
            sostenible mediante el fomento de prácticas agrícolas más eficientes y
            responsables  con  el  medio  ambiente.  Al  proporcionar  información
            precisa sobre variables  y  condiciones  agrícolas,  los  agricultores
            podrán reducir el uso innecesario de insumos como agua,  fertilizantes
            y energía. De igual forma, el aprovechamiento de tecnologías  de  bajo
            consumo energético como las redes  LPWAN  contribuye  a  disminuir  la
            huella ecológica de las operaciones agrícolas, promoviendo  un  modelo
            de producción más sostenible y resiliente ante el cambio climático.
      \item \textbf{Justificación Académica:} El presente proyecto  se  desarrolla
            bajo la modalidad de grado de investigación, investigación - creación, innovación
            reglamentada en el artículo 7 del acuerdo No.02 (marzo 2023) de consejo de facultad de
            ingeniería. Este proyecto se llevará a cabo con en el fin de  que  cada  uno  de  sus
            autores opten por el título de Ingeniero de Sistemas.
\end{itemize}