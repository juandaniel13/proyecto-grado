\chapter{Objetivos}

\section*{\textcolor{}{Objetivo General}}

{\large \textbf{Desarrollar una plataforma de software inteligente} que integre simulaciones de redes LPWAN y modelos de decisión agronómicos, potenciada por inteligencia artificial, para optimizar la selección de tecnologías LPWAN} en proyectos de agricultura de precisión en zonas rurales de Colombia.

\section*{\textcolor{}{Objetivos específicos:}}

\begin{enumerate}[leftmargin=*, align=left, label=\textbf{\arabic*.}]
    \item \textbf{Identificar y documentar} los requerimientos técnicos y agronómicos determinantes para la selección de tecnologías LPWAN en el contexto agrícola colombiano, a través de la revisión literaria y la consulta con expertos, estableciendo los requisitos funcionales y no funcionales de la plataforma.
    
    \item \textbf{Diseñar} una arquitectura de software que integre agentes de inteligencia artificial para la simulación y toma de decisiones de arquitecturas LPWAN.
    
    \item \textbf{Implementar} el sistema de decisión y simulación automatizada para selección de tecnologías LPWAN, bajo un esquema de desarrollo ágil, asegurando su afinidad con los requerimientos previamente definidos.
    
    \item \textbf{Evaluar} la precisión técnica de las recomendaciones mediante validación con expertos y mediciones reales.
    
    \item \textbf{Cuantificar} el impacto en eficiencia operativa y económica frente a métodos tradicionales.
    
    \item \textbf{Documentar detalladamente} la arquitectura de software, el proceso de desarrollo iterativo y los resultados de la validación operativa y económica para garantizar la reproducibilidad, escalabilidad y transferibilidad de la solución a otros contextos agropecuarios.
\end{enumerate}

