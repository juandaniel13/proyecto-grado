\chapter{Introducción}
La agricultura, pilar histórico  de  la  economía  y  sustento  de  millones  en
Colombia,  se  ve  crecientemente  amenazada  por  el  cambio  climático  y   la
variabilidad ambiental extrema, que generan pérdidas significativas y reducen la
competitividad del sector.  Para  mitigar  estos  riesgos,  la  adopción  de  la
agricultura de precisión, basada en redes de sensores y análisis  de  datos,  se
presenta  como  una  solución  indispensable.  Sin  embargo,  su  implementación
efectiva tropieza con una barrera crítica previa: la complejidad  técnica  y  la
incertidumbre en la toma de decisiones inicial para seleccionar y  desplegar  la
infraestructura de conectividad adecuada, como las redes  LPWAN  (LoRa,  Sigfox,
NB-IoT, entre otras), en diversos contextos agrícolas.\\


Ante  este  panorama,  la  incorporación  de  tecnologías   modernas   como   la
inteligencia artificial (AI) y las redes LPWAN (Low  Power  Wide  Area  Network)
ofrece nuevas  oportunidades  para  fortalecer  la  agricultura  inteligente  en
Colombia. En este contexto, se propone el desarrollo de una aplicación web  para
el análisis y simulación  de  arquitecturas  de  redes  LPWAN  con  inteligencia
artificial, orientada a mejorar la  toma  de  decisiones  en  el  despliegue  de
tecnologías de monitoreo agrícola. La aplicación  empleará  un  sistema  de  que
incluye un agente de IA, el cual a partir de un contexto determinado —como  tipo
de cultivo, ubicación geográfica, presupuesto  disponible  y  requerimientos  de
conectividad—,  identificará  las  variables  más  relevantes  y  realizarán  un
análisis comparativo entre las distintas tecnologías LPWAN, seleccionando la que
mejor se ajuste al escenario propuesto, proponiendo una aquitectura báse para su
despliegue y recomendando los sensores más adecuados para el tipo de cultivo que
se desea monitorear. Con base en dicho análisis, el sistema recomendará  la  red
más adecuada según criterios de rendimiento, costo, cobertura y escalabilidad.\\

En este contexto, el presente proyecto se desarrolla en el marco  del  semillero
SCISEN (Smart cities \& sensor network),  adscrito  al  grupo  de  investigación
LIDER (Laboratorio de Investigación y desarrollo en Electrónica y redes)  de  la
Universidad Distrital Francisco José de Caldas. Este grupo trabaja  en  diversas
líneas de investigación internas, entre las que se destacan  en  este  proyecto:
Agrointeligente (Smart Agro), internet de las cosas  (IoT)  y  el  desarrollo  y
programación de hardware, firmware y software  para  sistemas  de  comunicación.
Estas  líneas  se  encuentran  alineadas   con   los   ejes   de   investigación
institucionales en ciencias de la computación, desarrollo regional  sustentable,
infraestructura   y   tecnología,   redes    de    sensores    inalámbricos    y
telecomunicaciones. Asimismo, el  proyecto  está  directamente  relacionado  con
varios Objetivos de Desarrollo Sostenible (ODS) establecidos por la ONU: \\
\begin{itemize}
    \item  ODS 2. Hambre Cero: Este proyecto contribuye  a  una  agricultura  más
          resiliente y productiva al facilitar la toma de  decisiones  informada
          para el despliegue de tecnologías  de  agricultura  de  precisión.  Al
          ayudar a seleccionar la arquitectura  de  conectividad  (LPWAN)  más
          adecuda para cada  contexto,  la  herramienta  reduce  la  barrera  de
          entrada y el riesgo de inversión en sistemas de monitoreo y predicción
          agroclimática. Esto, a su vez, permite a los  productores  implementar
          soluciones más eficaces para anticipar fenómenos climáticos  adversos,
          optimizar recursos y reducir pérdidas, fortaleciendo así la  seguridad
          alimentaria  desde  una  perspectiva  de   planificación   tecnológica
          robusta.
    \item  ODS 9. Industria, Innovación e Infraestructura:  El  proyecto  es  una
          innovación  directa  en  el  ecosistema  tecnológico   agroindustrial.
          Desarrolla una herramienta digital avanzada (aplicación  web  con  IA)
          que resuelve un cuello de botella crítico en la cadena de valor de  la
          agricultura 4.0: la selección de arquitectura de  conectividad.  Al
          proporcionar un método accesible y basado en datos para diseñar  redes
          LPWAN  eficientes,  promueve  la   construcción   de  una arquitectura
          tecnológica más resiliente, accesible  y  adecuada  al  entorno  rural
          colombiano,   fomentando   directamente    la    innovación    y    la
          industrialización sostenible del sector.
    \item ODS 13. Acción por el Clima: La herramienta  contribuye  a  la  acción
          climática de manera indirecta pero  potente,  al  ser  un  facilitador
          clave para la agricultura climáticamente inteligente. Al optimizar  la
          elección de redes de sensores, maximiza la viabilidad y efectividad de
          los sistemas que monitorean variables climáticas y  ambientales.  Esto
          permite a los agricultores adaptar sus prácticas  con  base  en  datos
          precisos, mejorar la gestión de recursos como el agua y los insumos, y
          reducir  la  vulnerabilidad  de  los  cultivos,  promoviendo  así  una
          adaptación sistémica y basada en evidencia frente al cambio climático.
\end{itemize}

El desarrollo se llevará a cabo bajo la metodología ágil Scrumban,  que  combina
la flexibilidad de Kanban con la estructura iterativa de Scrum, favoreciendo  la
gestión adaptativa de tareas y la entrega  continua  de  valor.  Esta  propuesta
busca  contribuir  al  avance  de  la  agricultura  inteligente   en   Colombia,
facilitando  la  integración  de  herramientas  digitales  en   zonas   rurales,
optimizando el uso de recursos tecnológicos y promoviendo la sostenibilidad  del
sector.

Adicionalmente, se articula con el proyecto doctoral titulado “Estructuración de
un modelo para el análisis, simulación y  aplicación  de  tecnologías  LPWAN,  a
través de la integración  comunitaria  de  redes  IOT  al  agro  inteligente  en
sectores rurales colombianos", cuyo investigador principal  es  el  Ing.  Carlos
Andrés Martínez Alayón, bajo  la  dirección  del  docente  tutor  Roberto  Ferro
Escobar. Este proyecto ha sido institucionalizado sin recursos por el Consejo de
la Facultad de Ingeniería y está registrado en el  Sistema  de  Información  del
Centro de  Investigaciones  (SICIUD),  fortaleciendo  su  respaldo  académico  y
científico. Con las siguientes especificaciones:
\begin{itemize}
    \item Código SICIUD: 3370084523
    \item Estado: Vigente sin financiación
    \item Grupo  de  Investigación: Laboratorio   de
          Investigación y desarrollo
          en Electrónica y redes.
\end{itemize}
El desarrollo de este proyecto  contribuirá  con  los  siguientes  productos  de
investigación: La dirección de una tesis de pregrado, entrega de un  informe  de
investigación, el desarrollo de un producto de desarrollo web y un artículo para
la participación de una ponencia a un evento nacional o internacional.