\chapter{Metodología}

El proyecto se desarrollará bajo un enfoque iterativo e incremental con gestión ágil tipo Kanban, estructurado en fases definidas que permiten planificación inicial, desarrollo evolutivo por iteraciones y cierre formal del proyecto.

La metodología se divide en cuatro grandes fases:

\begin{enumerate}
    \item Planificación
    \item Ciclo de vida iterativo
    \item Despliegue
    \item Documentación y cierre    
\end{enumerate}

Aunque se adopta Kanban como método de gestión del flujo de trabajo, el proyecto se organiza en iteraciones funcionales que permiten entregar incrementos progresivos del sistema.

\section{Fase de Planificación}

Durante esta fase se establecen las bases estratégicas y técnicas del proyecto. Se definen:

\begin{itemize}
    \item Alcance y objetivos generales
    \item Requerimientos iniciales
    \item Arquitectura base del sistema
    \item Backlog inicial priorizado
    \item Configuración del entorno de desarrollo
\end{itemize}

Al finalizar esta fase se obtiene la aprobación formal del enfoque técnico y del plan de trabajo.

\section{Ciclo de Vida Iterativo}

El desarrollo se ejecuta mediante tres iteraciones incrementales, cada una con entregables funcionales.

Cada iteración incluye actividades de:

\begin{itemize}
    \item Análisis
    \item Diseño
    \item Implementación
    \item Pruebas
    \item Entrega del prototipo
\end{itemize}

El flujo de trabajo se gestiona mediante Kanban, permitiendo flexibilidad en prioridades y control visual del avance.

\subsection{Iteración 1 - Base del Sistema}

Objetivo: Construir la estructura fundamental del sistema.

Incluye:

\begin{itemize}
    \item Configuración de base de datos
    \item Autenticación y gestión de usuarios
    \item Estructura backend
    \item Interfaces iniciales
    \item Primer prototipo funcional
\end{itemize}

Resultado: Sistema base estable y navegable.

\subsection{Iteración 2 - Desarrollo Funcional Principal}

Objetivo: Implementar las funcionalidades centrales del sistema.

Incluye:

\begin{itemize}
    \item Desarrollo de módulos principales
    \item Integración de servicios externos (si aplica)
    \item Lógica de negocio avanzada
    \item Pruebas funcionales intermedias
\end{itemize}

Resultado: Versión funcional completa del sistema.

\subsection{Iteración 3 - Optimizar y Validar el Producto Final}

Objetivo: Refinar, optimizar y validar el producto final.

Incluye:
\begin{itemize}
    \item Optimización de rendimiento
    \item Corrección de errores
    \item Pruebas integrales
    \item Pruebas de usuario
    \item Ajustes finales
\end{itemize}

Resultado: Prototipo final validado y listo para despliegue.

\section{Fase de Despliegue}

Durante esta fase se realiza:

\begin{itemize}
    \item Configuración del entorno de producción
    \item Migración y configuración de base de datos
    \item Pruebas en ambiente real
    \item Verificación de estabilidad
    \item Se asegura que el sistema quede completamente operativo.
\end{itemize}

Resultado: Prototipo final validado y listo para despliegue.

\section{Fase de Documentación y Cierre}

Se realiza:
\begin{itemize}
    \item Documentación técnica
    \item Manual de usuario
    \item Documentación de arquitectura
    \item Informe final del proyecto
    \item Cierre formal
\end{itemize}