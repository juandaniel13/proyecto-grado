\chapter{Metodología}

El desarrollo del proyecto se fundamenta  en  una  metodología  ágil  basada  en
Kanban, combinada con un modelo de prototipado iterativo. Este  enfoque  permite
gestionar  el  flujo  de  trabajo  de  manera   continua,   facilitar   entregas
incrementales y adaptar el sistema de acuerdo con la retroalimentación  obtenida
durante el proceso de desarrollo. \\


\section{Gestión del Proyecto mediante Kanban} El proyecto se gestiona  mediante
un tablero Kanban, en el cual las actividades se  organizan  en  función  de  su
estado de avance. El uso de Kanban permite: 
\begin{itemize}
    \item Visualizar el progreso del proyecto.
    \item Priorizar actividades de manera dinámica.
    \item Facilitar entregas parciales y funcionales.
    \item Controlar el flujo de trabajo durante todo el desarrollo.
\end{itemize}

\noindent Por otro lado, el tablero propuesto para el proyecto se estructura de la siguiente manera:

\begin{itemize}
    \item Backlog
    \item Análisis
    \item Diseño
    \item Desarrollo
    \item Pruebas
    \item Finalizado
\end{itemize}

Se establecen límites de trabajo en progreso (WIP) con el fin de evitar sobrecarga y mantener eficiencia en la ejecución. La gestión del trabajo se realiza mediante priorización dinámica del backlog, permitiendo adaptar el desarrollo según hallazgos técnicos y validaciones parciales.

Kanban es utilizado exclusivamente como mecanismo de gestión del flujo de tareas, mientras que la estructura técnica del desarrollo se definió mediante el modelo de prototipado.

\section{Planificación inicial}

La fase de planificación se desarrolla previo al inicio del ciclo  de  vida  del
software y tiene como propósito establecer las bases generales del proyecto.  En
esta etapa se definen:

\begin{itemize}
    \item El alcance inicial del sistema.
    \item Los objetivos del proyecto.
    \item Los requisitos funcionales y no funcionales.
    \item Los actores involucrados.
    \item Los lineamientos metodológicos y técnicos.
\end{itemize}

Los resultados de esta fase alimentan el tablero Kanban y sirven como referencia
durante el desarrollo, permitiendo ajustes cuando el proyecto lo requiera.

\section{Ciclo de Vida del Desarrollo con Prototipado}

El ciclo de vida del sistema se desarrolla  de  manera  iterativa  a  través  de
prototipos, siguiendo las fases que se describen a continuación:

\subsection{Iteración 1 - Prototipo Inicial}

Incluye las siguientes etapas:

\begin{itemize}
    \item Análisis
    \begin{itemize}
        \item Refinamiento de historias de usuario prioritarias.
    \end{itemize}
\end{itemize}

\begin{itemize}
    \item Diseño
    \begin{itemize}
        \item Diseño de interfaz base.
        \item Diseño técnico preliminar.
    \end{itemize}
\end{itemize}

\begin{itemize}
    \item Implementación
    \begin{itemize}
        \item Desarrollo de la estructura funcional de la Web App.
        \item Configuración de base de datos.
        \item Implementación de lógica básica del sistema.
    \end{itemize}
\end{itemize}

\begin{itemize}
    \item Pruebas
    \begin{itemize}
        \item Pruebas funcionales iniciales.
        \item Validación de comportamiento general.
    \end{itemize}
\end{itemize}

El resultado de esta iteración es un prototipo funcional base.

\subsection{Iteración 2 - Prototipo Refinado}

\begin{itemize}
    \item Análisis
    \begin{itemize}
        \item Ajustes derivados del prototipo anterior.
        \item Definición de integración de inteligencia artificial.
    \end{itemize}
\end{itemize}

\begin{itemize}
    \item Diseño
    \begin{itemize}
        \item Diseño de integración del modelo de IA.
        \item Refinamiento arquitectónico.
    \end{itemize}
\end{itemize}

\begin{itemize}
    \item Implementación
    \begin{itemize}
        \item Integración del modelo de IA.
        \item Optimización del backend.
        \item Ajustes de interfaz.
    \end{itemize}
\end{itemize}

\begin{itemize}
    \item Pruebas
    \begin{itemize}
        \item Pruebas de integración.
        \item Validación de resultados predictivos.
        \item Corrección de defectos.
    \end{itemize}
\end{itemize}


El resultado es el prototipo final validado.

\subsection{Fase Final - Despliegue y Cierre}

Incluye:

\begin{itemize}
    \item Configuración del entorno productivo.
    \item Validación del pipeline de integración continua.
    \item Pruebas finales.
    \item Análisis de resultados.
    \item Documentación final.
\end{itemize}

Durante las iteraciones se realizaron despliegues incrementales en entornos de prueba, mientras que en esta fase se consolida la versión estable del sistema.