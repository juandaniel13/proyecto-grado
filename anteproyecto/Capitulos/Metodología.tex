\chapter{Metodología}

El desarrollo del proyecto se fundamenta  en  una  metodología  ágil  basada  en
Kanban, combinada con un modelo de prototipado iterativo. Este  enfoque  permite
gestionar  el  flujo  de  trabajo  de  manera   continua,   facilitar   entregas
incrementales y adaptar el sistema de acuerdo con la retroalimentación  obtenida
durante el proceso de desarrollo. \\

La selección de Kanban se justifica por el tamaño reducido del equipo de trabajo
y la necesidad de mantener un control flexible y visual de las actividades,  sin
la imposición de roles rígidos. Por su parte, el modelo de  prototipado  permite
construir versiones funcionales del sistema desde etapas tempranas, favoreciendo
la validación progresiva de los requisitos y la mejora continua del producto.

\section{Planificación inicial}

La fase de planificación se desarrolla previo al inicio del ciclo  de  vida  del
software y tiene como propósito establecer las bases generales del proyecto.  En
esta etapa se definen:

\begin{itemize}
    \item El alcance inicial del sistema.
    \item Los objetivos del proyecto.
    \item Los requisitos funcionales y no funcionales.
    \item Los actores involucrados.
    \item Los lineamientos metodológicos y técnicos.
\end{itemize}

Los resultados de esta fase alimentan el tablero Kanban y sirven como referencia
durante el desarrollo, permitiendo ajustes cuando el proyecto lo requiera.

\section{Gestión del Proyecto mediante Kanban} El proyecto se gestiona  mediante
un tablero Kanban, en el cual las actividades se  organizan  en  función  de  su
estado de avance.  Cada  tarea  representa  una  unidad  de  trabajo  que  puede
corresponder a análisis, diseño, desarrollo, pruebas  o  ajustes  derivados  del
prototipado. \\ El uso de Kanban permite: \begin{itemize}
    \item Visualizar el progreso del proyecto.
    \item Priorizar actividades de manera dinámica.
    \item Facilitar entregas parciales y funcionales.
    \item Controlar el flujo de trabajo durante todo el desarrollo.
\end{itemize}

\section{Ciclo de Vida del Desarrollo con Prototipado}

El ciclo de vida del sistema se desarrolla  de  manera  iterativa  a  través  de
prototipos, siguiendo las fases que se describen a continuación:

\subsection{Análisis} En esta fase se recopilan y  analizan  los  requerimientos
funcionales y no funcionales del sistema.  Se  documentan  las  necesidades  del
usuario, se refinan y detallan  las  historias  de  usuario  definidas  en  la
planificación inicial.
comprender interacciones complejas. \subsection{Diseño}  Se  realiza  el  diseño
general del sistema, incluyendo la estructura de sus componentes y la  propuesta
de interfaz de usuario. Esta fase busca definir cómo interactúan  los  elementos
del sistema y cómo será la experiencia del usuario.  \subsection{Implementación}
Corresponde a la construcción del prototipo mediante el  desarrollo  del  código
necesario para materializar las funcionalidades definidas.  Cada  implementación
da lugar a una versión funcional del sistema. \subsection{Pruebas}  Se  realizan
pruebas para verificar el correcto  funcionamiento  del  prototipo,  identificar
errores  y  validar  que  las  funcionalidades  desarrolladas  cumplan  con  los
requerimientos  establecidos.  \subsection{Validación  y  Retroalimentación}  El
prototipo es evaluado con el fin de  identificar  oportunidades  de  mejora.  La
retroalimentación obtenida se traduce en nuevas tareas o  ajustes  que  ingresan
nuevamente  al  tablero  Kanban,  dando  inicio  a  una  nueva   iteración.   La
retroalimentación y puntos clave obtenidos en la reunión con el usuario final se
plasmarán en un acta con un formato claro y preciso. \\


\textbf{Resumen del ciclo:} Este ciclo se repite  tantas  veces
como sea necesario  hasta  obtener  un  prototipo  final  que  cumpla  con  los
objetivos del proyecto.

\section{Despliegue y Mantenimiento}

Las actividades de despliegue y mantenimiento se consideran posteriores al ciclo
de  desarrollo.  El  sistema  podrá  ser  desplegado  de   manera   incremental,
permitiendo la publicación de versiones funcionales sin afectar su  estabilidad.
Posteriormente, se realizarán tareas de  mantenimiento  correctivo  y  evolutivo
según sea necesario.