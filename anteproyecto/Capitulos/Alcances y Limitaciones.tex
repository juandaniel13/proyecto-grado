\chapter{Alcances y Limitaciones}

\section{Alcances del proyecto}

\subsection{Desarrollo de una  Web  App  Funcional}  El  proyecto  contempla  la
construcción de una plataforma web plenamente operativa que permita gestionar la
información agrícola, visualizar datos relevantes y  acceder  a  funcionalidades
clave definidas  en  los  requisitos  del  sistema.
\subsection{Integración  de Modelos de IA para  Predicción}  El  sistema  incluirá  model
de inteligencia artificial generativa capaces de usar datos históricos, documentación relevante y
herramientas (tools), con  el  fin
de predecir variables agrícolas específicas. La IA  será  un  componente  activo
dentro de la Web App, permitiendo generar  análisis  y  pronósticos  dentro  del
flujo de la aplicación.
\subsection{Proceso Iterativo de Prototipos} El  alcance
incluye un ciclo iterativo de creación, evaluación y mejora de prototipos (fase
7.3). Cada iteración permitirá refinar tanto la interfaz como el desempeño
funcional y predictivo del sistema hasta obtener  un  prototipo  final  robusto.
\subsection{Arquitectura Base Sólida y Escalable} El proyecto abarca el diseño y
construcción de una arquitectura inicial estable  (fase  7.1),  que  sirva  como
plataforma para las iteraciones posteriores.  La  arquitectura  será  escalable,
modular  y   preparada   para   recibir   futuras   expansiones   del   sistema.
\subsection{Validación Funcional y Técnica} El proyecto comprende la  validación
del comportamiento de la Web App mediante pruebas de funcionamiento, análisis de
desempeño y revisión del comportamiento de los modelos de IA. Esto  asegura  que
el   sistema   cumpla   con   los    criterios    de    calidad    establecidos.
\subsection{Implementación  Final   del   Sistema}   El   alcance   incluye   la
consolidación de todas las funcionalidades definitivas, la integración global de
los componentes, pruebas finales y la entrega de una versión estable, lista para
operar. \subsection{Generación de Informes y Documentación Técnica}  Se  incluye
la elaboración de documentación técnica y metodológica  relevante,  informes  de
análisis, justificaciones de  decisiones  de  diseño  y documentación
en actas estructuradas del feedback del usuario final, con el fin  de  garantizar  claridad,  trazabilidad  y  soporte  para
futuras mejoras.


\section{Limitaciones del proyecto}

\subsection{No se realizará integración con sensores físicos ni  con  redes  WSN
    reales.} El sistema operará exclusivamente con datos de  entrada  proporcionados
por el usuario o mediante bases de conocimiento de referencia. \subsection{No se implementará
    una red LPWAN operativa.} El análisis comparativo será conceptual  y  basado  en
literatura técnica,  métricas  estimadas  y  modelos  analíticos  derivados  del
contexto  ingresado.  \subsection{El  sistema  no  ejecutará   simulaciones   de
    propagación, consumo energético o comportamiento físico de dispositivos IoT.} La
aplicación  procesará  información  contextual  para   ofrecer   recomendaciones
fundamentadas,  mas  no   simulaciones   numéricas   avanzadas.   \subsection{La
    inteligencia artificial no generará predicciones cuantitativas de comportamiento
    climático  o  agrícola.}  Su  función   se   limitará   a   la   identificación,
interpretación  y  estructuración  de  variables  relevantes  para  la  toma  de
decisiones tecnológicas. \subsection{Los resultados dependen  de  la  calidad  y
    precisión del contexto ingresado por el usuario.} Ambigüedades,  inconsistencias
o información insuficiente pueden afectar la pertinencia de las recomendaciones.
\subsection{Uso de planes gratuitos y APIs con acceso limitado} El desarrollo  y
despliegue del sistema se realizará utilizando  planes  gratuitos  de  servicios
externos,  tales  como  APIs  de  inteligencia  artificial  y   plataformas   de
despliegue,  lo  que  puede  imponer  restricciones  en  rendimiento,  escalabilidad   y
disponibilidad debido a las limitaciones presupuestales.

\section{Productos y Resultados de Investigación}

A continuación se presentan los siguientes productos principales que resultarán del proyecto de investigación.

\begin{table}[H]
    \centering
    \small % Tamaño de fuente ligeramente menor para mejorar el ajuste
    \begin{tabularx}{\textwidth}{@{} >{\raggedright\arraybackslash}p{4cm} >{\raggedright\arraybackslash}X c @{}}
        \toprule
        \textbf{Producto}                                   & \textbf{Descripción}                                                                                                                                                                                                & \textbf{Cantidad} \\
        \midrule
        Documento final de investigación (Trabajo de grado) & Informe que consolida todo el proceso investigativo: marco teórico, metodología, resultados, conclusiones, recomendaciones y trabajo futuro. Constituye el requisito para optar al título de Ingeniero de Sistemas. & 1                 \\
        \addlinespace
        Plataforma web                                     & Plataforma funcional para el análisis y simulación visual de arquitecturas de redes LPWAN con agentes de IA generativa, orientada a la toma de decisiones en agricultura de precisión.                              & 1                 \\
        \addlinespace
        Póster científico                                   & Material gráfico de divulgación que sintetiza los objetivos, metodología, resultados y conclusiones del proyecto para eventos académicos.                                                                           & 1                 \\
        \addlinespace
        Evento científico con componente de apropiación     & Participación en congresos o jornadas académicas para socializar los resultados y promover la apropiación social del conocimiento.                                                                                  & 1                 \\
        \bottomrule
    \end{tabularx}
    \caption{Productos y resultados de investigación esperados. Elaboración propia.}
    \label{tab:productos_completos}
\end{table}