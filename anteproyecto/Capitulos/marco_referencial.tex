
\chapter{Marco Referencial}

\section{Marco Teórico}

\subsection{Agricultura 4.0}

\subsubsection{Contexto y relevancia para Colombia}
La Agricultura 4.0 constituye la integración de  tecnologías  emergentes  —tales
como el Internet de las Cosas (IoT), analítica avanzada de datos, automatización
e inteligencia  artificial—  en  los  sistemas  productivos  agrícolas,  con  el
propósito de optimizar procedimientos, incrementar la productividad y fortalecer
la sostenibilidad de la actividad agrícola \parencite[]{shafi2019precision}.  En
el contexto colombiano, este enfoque adquiere una especial importancia debido al
peso estratégico del sector rural y a la diversidad ecológica  y  climática  que
caracteriza los sistemas de producción del país.

No obstante, la literatura evidencia la existencia de rezagos  estructurales  en
materia de  adopción  tecnológica.  Únicamente  el  1.7\%  de  las  Unidades  de
Producción Agrícola (UPA) cuentan con conectividad a Internet y  solo  el  6.6\%
poseen activos de Tecnologías de la Información y las  Comunicaciones,  lo  cual
supone una limitación crítica para el despliegue de  infraestructuras  digitales
aplicables a esquemas de Agricultura 4.0 \parencite[]{alvarez2022colombian}

Asimismo, un reporte del año 2021 de la corporación colombiana de  investigación
agropecuaria   (AGROSAVIA)   sostiene   que   la   transición   hacia   sistemas
agroindustriales inteligentes requiere  no  solo  inversión  en  infraestructura
tecnológica, sino también el fortalecimiento de capacidades  humanas,  así  como
procesos  efectivos  de  transferencia  de  conocimiento   técnico   hacia   los
productores rurales \parencite[]{florez2021agroindustria}.

De este modo, la Agricultura 4.0 en Colombia se configura no únicamente como una
transformación técnica, sino  como  un  proceso  socio-tecnológico  que  implica
cambios en la organización productiva, la  formación  de  capital  humano  y  la
gobernanza del conocimiento agrícola.

\subsubsection{Retos: brecha digital e ineficiencias}
A  pesar  de  las  ventajas  potenciales  que  supone  la  Agricultura  4.0,  su
implementación enfrenta múltiples desafíos. Uno de los más significativos es  la
denominada brecha digital rural, entendida como la desigualdad en  el  acceso  a
tecnologías digitales entre agricultores  tecnificados  y  pequeños  productores
tradicionales.  Tovar-Quiroz  (2023)  evidencia  que  esta  brecha  no  solo  se
manifiesta en infraestructura, sino también en alfabetización digital, acceso  a
plataformas inteligentes y capacidad  de  inversión  en  innovación  tecnológica
\parencite[]{quiroz2023agricultura}.

A ello  se  suma  la  baja  disponibilidad  de  conectividad  en  zonas  rurales
colombianas,  condición  imprescindible  para  soportar  sistemas  de  monitoreo
remoto,  transmisión  de  datos  sensorados  y  comunicación  distribuida  entre
dispositivos IoT \parencite[]{alvarez2022colombian}.

AGROSAVIA advierte además que la adopción acelerada de tecnologías  inteligentes
podría agravar desigualdades existentes  si  no  se  acompaña  de  políticas  de
democratización   tecnológica   y   apropiación    social    del    conocimiento
\parencite[]{florez2021agroindustria}.

Finalmente,  en  términos  operativos,  la  integración  de  modelos  de   datos
provenientes  de  sensores,  imágenes  satelitales  y  simulaciones  agronómicas
plantea grandes retos para la gestión eficiente de  información  y  la  toma  de
decisiones automatizada. Tovar-Quiroz (2023) identifica diez dominios en los que
convergen  estas  tecnologías,  destacando   la   necesidad   de   arquitecturas
interoperables para lograr la integración  funcional  de  múltiples  fuentes  de
datos \parencite[]{quiroz2023agricultura}.

\subsubsection{El rol del IoT y la Agricultura de Precisión}
La Agricultura de Precisión se constituye como un fundamento metodológico de  la
Agricultura 4.0. Esta se define como una estrategia  de  gestión  basada  en  la
captura y análisis de datos espacio-temporales, con  el  objetivo  de  optimizar
decisiones relativas al uso de insumos, manejo de suelos y gestión  de  cultivos
\parencite[]{quiroz2023agricultura}.

En  este  marco,  el  IoT  emerge  como  un  componente  clave  y  se  presentan
arquitecturas IoT aplicables a la  agricultura  colombiana,  donde  se  integran
sensores ambientales, redes de comunicación de baja potencia  y  plataformas  de
visualización y análisis de datos \parencite[]{Cognitive-LPWAN}

Por  su  parte,  AGROSAVIA  enfatiza  que  la  incorporación  de   sistemas   de
sensorización y teledetección puede transformar los  procesos  tradicionales  de
cultivo en sistemas de control inteligente, combinando el conocimiento  empírico
local       con        modelos        predictivos        y        automatización
\parencite[]{florez2021agroindustria}.

En consecuencia, tanto la Agricultura de Precisión como las infraestructuras IoT
proveen las bases tecno-operativas para evolucionar hacia modelos agrícolas  más
robustos, sostenibles y basados en evidencia cuantificable. Este enfoque integra
los datos provenientes del entorno físico con modelos de decisión computacional,
reduciendo la incertidumbre operativa y permitiendo una mayor eficiencia  en  el
uso de recursos críticos como agua, fertilizantes y energía.


\subsection{Redes LPWAN: Tecnologías Habilitadoras de la Agricultura de
      Precisión}

\subsubsection{Fundamentos: Toppologías, arquitectura y objetivos de diseño}
Las Redes de Área Amplia de Baja Potencia (LPWAN,  por  sus  siglas  en  inglés)
constituyen una categoría de tecnologías de  comunicación  inalámbrica  diseñada
para resolver el compromiso fundamental en la conectividad del Internet  de  las
Cosas (IoT): la necesidad de operar con bajo consumo de energía y largo  alcance
\parencite[]{diane2025systematic}. Este equilibrio es crucial para el despliegue
de  dispositivos  alimentados   por   batería   que   deben   transmitir   datos
intermitentemente    a    lo    largo    de    extensas    áreas     geográficas
\parencite[]{chen2018cognitive}.

Los objetivos de diseño  de  las  redes  LPWAN  están  orientados  a  la  máxima
eficiencia y efectividad:
\begin{itemize}
      \item \textbf{Largo rango de transmisión:} Las tecnologías LPWAN están
            diseñadas para ofrecer comunicaciones de largo  alcance,  alcanzando
            varios kilómetros en zonas rurales y entre 2  y  5  km  en  entornos
            urbanos, especialmente aquellas basadas en  infraestructura  celular
            como NB-IoT. Muchas de estas tecnologías operan en  la  banda  Sub-1
            GHz, cuyas características permiten  una  mayor  penetración  de  la
            señal, menor  atenuación  y  menor  efecto  de  desvanecimiento  por
            multitrayectoria,  además  de  presentar  menos  interferencias   en
            comparación con la banda  de  2.4  GHz.  Tecnologías  como  LoRaWAN,
            SigFox, IEEE 802.11ah, IEEE 802.15.4g, Weightless y DASH7 aprovechan
            estas ventajas, mientras que RPMA opera  en  2.4  GHz.  El  tipo  de
            modulación es  un  factor  clave  en  el  alcance  de  comunicación,
            predominando la modulación de banda estrecha, como Ultra  Narrowband
            (UNB), que permite transmisiones de varios kilómetros con bajo nivel
            de ruido, y la modulación de  espectro  ensanchado,  que  mejora  la
            resistencia a interferencias mediante técnicas  como  DSSS,  FHSS  y
            CSS,    empleadas    por    tecnologías    como    LoRa    y    RPMA
            \parencite[]{diane2025systematic}.
      \item \textbf{Eficiencia Energética:} En las redes LPWAN, el consumo
            energético de los nodos depende principalmente de  la  topología  de
            red, los ciclos  de  operación  y  los  protocolos  de  comunicación
            empleados. Dado que los nodos suelen estar alimentados por  baterías
            y se despliegan a gran escala, es fundamental maximizar su vida útil
            para reducir los costos de mantenimiento. Por esta razón, las  LPWAN
            priorizan la topología en estrella  sobre  la  topología  en  malla,
            evitando  el  sobreconsumo  energético  de  los  nodos  intermedios.
            Además, el uso del duty cycle permite  disminuir  significativamente
            el consumo de energía al  mantener  los  nodos  en  estado  inactivo
            durante  ciertos  intervalos,  aunque  este  mecanismo  puede  estar
            limitado por regulaciones del espectro.  Finalmente,  el  empleo  de
            protocolos  de  acceso  al  medio  simplificados,  como   ALOHA   en
            tecnologías  como  LoRaWAN  y  SigFox,  contribuye  a   reducir   la
            complejidad de los dispositivos y su  consumo  energético,  mientras
            que NB-IoT utiliza mecanismos de acceso aleatorio propios  de  redes
            celulares \parencite[]{diane2025systematic}.
      \item \textbf{Escalabilidad:} La escalabilidad constituye un aspecto
            fundamental en las LPWAN, ya que numerosos escenarios de  aplicación
            demandan la capacidad de gestionar cientos de miles de  dispositivos
            de manera simultánea. Este concepto hace referencia a  la  habilidad
            de la red para conectar una gran cantidad de nodos  sin  afectar  la
            calidad ni la continuidad de los servicios ofrecidos.  No  obstante,
            la coexistencia de un elevado número de dispositivos puede  provocar
            interferencias, impactando de forma negativa el desempeño de la red.
            Para enfrentar estos desafíos, se implementan distintas  estrategias
            orientadas a mejorar la escalabilidad, tales como la comunicación  a
            través de múltiples canales, la instalación de varios gateways y los
            mecanismos de adaptación dinámica de la velocidad de transmisión  de
            datos \parencite[]{diane2025systematic}.

      \item \textbf{Bajo costo:} El creciente interés por las tecnologías LPWAN
            se debe, en gran medida, al bajo  costo  de  los  dispositivos.  Por
            ejemplo, equipos basados en LoRa o SigFox pueden adquirirse  por  un
            valor aproximado de 3 a 5 dólares.  Para  reducir  la  inversión  de
            capital, estas tecnologías emplean diversas estrategias, entre ellas
            el uso de transceptores con formas de onda menos complejas,  lo  que
            permite  disminuir  el  tamaño  del  hardware,  la  tasa  máxima  de
            transmisión y los requerimientos  de  memoria.  Como  resultado,  se
            reduce  la  complejidad  del  diseño  y,  por  ende,  el  costo   de
            fabricación.\\

            Adicionalmente, una estación base puede gestionar decenas  de  miles
            de  dispositivos  finales  distribuidos  en  un   área   de   varios
            kilómetros, lo que contribuye a una disminución significativa de los
            costos operativos para los proveedores de red. Asimismo,  las  LPWAN
            operan tanto en bandas no licenciadas,  como  la  banda  ISM  o  los
            espacios  en  blanco  de  televisión,  como  en  bandas  licenciadas
            propiedad  de  los  operadores,  evitando  así  costos   adicionales
            asociados     al     uso      del      espectro      radioeléctricos
            \parencite[]{diane2025systematic}.
      \item \textbf{Calidad del servicio} El creciente interés por las
            tecnologías LPWAN se debe, en gran medida,  al  bajo  costo  de  los
            dispositivos. Por ejemplo, equipos basados en LoRa o  SigFox  pueden
            adquirirse por un valor aproximado de 3 a 5 dólares. Para reducir la
            inversión   de   capital,   estas   tecnologías   emplean   diversas
            estrategias, entre ellas el uso de transceptores con formas de  onda
            menos complejas, lo que permite disminuir el tamaño del hardware, la
            tasa máxima de transmisión y los  requerimientos  de  memoria.  Como
            resultado, se reduce la complejidad del diseño y, por ende, el costo
            de fabricación.\\

            Adicionalmente, una estación base puede gestionar decenas  de  miles
            de  dispositivos  finales  distribuidos  en  un   área   de   varios
            kilómetros, lo que contribuye a una disminución significativa de los
            costos operativos para los proveedores de red. Asimismo,  las  LPWAN
            operan tanto en bandas no licenciadas,  como  la  banda  ISM  o  los
            espacios  en  blanco  de  televisión,  como  en  bandas  licenciadas
            propiedad  de  los  operadores,  evitando  así  costos   adicionales
            asociados     al     uso      del      espectro      radioeléctricos
            \parencite[]{diane2025systematic}.
      \item \textbf{Gestión de la interferencia:} El creciente interés por las
            tecnologías LPWAN se debe, en gran medida,  al  bajo  costo  de  los
            dispositivos. Por ejemplo, equipos basados en LoRa o  SigFox  pueden
            adquirirse por un valor aproximado de 3 a 5 dólares. Para reducir la
            inversión   de   capital,   estas   tecnologías   emplean   diversas
            estrategias, entre ellas el uso de transceptores con formas de  onda
            menos complejas, lo que permite disminuir el tamaño del hardware, la
            tasa máxima de transmisión y los  requerimientos  de  memoria.  Como
            resultado, se reduce la complejidad del diseño y, por ende, el costo
            de fabricación.\\

            Adicionalmente, una estación base puede gestionar decenas  de  miles
            de  dispositivos  finales  distribuidos  en  un   área   de   varios
            kilómetros, lo que contribuye a una disminución significativa de los
            costos operativos para los proveedores de red. Asimismo,  las  LPWAN
            operan tanto en bandas no licenciadas,  como  la  banda  ISM  o  los
            espacios  en  blanco  de  televisión,  como  en  bandas  licenciadas
            propiedad  de  los  operadores,  evitando  así  costos   adicionales
            asociados     al     uso      del      espectro      radioeléctricos
            \parencite[]{diane2025systematic}.
\end{itemize}




\subsubsection{Análisis Comparativo de Topologías y Arquitecturas LPWAN}
El funcionamiento de las tecnologías LPWAN se fundamenta  en  los  objetivos  de
diseño discutidos previamente y en las diferentes estrategias existentes para su
implementación. En este apartado  se  analizan  las  distintas  arquitecturas  y
topologías LPWAN, así como los aspectos relacionados con su interoperabilidad.

\item \textbf{Topologías} Los dos tipos principales de topología presentes en el
espectro LPWAN son la topología en  malla  y  la  topología  en  estrella,
siendo esta última la más utilizada frente a  las  estructuras  en  malla,
debido a  su  valor  agregado  en  términos  de  eficiencia  energética  y
ampliación del rango de cobertura \parencite[]{chilamkurthy2022low}.

\begin{itemize}
      \item \textbf{Topología de estrella:} La topología de estrella consiste en
            una red de tipo punto a punto (P2P, por sus siglas en inglés), en la
            cual todos los nodos se  conectan  a  un  nodo  central,  denominado
            \textit{hub} o \textit{gateway}. Este gateway actúa  como  el  único
            medio de comunicación entre los nodos de la topología, enrutando los
            mensajes hacia un servidor central, donde se gestionan aspectos como
            la redundancia, la detección de fallos y la seguridad. Este  enfoque
            es ampliamente utilizado en aplicaciones de monitoreo y en  entornos
            peligrosos, donde el  despliegue  de  cableado  representa  un  alto
            riesgo \parencite[]{chilamkurthy2022low}.\\

            Entre las principales ventajas de una estructura basada en un  único
            hub se encuentran la alta velocidad de transmisión de  mensajes,  la
            escalabilidad — ya que es posible  añadir  nuevos  nodos  de  manera
            sencilla mediante su conexión  directa  al  gateway—  y  el  impacto
            limitado ante fallos a nivel de nodo, puesto que la  desconexión  de
            un nodo final no afecta al funcionamiento del resto de  la  red.  No
            obstante, esta topología presenta desventajas  significativas,  como
            la existencia de un único punto  de  fallo;  si  el  gateway  o  hub
            central deja de funcionar, toda la red se vuelve  inoperable  y  los
            nodos       finales       dejan       de       ser        accesibles
            \parencite[]{chilamkurthy2022low}.

      \item \textbf{Topología de malla:} Esta topología está compuesta por tres
            tipos  de  nodos:  nodos  sensores,  gateways   y   nodos   sensores
            enrutadores. En una topología de malla completamente conectada, cada
            nodo se comunica directamente con todos los demás nodos de  la  red.
            En contraste, en una topología de malla parcial, solo algunos  nodos
            mantienen múltiples conexiones,  mientras  que  otros  se  comunican
            únicamente con aquellos nodos con los que  intercambian  información
            de manera frecuente \parencite[]{chilamkurthy2022low}.\\

            Entre las ventajas de la topología de malla  se  destaca  su  diseño
            redundante, el cual permite la existencia de rutas alternativas para
            la transmisión de datos, mitigando el problema del  punto  único  de
            fallo presente en la topología en estrella. Asimismo, este  tipo  de
            topología  soporta  el  intercambio  de  datos   bajo   un   enfoque
            \textit{full-duplex}  (FD),  lo  que   contribuye   a   mejorar   la
            escalabilidad de la red. Sin embargo, también  presenta  desventajas
            importantes, como el aumento de la complejidad debido a la presencia
            de múltiples enlaces entre  nodos,  el  incremento  de  la  latencia
            ocasionado por  la  comunicación  multi-salto,  el  mayor  costo  de
            implementación y una reducción en la eficiencia energética  derivada
            de su diseño redundante \parencite[]{chilamkurthy2022low}.
\end{itemize}

\item \textbf{Arquitecturas} Los dos tipos principales de topología presentes en
el espectro LPWAN son la topología en malla y la  topología  en  estrella,
siendo esta última la más utilizada frente a  las  estructuras  en  malla,
debido a  su  valor  agregado  en  términos  de  eficiencia  energética  y
ampliación del rango de cobertura \parencite[]{chilamkurthy2022low}.

\begin{itemize}
\item \textbf{Compoentes principales de una arquitectura LPWAN} La arquitectura
típica de una  red  LPWAN  se  compone  de  varios  elementos  funcionales
claramente diferenciados:

\begin{enumerate}
\item \textbf{Dispositivos finales (End Devices o Nodos LPWAN):} Son los
encargados de capturar datos del entorno (sensores, actuadores) o ejecutar
las acciones según la aplicación. Operan con recursos limitados  (energía,
memoria  y  capacidad  de  cómputo)  y  se  comunican   mediante   enlaces
inalámbricos      de      largo      alcance      y      bajo      consumo
\parencite[]{chilamkurthy2022low}.
\item \textbf{Estaciones de acceso / gateways / concentradores:} Actúan como el
punto de enlace entre los dispositivos finales y la infrastrctura de  red.
Reciben los datos transmitidos por los nodos, garantizan la integridad del
enlace radioeléctrico (BER, seguridad, QoS) y reenvía la información hacia
el núcleo de la red. En algunas  tecnologías,  estos  dispositivos  pueden
incorporar  capcidades  de   edge   computing   y   almacenamiento   local
\parencite[]{chilamkurthy2022low}.
\item \textbf{Núcleo de red (Core o Network Server):} Es responsable del control
y enrutamiento del tráfico, la traducción de protocolos, la gestión de  la
movilidad, el control de admisión y,  en  ciertos  casos,  el  tratamiento
prioritario  de  los  datos,  También  cumple   un   rpñ   clave   en   la
interoperabilidad ente tecnologías y en la descarga de procesamiento hacia
el borde \parencite[]{chilamkurthy2022low}.
\item \textbf{Servidores de aplicación y nube:} Se encargan del almacenamiento,
procesamiento  a  análisis  de  los  datos  recolectados,  apoyándose   en
plataformas  de  cloud  y  técnicas  de  big  data.  Además,  permiten  la
visualización,  exportación  de  la  información  y  la  integración   con
aplicaciones       de        IOT        y        servicios        externos
\parencite[]{chilamkurthy2022low}.

\end{itemize}

\item \textbf{Tipos de Arquitectura LPWAN} Las LPWAN pueden organizarse bajo
diferentes enfoques arquitectónicos,  dependiendo  de  la  tecnología,  el
entorno de despliegue y los requerimientos de la aplicación:

\begin{enumerate}
      \item  \textbf{Arquitectura básica o tradicional LPWAN:}
            Presenta conectividad directa entre los dispositivos  finales  y  el
            gateway   (topología   de   estrella).    Es    simple,    eficiente
            energéticamente y adecuada para aplicaciones  de  monitoreo  a  gran
            escala. La comunicación se realiza en un solo salto, lo  que  reduce
            la       complejidad       de        los        nodos        finales
            \parencite[]{chilamkurthy2022low}.


      \item \textbf{ Arquitecturas híbridas con tecnologías de acceso externas:}
            En este enfoque, tecnologías como ZigBee, Wi-Fi  u  otras  redes  de
            corto alcance proporcionan conectividad inicial a los  dispositivos,
            mientras que una LPWAN actúa como red de transporte hacia  la  nube.
            Este tipo de arquitectura es común  en  LPWAN  celulares,  donde  el
            gateway de la red local se integra  con  la  infraestructura  LPWAN,
            formando una arquitectura mixta \parencite[]{chilamkurthy2022low}.

      \item \textbf{Arquitecturas híbridas multi-LPWAN:} Utilizan múltiples
            tecnologías LPWAN (por ejemplo, LoRa y SigFox) de manera  simultánea
            para conectar diferentes tipos de dispositivos o  cubrir  áreas  con
            requisitos heterogéneos. Cada tecnología gestiona los  nodos  dentro
            de su zona de cobertura, y el núcleo  de  red  centraliza  funciones
            como autenticación, registro  reminder,  asignación  de  recursos  y
            control  del  tráfico.  Este  enfoque  es  especialmente  útil  para
            aplicaciones complejas  que  demandan  flexibilidad,  redundancia  y
            diversidad tecnológica \parencite[]{chilamkurthy2022low}.

      \item  \text{Arquitecturas LPWAN cognitivas:} Representan una evolución
            hacia arquitecturas  inteligentes,  apoyadas  en  IA  y  aprendizaje
            automático.  Permiten  la  coexistencia   e   interoperabilidad   de
            múltiples tecnologías LPWAN, optimizando  dinámicamente  el  uso  de
            recursos de red. Estas arquitecturas  son  clave  para  aplicaciones
            avanzadas como ciudades  inteligentes,  Green  IoT,  salud  digital,
            hogares       inteligentes        y        sistemas        autónomos
            \parencite[]{chilamkurthy2022low}.

\end{enumerate}
\end{itemize}



\subsubsection{Análisis Comparativo de Tecnologías LPWAN}
Las tecnologías de Red de Área Extensa  de  Baja  Potencia  (LPWAN)  se  dividen
principalmente en dos categorías según el espectro de frecuencia  que  utilizan:
banda con licencia (celular), gestionada por el  consorcio  3GPP,  y  banda  sin
licencia (ISM), que engloba a tecnologías propietarias y abiertas. Este análisis
compara  sus  características  técnicas  clave,  rendimiento  e  idoneidad  para
distintos objetivos de diseño IoT.

\begin{itemize}
      \item \textbf{Tecnologías en Banda con Licencia (3GPP):} Estas tecnologías
            operan en espectro licenciado (principalmente 700-900 MHz),  lo  que
            garantiza  una  comunicación  confiable  sin  interferencias,   pero
            conlleva costos de suscripción. Son ideales  para  aplicaciones  que
            requieren  alta  confiabilidad,  movilidad  y   cobertura   nacional
            \parencite[]{chilamkurthy2022low}.
            \begin{itemize}
                  \item \textbf{LTE-M (LTE Cat M1 / eMTC):} Basada en LTE,
                        ofrece la mayor tasa de datos (hasta  1  Mbps),  soporta
                        movilidad (handover) y voz (VoLTE). Es la  más  adecuada
                        para aplicaciones que requieren latencia baja (150 ms) y
                        una mayor cantidad de datos, como rastreo de  activos  y
                        wearables,   aunque   con    un    consumo    energético
                        moderado-alto.
                  \item \textbf{NB-IoT:} Optimizada para IoT, utiliza un ancho
                        de  banda  muy  estrecho  (200  kHz).  Destaca  por   su
                        excepcional \textbf{cobertura} (hasta 15 km, presupuesto
                        de enlace de  164  dB)  y  gran  \textbf{profundidad  de
                              penetración}, siendo óptima para sensores  estáticos  en
                        ubicaciones  remotas  o  subterráneas  (e.g.,  medidores
                        inteligentes). Su latencia es mayor (hasta 10 s).
                  \item \textbf{EC-GSM-IoT:} Una evolución de GSM/GPRS. Ofrece
                        un   equilibrio   entre   cobertura,   capacidad    (50k
                        dispositivos/celda)    y    coste,    aprovechando    la
                        infraestructura  GSM  existente.  Es  una  opción   para
                        modernizar redes M2M tradicionales.
            \end{itemize}

      \item \textbf{Tecnologías en Banda Sin Licencia (Non-3GPP):} Operan en
            bandas ISM sub-GHz (e.g., 868 MHz, 915 MHz) o 2.4 GHz, sin costos de
            espectro pero sujetas a restricciones  de  duty  cycle  y  potencial
            interferencia.   Permiten   el   despliegue   de   redes    privadas
            \parencite[]{chilamkurthy2022low}.
            \begin{itemize}
                  \item \textbf{SigFox:} Utiliza Banda Ultra Estrecha (UNB). Se
                        caracteriza  por   su   \textbf{extremo   bajo   costo},
                        \textbf{mayor    alcance}    (50     km     rural)     y
                        \textbf{minimización
                              del  consumo  energético}  (6  nA  en  sleep).  Es
                        adecuada para
                        aplicaciones  de  uplink  muy  esporádico  con  mensajes
                        diminutos (12 bytes), pero su escalabilidad de carga  es
                        limitada (140 mensajes/día) y la capacidad  de  downlink
                        es muy reducida.
                  \item \textbf{LoRaWAN:} Utiliza modulación CSS de espectro
                        ensanchado.   Ofrece   un    buen    equilibrio    entre
                        \textbf{alcance}   (15   km   rural),    \textbf{consumo
                              energético} y \textbf{flexibilidad}  (diferentes  clases
                        de dispositivos A/B/C). Su arquitectura de red abierta y
                        la capacidad de desplegar gateways privados la hacen muy
                        popular  para  redes  IoT  corporativas  y  de  ciudades
                        inteligentes. La tasa de datos es baja (0.3-50  kbps)  y
                        depende del factor de ensanchamiento (SF).
                  \item \textbf{RPMA (Ingenu):} Opera en la banda de 2.4 GHz, lo
                        que  le  otorga  un  gran  ancho   de   banda   y   alta
                        \textbf{escalabilidad estructural}. Ofrece la mayor tasa
                        de datos entre las tecnologías sin  licencia  (624  kbps
                        uplink). Sin embargo, su  alcance  es  limitado  (10  km
                        rural) debido a la mayor atenuación de la  frecuencia  y
                        sufre  interferencia  de   otras   tecnologías   (Wi-Fi,
                        Bluetooth).
                  \item \textbf{Telensa:} Tecnología UNB especializada en
                        aplicaciones   de   control   como   alumbrado   público
                        inteligente.    Ofrece    comunicación     bidireccional
                        full-duplex y una vida útil de batería de ~8 años,  pero
                        con un alcance modesto (4  km  rural)  y  baja  tasa  de
                        datos.
                  \item \textbf{Weightless:} Conjunto de estándares abiertos.
                        \textbf{Weightless-P} es  el  más  completo,  ofreciendo
                        comunicación      bidireccional      confiable       con
                        acknowledgments,  buena  gestión  de  interferencias   y
                        movilidad, comparable a una versión ligera de  LTE  para
                        IoT.
                  \item \textbf{DASH7 (D7AP):} Protocolo derivado de RFID
                        activo.  Se  destaca  por  soportar  \textbf{movilidad},
                        comunicación    \textbf{asíncrona}    y     \textbf{baja
                              latencia},  ideal  para  activos  en   movimiento   como
                        logística. Su principio  de  diseño  es  BLAST  (Bursty,
                        Light, Asynchronous, Stealth, Transitional).
                  \item \textbf{NB-Fi:} Enfocada en lograr la \textbf{máxima
                              cobertura y penetración} (hasta 30 km, presupuesto
                        de enlace
                        de 174 dB) en banda sin licencia, con una tasa de  datos
                        mínima (11 bps). Es adecuada para aplicaciones donde  el
                        único requisito es recibir una  señal  de  sensores  muy
                        remotos.
            \end{itemize}

\end{itemize}

\item \textbf{Comparativa Sintética por Objetivos de Diseño:} La elección de la
tecnología óptima implica  negociar  entre  distintos  objetivos,  ya  que
ninguna domina en todos los aspectos.

\begin{itemize}
      \item \textbf{Eficiencia Energética y Vida Útil:} Las tecnologías
            \textbf{sin licencia}  (especialmente  SigFox,  LoRaWAN,  D7AP)  son
            generalmente superiores. SigFox es líder absoluto en consumo en modo
            sleep. Las tecnologías celulares (NB-IoT, LTE-M)  implementan  modos
            de ahorro (PSM, eDRX) para alcanzar vidas de ~10 años.
      \item \textbf{Costo Total:} Las tecnologías \textbf{sin licencia} tienen
            ventaja  al  eliminar  tarifas  de  suscripción  y  permitir   redes
            privadas. SigFox y NB-Fi ofrecen costos operativos  muy  bajos.  Las
            tecnologías \textbf{con licencia} implican un costo recurrente, pero
            aprovechan infraestructura existente.
      \item \textbf{Cobertura y Penetración:} \textbf{NB-IoT} lidera en la banda
            con  licencia.  En  la  banda  sin   licencia,   \textbf{SigFox}   y
            \textbf{NB-Fi} ofrecen los mayores alcances,  mientras  que  D7AP  y
            Weightless-P tienen un alcance más corto.
      \item \textbf{Escalabilidad:}
            \begin{itemize}
                  \item \textit{Escalabilidad Estructural (dispositivos por
                              celda):} Superior en tecnologías  \textbf{sin  licencia}
                        como  D7AP,  Weightless-P,  LoRaWAN  y   SigFox   (hasta
                        millones). Las tecnologías celulares manejan decenas  de
                        miles (NB-IoT: ~50k, LTE-M: ~80k-1M).
                  \item \textit{Escalabilidad de Carga (mensajes por
                              dispositivo):}  Superior  en   tecnologías   \textbf{con
                              licencia}, al no tener restricciones de duty cycle.  Las
                        tecnologías sin licencia están limitadas por  regulación
                        (e.g., 140 mensajes/día en SigFox, duty cycle del 1%  en
                        EU para LoRa).
            \end{itemize}
      \item \textbf{Manejo de Interferencias:} Las tecnologías \textbf{con
                  licencia} están libres de interferencias  accidentales.  Entre
            las sin
            licencia, \textbf{LoRaWAN} (CSS) y \textbf{Weightless-P} (saltos  de
            frecuencia + FEC) son muy robustas. Las que usan UNB (SigFox, NB-Fi)
            también minimizan el riesgo.
      \item \textbf{Calidad de Servicio (QoS), Movilidad y Latencia:} Las
            tecnologías \textbf{con licencia}  (especialmente  LTE-M  y  NB-IoT)
            ofrecen la mejor QoS garantizada, soporte nativo para handover y las
            latencias más bajas. Entre las sin licencia, \textbf{Weightless-P} y
            \textbf{D7AP} son las que mejor soportan movilidad y  comunicaciones
            bidireccionales confiables.
\end{itemize}

\item \textbf{Discusión y Aplicabilidad:} No existe una tecnología LPWAN única
óptima para todos los casos. La selección debe basarse en  los  requisitos
específicos de la aplicación:
\begin{itemize}
      \item \textbf{Aplicaciones de Monitoreo Masivo y Estático} (e.g.,
            medidores, agricultura): \textbf{NB-IoT} (por cobertura/penetración)
            o \textbf{SigFox/LoRaWAN} (por coste y energía).
      \item \textbf{Aplicaciones con Movilidad o Baja Latencia} (e.g.,
            logística,  wearables):  \textbf{LTE-M}  (mejor  opción  celular)  o
            \textbf{D7AP/Weightless-P} (en redes privadas).
      \item \textbf{Redes Privadas y Control Industrial} (e.g., ciudades
            inteligentes,   fabricas):    \textbf{LoRaWAN}    (flexibilidad    y
            ecosistema) o \textbf{Weightless-P} (QoS y confiabilidad).
      \item \textbf{Aplicaciones con Datos más Voluminosos o Voz} (e.g.,
            vigilancia,   teleasistencia):   \textbf{LTE-M}   es    la    opción
            predominante.
\end{itemize}
La tendencia futura apunta hacia la convergencia  y  coexistencia  de  múltiples
tecnologías en arquitecturas híbridas, donde una plataforma de gestión unificada
pueda  seleccionar  dinámicamente  la  mejor  conexión  disponible   para   cada
dispositivo y aplicación \parencite[][]{}.



\subsubsection{Aplicación en el Monitoreo Agrícola}
Las LPWAN proporcionan la \textbf{base tecnológica} sobre la cual  se  sustentan
los sistemas de Agricultura de Precisión. Su aplicación en el monitoreo agrícola
es  crucial  por  las  siguientes  razones,  que  justifican  la  necesidad   de
seleccionar la tecnología adecuada:

\begin{itemize}
      \item \textbf{Viabilidad en Campos Extensos:} La capacidad de largo
            alcance permite conectar sensores distribuidos en  grandes  parcelas
            de cultivo donde otras tecnologías son  inviables  u  extremadamente
            caras de implementar. Esto es fundamental para obtener datos de alta
            resolución espacial \parencite[]{chen2018cognitive}.
      \item \textbf{Sostenibilidad Operativa:} El bajo consumo energético
            garantiza que los sensores permanezcan operativos  durante  periodos
            prolongados  (años)  sin  intervención  humana   para   recargar   o
            reemplazar baterías, reduciendo  significativamente  los  costos  de
            mantenimiento y operación.
      \item \textbf{Insumo para la Inteligencia Artificial (IA):} Las LPWAN son
            el canal de comunicación que traslada los datos  críticos  de  campo
            (clima, suelo, estado hídrico) hacia  las  plataformas  de  IA,  las
            cuales procesan esta información para generar prescripciones
            agronómicas  \parencite[]{chen2018cognitive}.   La   selección
            errónea
            de la LPWAN puede resultar en fallas de conectividad, comprometiendo
            la  calidad  de  los  datos  y,  por  ende,  la  precisión  de   las
            recomendaciones del sistema inteligente.
\end{itemize}

\subsection{Sensórica y Variables Agronómicas}

La agricultura de precisión se fundamenta en la  captura  sistemática  de  datos
agronómicos mediante sensores, los cuales permiten  monitorear  las  condiciones
del cultivo y el ambiente en tiempo real o casi real. Esta capacidad de medición
precisa y continua resulta esencial para la toma de  decisiones  informadas,  la
optimización de insumos y la mitigación de riesgos asociados a  la  variabilidad
climática. En el contexto del  despliegue  de  redes  LPWAN  para  el  monitoreo
agrícola, la selección  de  la  sensórica  adecuada  y  la  comprensión  de  las
variables que  miden  constituyen  un  paso  crítico  previo  al  diseño  de  la
arquitectura de comunicación.

\subsubsection{Sensórica en la agricultura de presición}
Los sensores son dispositivos que  convierten  señales  del  mundo  físico  (por
ejemplo, humedad, temperatura, composición química) en datos digitales, actuando
como la interfaz primaria entre el campo y los sistemas de información (Applying
IoT Sensors and Big Data to  Improve  Precision  Crop.pdf).  Su  integración  en
componentes de maquinaria, suelo, plantas  o  animales  proporciona  información
vital sobre el  estado  del  sistema  agroproductivo  \parencite{alahmad2023applying}.

Para el desarrollo de sistemas IoT agrícolas (Ag-IoT), la selección  del  sensor
apropiado debe considerar factores como bajo consumo de energía,  compatibilidad
en la transferencia de información,  precisión,  sensibilidad,  repetibilidad  y
durabilidad \parencite{alahmad2023applying}. Existe una amplia variedad de sensores clasificables  según  el
parámetro físico que miden:

\begin{itemize}
      \item \textbf{Sensores Químicos:} Miden propiedades como el pH del suelo y
            agua, conductividad eléctrica, salinidad, y concentraciones de gases
            (CO₂, O₂, CH₄) y nutrientes (nitratos). Se dividen principalmente en
            fotodetectores y electroquímicos \parencite{alahmad2023applying}.
      \item \textbf{Sensores Ópticos:} Utilizan la reflectancia de la luz en
            diferentes longitudes de onda para determinar materia  orgánica  del
            suelo, humedad, color, contenido de  clorofila  en  plantas,  estrés
            hídrico y  detección  de  enfermedades  foliares.  Tecnologías  como
            cámaras multiespectrales, hiperespectrales y el Índice de Vegetación
            de Diferencia Normalizada (NDVI) son clave para la  teledetección  y
            estimación de rendimiento \parencite{alahmad2023applying}.
      \item \textbf{Sensores Electromecánicos y de Humedad:} Incluyen sensores
            de humedad del suelo (que miden la constante dieléctrica),  sensores
            de presión, acelerómetros y celdas de carga. Son fundamentales  para
            medir la compactación del suelo, el crecimiento de frutos, el viento
            y el peso continuo de las plantas \parencite{alahmad2023applying}.
      \item \textbf{Sensores Acústicos:} Detectan cambios en las frecuencias
            sonoras,  útiles  para  identificar  plagas  (como  barrenadores  de
            madera) mediante el sonido que generan al alimentarse o  moverse,  y
            para estimar la altura del dosel del cultivo \parencite{alahmad2023applying}.
      \item \textbf{Sensores Térmicos:} Monitorean la temperatura de hojas,
            suelo y ambiente. Los datos de temperatura foliar se  utilizan  para
            predecir la producción, estimar la evapotranspiración y programar el
            riego \parencite{alahmad2023applying}.
\end{itemize}

\subsubsection{ Variables Agronómicas Críticas para el Monitoreo}
Las variables agronómicas monitoreables definen el estado de salud del cultivo y
del suelo, guiando las intervenciones de manejo. La  cantidad  y  frecuencia  de
datos requeridos dependen del cultivo específico, las condiciones ambientales  y
los objetivos del productor \parencite{alahmad2023applying}. Las  variables  más  relevantes,  vinculadas  a  los
sensores antes descritos, incluyen:

\begin{itemize}
      \item \textbf{Variables Edáficas (del Suelo):}
            \begin{itemize}
                  \item \textbf{Humedad del  Suelo:  }Parámetro  fundamental  para  la
                        programación eficiente  del  riego.  Se  mide  comúnmente  con
                        sensores capacitivos o de resistencia eléctrica \parencite{alahmad2023applying}.
                  \item \textbf{Nutrientes (N, P, K) y pH:} Determinan  la  fertilidad
                        del  suelo  y  la   necesidad   de   fertilización.   Sensores
                        electroquímicos  y  ópticos  permiten  estimaciones  in  situ,
                        aunque su medición en tiempo real sigue siendo un  desafío  en
                        desarrollo \parencite{alahmad2023applying}.
                  \item \textbf{Temperatura  del  Suelo:  }Afecta  la  germinación  de
                        semillas y la  actividad  microbiana.  Se  mide  con  sensores
                        térmicos  \parencite{alahmad2023applying}.
                  \item \textbf{Conductividad Eléctrica (CE):} Indica la salinidad del
                        suelo y la  concentración  de  iones.  Se  mide  con  sensores
                        eléctricos \parencite{alahmad2023applying}.
            \end{itemize}
      \item \textbf{Variables Ambientales y Climáticas:}
            \begin{itemize}
                  \item \textbf{Temperatura  y  Humedad  del  Aire:}  Condicionan   la
                        evapotranspiración, la aparición de enfermedades y  el  estrés
                        térmico de los cultivos. Se monitorean con sensores térmicos y
                        de humedad (capacitivos o resistivos)  \parencite{alahmad2023applying}.
                  \item \textbf{Precipitación y Viento: }Afectan  la  programación  de
                        riego, la aplicación de agroquímicos y los riesgos de erosión.
                        Se miden con pluviómetros y anemómetros (sensores mecánicos  o
                        acústicos) \parencite{alahmad2023applying}.
                  \item \textbf{Radiación  Solar:}  Determina  la  fotosíntesis  y  el
                        crecimiento. Se mide con sensores de radiación o  piranómetros
                        \parencite{alahmad2023applying}.

            \end{itemize}
      \item \textbf{Variables del Cultivo:}
            \begin{itemize}
                  \item \textbf{Salud  y  Vigor  Vegetal:  }Se  evalúa  mediante
                        índices espectrales (como NDVI) obtenidos  con  sensores
                        ópticos montados en drones, tractores o  satélites,  que
                        indican el contenido de clorofila y la cobertura  foliar
                        \parencite{alahmad2023applying}.
                  \item \textbf{Detección   de   Estrés   y   Enfermedades:    }
                        Combinaciones de sensores ópticos (para detectar cambios
                        de color o textura), térmicos (para  identificar  estrés
                        hídrico) y acústicos (para detectar plagas) permiten una
                        identificación temprana \parencite{alahmad2023applying}.
                  \item \textbf{Estado Fenológico y  Rendimiento:  }  Cámaras  y
                        sensores  de  flujo  de  masa  permiten  monitorear   el
                        desarrollo  del  cultivo  y  estimar  o  cuantificar  el
                        rendimiento  en   la   cosecha   \parencite{ pyingkodi2022sensor}
            \end{itemize}
\end{itemize}



\subsection{Simulación de Redes LPWAN}
\subsubsection{Estado del Arte de Herramientas de Simulación y sus Métricas de
      Desempeño}

El siguiente análisis se basa en el artículo: \textit{A Survey on LoRaWAN
      Technology: Recent Trends,  Opportunities,  Simulation  Tools  and  Future
      Directions} \parencite{almuhaya2022survey}, la simulación de redes LPWAN
(específicamente LoRaWAN) es crucial para evaluar el rendimiento  de  la  red  a
gran escala. A continuación, se detallan las herramientas de simulación tratadas
en el artículo y sus métricas de desempeño.

\begin{table}[H] centering
      \begin{tabularx}{\textwidth}{l>{\raggedright\arraybackslash}X>{\raggedright\arraybackslash}X>{\raggedright\arraybackslash}X}
            \toprule \textbf{Herramienta}                             & \textbf{Descripción y Plataforma}                                  &
            \textbf{Características  Clave}                           & \textbf{Métricas  de   Desempeño
            Evaluadas}                                                                                                                                                   \\ \midrule LoRaSim & Simulador de  eventos  discretos  y
            probabilístico implementado en Python.                    & Simula  un  único  gateway
            (puerta de enlace) LoRaWAN que sirve a varios  dispositivos  finales
            (EDs). Incorpora el efecto de captura y la asignación de  factor  de
            dispersión (SF)  y  potencia.                             & Packet  Reception  Ratio  (PRR)  y
            capacidad de la red (escalabilidad).                                                                                                                         \\ \addlinespace

            LoRaWANSim                                                & Módulo para el simulador de red ns-3 basado en  C++.               &
            Extiende el modelo ALOHA para incluir el  efecto  de  captura  y  la
            interferencia inter-SF. Soporta comunicación  multicanal,  multi-SF,
            multi-gateway y bidireccional. Soporta las Clases A y C de  LoRaWAN.
                                                                      & Consumo de energía, Packet Delivery Ratio (PDR) y latencia  de  la
            red.                                                                                                                                                         \\ \addlinespace

            FAD                                                       & Simulador de  eventos  discretos  implementado  en  Python.        &
            Enfocado en simulaciones de gran  escala.  Introduce  un  modelo  de
            canal LoRaWAN realista y considera el efecto de  captura. & PDR  y
            consumo de energía.                                                                                                                                          \\ \addlinespace

            Simu-LoRa                                                 & Implementado en  MATLAB.                                           & Diseñado  para  evaluar  el
            rendimiento de los dispositivos finales  y  diferentes  factores  de
            dispersión (SF). Se centra en la capa  física  (PHY).     & Eficiencia
            energética y PDR.                                                                                                                                            \\ \bottomrule
      \end{tabularx}
      \caption{Comparativa de herramientas de simulación para redes LoRaWAN}
      \label{tab:lorawan_tools}
\end{table}


\subsection{Agentes de Inteligencia Artificial}

\subsubsection{Definición y diferencia con la IA generativa tradicional}

Un agente de Inteligencia Artificial (IA) se define como una entidad que percibe
su entorno y actúa dentro de él de forma autónoma con el objetivo de lograr  sus
metas  \parencite{artint3e}.  Los  agentes  son  una  pieza  fundamental  de  la
arquitectura de los sistemas de IA\parencite{gutowska2025aiagents}.

La diferencia con la IA generativa tradicional  (como  los  grandes  modelos  de
lenguaje) radica en que un agente de IA no solo genera contenido  o  respuestas,
sino que también incluye capacidades de percepción, razonamiento y acción en  un
entorno para alcanzar un objetivo  específico  \parencite{gutowska2025aiagents}.
La IA generativa puede ser un componente que el agente utiliza  para  razonar  o
planificar.

\subsubsection{Características clave: autonomía, capacidad de decisión,
      autoaprendizaje y adaptación}

Las características fundamentales de los agentes computacionales incluyen:

\begin{itemize}
      \item \textbf{Autonomía:} La capacidad de operar sin la necesidad de un
            control humano o de la interacción directa en cada paso.  Un  agente
            de  IA  puede  ejecutar  tareas  y   tomar   decisiones   de   forma
            independiente \parencite{gutowska2025aiagents}.
      \item \textbf{Capacidad de Decisión:} Un agente posee un espacio de diseño
            de  agente  que  le  permite  tomar  decisiones   basadas   en   sus
            percepciones y metas. Estos diseños progresan gradualmente desde  lo
            simple hasta lo complejo \parencite{artint3e}.
      \item \textbf{Autoaprendizaje y Adaptación:} Aunque no se detalla
            explícitamente  el   ``autoaprendizaje''   o   ``adaptación''   como
            términos, el libro  sobre  Fundamentos  de  Agentes  Computacionales
            aborda  cómo  la  IA  moderna  se  explica  a  través   de   agentes
            computacionales, y cómo  las  nuevas  ediciones  incluyen  capítulos
            sobre aprendizaje profundo (deep learning) e IA generativa,  lo  que
            implica  mecanismos   de   mejora   y   ajuste   de   comportamiento
            \parencite{artint3e}.
\end{itemize}

\subsubsection{Arquitectura de sistemas multiagente y flujos de trabajo}

Un  Sistema  Multiagente   es   una   arquitectura   donde   múltiples   agentes
especializados     interactúan     para     resolver     problemas     complejos
\parencite{gutowska2025aiagents}.

La Arquitectura  de  Referencia  Multiagente  (MARA)  es  una  documentación  de
Microfosft que se enfoca en los desafíos únicos de orquestar, gobernar y escalar
sistemas donde varios agentes colaboran, en lugar  de  centrarse  en  un  agente
individual \parencite{multiagent_architecture}.

Los flujos de trabajo en estos sistemas implican:

\begin{itemize}
      \item \textbf{Orquestación:} Gestionar las interacciones de los agentes.
      \item \textbf{Gobernanza:} Establecer reglas y límites.
      \item \textbf{Escalabilidad:} Asegurar que el sistema pueda crecer y
            manejar más agentes o tareas \parencite{multiagent_architecture}.
\end{itemize}

\subsubsection{Aplicación en sistemas de recomendación y soporte a la decisión}

Los agentes de IA son la  base  de  los  agentes  computacionales,  un  concepto
central en la IA que se aplica a diversas áreas.

\begin{itemize}
      \item \textbf{Soporte a la Decisión:} El concepto de agentes se relaciona
            directamente  con  los  modelos  de  razonamiento  y  de  causalidad
            \parencite{artint3e}. Los agentes pueden ser diseñados para  razonar
            y evaluar la información, lo que es esencial  para  el  apoyo  a  la
            decisión.

      \item \textbf{Sistemas de Recomendación:} Al igual que en el soporte a la
            decisión,  los  agentes  se  utilizan  para  aplicar   técnicas   de
            aprendizaje profundo y razonamiento a grandes  volúmenes  de  datos,
            una  tarea  habitual  en  los  sistemas  de  recomendación  modernos
            \parencite{artint3e}.
\end{itemize}

\subsubsection{Gobernanza y límites de los agentes autónomos}

La \textbf{gobernanza} es un componente clave en la Arquitectura  de  Referencia
Multiagente. El objetivo es establecer el marco para la orquestación  y  gestión
de los agentes \parencite{multiagent_architecture}.

Los límites y desafíos de estos sistemas incluyen:

\begin{itemize}
      \item \textbf{Complejidad:} La necesidad de diseñar el sistema para el
            cambio,  equilibrando  la  extensibilidad  a  largo  plazo  con  una
            ingeniería       pragmática        para        el        lanzamiento
            \parencite{multiagent_architecture}.
      \item \textbf{Ética y Responsabilidad:} La IA, incluido el diseño de
            agentes, tiene un impacto  social  que  debe  ser  considerado.  Las
            características de autonomía y capacidad de decisión de los  agentes
            plantean desafíos éticos importantes \parencite{artint3e}.
\end{itemize}


\subsection{Metodologías Ágiles (Scrum + Kanban)}

\noindent\textit{La información presentada en esta sección se basa completamente
      en \parencite[]{grotenfelt2021agile}}

Las metodologías ágiles representan un conjunto de prácticas y marcos de trabajo
cuyo objetivo principal es la entrega rápida y continua  de  valor  al  cliente,
fomentando  la  adaptabilidad  y  la  respuesta  al  cambio.  El  desarrollo  de
\textit{software} ágil y su implementación se centran  en  el  estudio  de  cómo
estos enfoques pueden afectar el trabajo de un equipo de desarrollo. Dos de  los
marcos ágiles más populares son  Scrum  y  Kanban,  que  a  menudo  se  utilizan
conjuntamente.

\subsubsection{Marco de Scrum}

Scrum es uno de los marcos de  trabajo  ágiles  más  utilizados,  diseñado  para
implementar las prácticas y valores de la agilidad. Este marco  se  basa  en  la
división del trabajo en periodos de tiempo fijos y cortos llamados sprints.

Un sprint es un \textit{time-box} que tiene una duración  de  un  mes  o  menos,
durante el cual se crea un incremento  de  producto  potencialmente  utilizable,
integrado y completo. El sprint actúa como el contenedor para  todos  los  demás
eventos de Scrum.

\subsubsection{Roles de Scrum}

El equipo de Scrum está compuesto por tres roles principales:

\begin{itemize}
      \item Product Owner (PO): Responsable de maximizar el valor del producto y
            gestionar la pila del producto.
      \item Scrum Master (SM): Asegura que Scrum sea comprendido e implementado,
            y ayuda al equipo a eliminar impedimentos.
      \item Equipo de Desarrollo (Development Team): Profesionales encargados de
            crear el incremento de producto utilizable;  son  autoorganizados  y
            multifuncionales.
\end{itemize}

\subsubsection{Eventos de Scrum}

\begin{itemize}
      \item Planificación del Sprint (Sprint Planning)
      \item Scrum Diario (Daily Scrum)
      \item Revisión del Sprint (Sprint Review)
      \item Retrospectiva del Sprint (Sprint Retrospective)
\end{itemize}

\subsubsection{Framework de Kanban}

Kanban se centra en el flujo del trabajo, la  visualización  del  proceso  y  la
limitación del trabajo  en  curso  (WIP).  El  nombre  proviene  del  japonés  y
significa “señal visual”.

\subsubsection{Prácticas Clave}

\begin{itemize}
      \item Visualización del flujo de trabajo mediante el tablero Kanban.
      \item Limitación del WIP para evitar sobrecarga y mejorar la eficiencia
            del flujo.
\end{itemize}

\subsubsection{Métricas Ágiles}

Las métricas permiten inspeccionar  el  progreso  y  detectar  oportunidades  de
mejora.

\subsubsection{Métricas de Progreso y Alcance}

\begin{itemize}
      \item Gráfico burn-down: muestra la cantidad de trabajo restante en
            función del tiempo.
      \item Gráfico burn-up: indica el trabajo completado frente al alcance
            total.
\end{itemize}

\subsubsection{Métricas de Flujo y Eficiencia}

\begin{itemize}
      \item Tiempo de ciclo (cycle time): mide el tiempo desde que una tarea
            entra en estado “en progreso” hasta que se completa.
      \item Tiempo de espera (lead time): mide el tiempo total desde que se
            solicita una tarea hasta su entrega final.
\end{itemize}



\section{Marco Conceptual}

El marco conceptual establece los conceptos fundamentales dentro del contexto de
esta investigación, orientada al  diseño  y  evaluación  de  una  plataforma  de
software con IA basada en simulación  de  redes  LPWAN  y  modelos  de  decisión
agronómica para Agricultura de Precisión en entornos rurales. Este marco permite
delimitar el significado operativo de los términos clave y articularlos  en  una
estructura coherente que conecta la infraestructura tecnológica con la  toma  de
decisiones inteligentes en el sector agrícola.

\subsection{Agricultura 4.0}
Se entiende por \textbf{Agricultura  4.0}  el  paradigma  socio-tecnológico  que
integra tecnologías digitales avanzadas —Internet de las Cosas (IoT),  analítica
de datos, inteligencia artificial (IA), automatización y computación en la nube—
en los sistemas productivos agrícolas, con el objetivo de optimizar  el  uso  de
recursos, incrementar la productividad y mejorar la sostenibilidad ambiental.

En esta investigación, la Agricultura 4.0 se concibe como el \textbf{marco
      aplicativo} que da sentido al uso de redes LPWAN, la sensórica distribuida
y los
modelos  de  decisión  agronómica,  especialmente  en  contextos   rurales   con
limitaciones de conectividad.

\subsection{Agricultura de Precisión}
La \textbf{Agricultura de Precisión} es un enfoque operativo de  la  Agricultura
4.0 basado en la gestión diferencial de los cultivos  mediante  el  análisis  de
datos espacio-temporales. Su propósito es reducir la variabilidad  productiva  y
optimizar la aplicación de insumos (agua, fertilizantes y energía) en función de
las condiciones reales del suelo, el ambiente y el estado del cultivo.

Dentro  del  presente  trabajo,  la  Agricultura  de  Precisión  actúa  como  el
\textbf{dominio funcional} sobre el cual se definen las  variables  agronómicas,
la densidad de sensores y los requisitos de comunicación de la red.

\subsection{Internet de las Cosas (IoT) Agrícola}
El  \textbf{IoT  agrícola}  se  define  como  la  red  de  dispositivos  físicos
interconectados —sensores, actuadores y nodos inteligentes— que  capturan  datos
del entorno agrícola y los transmiten a plataformas  de  procesamiento  para  su
análisis y explotación.

En este estudio, el IoT agrícola se materializa mediante nodos sensores de  bajo
consumo energético conectados a través de redes LPWAN, actuando como la capa  de
adquisición de datos del sistema.

\subsection{Redes LPWAN}
Las \textbf{Redes de Área Amplia de Baja Potencia (LPWAN)}  son  tecnologías  de
comunicación inalámbrica diseñadas para  ofrecer  largo  alcance,  bajo  consumo
energético y bajo costo, permitiendo la  conectividad  de  dispositivos  IoT  en
entornos extensos y de difícil acceso.

Conceptualmente, las LPWAN constituyen la \textbf{infraestructura de
      comunicaciones} que habilita la transmisión de datos agronómicos desde  el
campo
hasta  las  plataformas  de  procesamiento,  siendo  un  factor  crítico  en  el
rendimiento del sistema de Agricultura de Precisión.

\subsection{Arquitectura LPWAN}
La \textbf{arquitectura LPWAN} se refiere a la  organización  funcional  de  los
componentes que conforman la red,  incluyendo  dispositivos  finales,  gateways,
núcleo de red y servidores de aplicación. Esta arquitectura define cómo fluye la
información, cómo se gestiona la escalabilidad y cómo se garantiza la eficiencia
energética y la calidad del servicio.

En el contexto de esta investigación, la arquitectura  LPWAN  es  el  objeto  de
evaluación  mediante  simulación,  considerando  diferentes  configuraciones   y
parámetros de diseño.

\subsection{Topología de Red}
La \textbf{topología de red} describe la forma en que los nodos se interconectan
dentro de una LPWAN, principalmente  bajo  esquemas  de  estrella  o  malla.  La
topología influye  directamente  en  el  consumo  energético,  la  latencia,  la
escalabilidad y la resiliencia de la red.

Este concepto es clave para analizar el impacto de la densidad de sensores y  el
número de gateways sobre el desempeño global de la red.

\subsection{Sensórica Agrícola}
La  \textbf{sensórica  agrícola}  comprende  el  conjunto  de  sensores  físicos
empleados para medir variables del suelo, del ambiente y del estado del cultivo.
Estos sensores transforman fenómenos  físicos  y  químicos  en  datos  digitales
utilizables por los sistemas de análisis y decisión.

En esta investigación, la sensórica define la carga de datos que  debe  soportar
la red LPWAN y condiciona la frecuencia de reporte y la vida útil de los nodos.

\subsection{Variables Agronómicas}
Las \textbf{variables agronómicas} son los parámetros medibles que describen  el
estado  del  sistema  agrícola,  tales  como  humedad  del  suelo,  temperatura,
precipitación, pH, índices de vegetación y variables microclimáticas.

Estas variables constituyen la \textbf{entrada  principal}  de  los  modelos  de
decisión agronómica  y  determinan  los  requisitos  de  resolución  espacial  y
temporal del sistema de monitoreo.

\subsection{Densidad de Nodos}
La \textbf{densidad de nodos} se define como el número de  sensores  desplegados
por unidad de área (e.g., nodos por hectárea). Este concepto  está  directamente
relacionado con la resolución espacial de los datos y con la carga de tráfico en
la red LPWAN.

En este trabajo, la densidad de nodos es una  variable  clave  para  evaluar  el
rendimiento de la arquitectura de red mediante simulación.

\subsection{Simulación de Redes LPWAN}
La \textbf{simulación de redes LPWAN} es una técnica de  modelado  computacional
que permite reproducir el comportamiento de una red bajo diferentes  condiciones
de configuración, tráfico y entorno, sin necesidad de un despliegue físico.

Conceptualmente, la simulación actúa como una \textbf{herramienta de  evaluación
      y
      apoyo a la toma de decisiones}, permitiendo analizar métricas de desempeño
como la
entrega de paquetes, la latencia y el consumo energético.

\subsection{Desempeño de la Red}
El \textbf{desempeño de la red LPWAN} se refiere al  conjunto  de  métricas  que
caracterizan  su  comportamiento  operativo,  tales  como  tasa  de  entrega  de
paquetes, retardo, consumo energético y escalabilidad.

Estas métricas permiten evaluar la idoneidad  de  una  arquitectura  LPWAN  para
soportar aplicaciones de Agricultura de Precisión en escenarios rurales.

\subsection{Modelos de Decisión Agronómica}
Los  \textbf{modelos  de  decisión  agronómica}  son  sistemas  computacionales,
apoyados en técnicas  de  inteligencia  artificial  y  análisis  de  datos,  que
transforman  los  datos  recolectados  en   recomendaciones   o   prescripciones
agronómicas.

En esta investigación, dichos  modelos  dependen  directamente  de  la  calidad,
oportunidad y continuidad de los datos transmitidos por la red LPWAN.

\subsection{Agentes de Inteligencia Artificial}
Los   \textbf{agentes   de   inteligencia   artificial   (IA)}   son   entidades
computacionales autónomas capaces de  percibir  su  entorno,  razonar  sobre  la
información  disponible  y  ejecutar  acciones  orientadas  al  cumplimiento  de
objetivos específicos. Estos agentes  pueden  integrar  modelos  de  aprendizaje
automático, reglas de decisión y acceso a herramientas externas  para  adaptarse
dinámicamente a diferentes escenarios.

En esta investigación, los agentes de IA actúan  como  componentes  inteligentes
encargados de interpretar los datos agronómicos  recolectados,  interactuar  con
los resultados de la simulación de redes LPWAN y apoyar la toma de decisiones en
contextos de Agricultura de Precisión.

\subsection{Aplicación Web}
Una \textbf{aplicación web} es una plataforma de software accesible mediante  un
navegador, que integra interfaces de usuario, lógica de negocio y  servicios  de
procesamiento de datos. Estas aplicaciones facilitan la  interacción  entre  los
usuarios  y  los  sistemas  computacionales  complejos  de  forma  intuitiva   y
centralizada.

Dentro  del  contexto  de  este  trabajo,  la  aplicación  web   constituye   la
\textbf{capa de interacción} del sistema, permitiendo visualizar los  resultados
de la simulación de la red LPWAN, gestionar  configuraciones  del  despliegue  y
acceder a las recomendaciones generadas por los modelos y agentes de decisión.

\subsection{Retrieval-Augmented Generation (RAG)}
El  \textbf{Retrieval-Augmented  Generation  (RAG)}  es  una   arquitectura   de
inteligencia artificial  que  combina  modelos  generativos  con  mecanismos  de
recuperación de información desde fuentes externas, tales como bases  de  datos,
repositorios documentales o históricos de mediciones.

En esta investigación, RAG permite a los agentes de IA enriquecer  sus  procesos
de  razonamiento  incorporando  conocimiento  contextual  proveniente  de  datos
agronómicos históricos, resultados de simulación y parámetros de la  red  LPWAN,
mejorando la calidad y pertinencia de las recomendaciones generadas.

\subsection{Arquitectura Multiagente}
La \textbf{arquitectura multiagente} es un enfoque de diseño de sistemas  en  el
cual múltiples agentes de IA  interactúan  entre  sí  de  manera  cooperativa  o
coordinada, cada uno  con  responsabilidades  y  capacidades  específicas,  para
resolver problemas complejos de forma distribuida.

En el marco  de  este  estudio,  la  arquitectura  multiagente  permite  separar
funciones como el análisis de desempeño de la red, la interpretación  agronómica
de los datos y la generación de recomendaciones,  favoreciendo  la  modularidad,
escalabilidad y robustez del sistema.

\subsection{Orquestación de Agentes}
La \textbf{orquestación de agentes} se refiere al  conjunto  de  mecanismos  que
gestionan la coordinación, secuenciación y comunicación entre múltiples  agentes
de IA, definiendo flujos de interacción, dependencias y reglas de ejecución.

En esta investigación, la orquestación de agentes asegura que  los  procesos  de
simulación de la red, análisis de datos agronómicos y generación  de  decisiones
se realicen de  manera  coherente,  controlada  y  reproducible,  alineando  las
acciones de los agentes con los objetivos del sistema.

\subsection{Gobernanza de Sistemas de IA}
La \textbf{gobernanza de sistemas de IA} comprende los principios,  políticas  y
mecanismos técnicos orientados a garantizar el uso responsable,  transparente  y
controlado  de  los  sistemas  inteligentes.  Esto  incluye  aspectos  como   la
trazabilidad de decisiones, el control de salidas, la  gestión  de  datos  y  la
validación de resultados.

