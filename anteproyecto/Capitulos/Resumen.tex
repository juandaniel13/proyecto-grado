\chapter{Resumen}

El sector agrícola  colombiano  enfrenta  múltiples  desafíos  derivados  de  la
limitada adopción tecnológica, la falta de conectividad en zonas  rurales  y  la
creciente vulnerabilidad ante fenómenos climáticos. Estas condiciones dificultan
el monitoreo oportuno de los  cultivos  y  la  toma  de  decisiones  informadas,
afectando  la  productividad  y  sostenibilidad  del  campo.   Frente   a   esta
problemática, se propone el desarrollo de una aplicación web para el análisis  y
simulación de redes  LPWAN  (Low  Power  Wide  Area  Network)  con  inteligencia
artificial, orientada a fortalecer la agricultura inteligente en Colombia.

La aplicación integrará agentes de IA capaces de analizar contextos  específicos
—como tipo de cultivo, ubicación, condiciones ambientales  y  presupuesto—  para
identificar las variables agrícolas más relevantes y realizar predicciones sobre
el rendimiento de  diferentes  tecnologías  LPWAN.  Posteriormente,  el  sistema
comparará protocolos como LoRa, Sigfox y NB-IoT,  determinando  cuál  ofrece  un
mejor desempeño en función de parámetros como cobertura, costo, escalabilidad  y
eficiencia energética.

El desarrollo  se  llevará  a  cabo  adoptando  las  prácticas  del  enfoque  de
desarrollo de Rapid Application Development (RAD), la metodología  utilizada  se
base en el prototipado rápido e implementación del feedback  del  usuario  final
para cerrar la brecha entre el producto propuesto por el equipo de dessarrollo y
el esperado por el usuario final, esto con el fin de asegurar la satisfación del
usuairo final. Este proyecto busca contribuir al cumplimiento de  los  Objetivos
de Desarrollo  Sostenible  (ODS),  promoviendo  la  innovación  tecnológica,  la
sostenibilidad rural y la resiliencia climática del sector agrícola colombiano.

\section{Palabras Clave}

inteligencia  artificial;  redes  LPWAN;   agricultura   inteligente;   análisis
predictivo; sostenibilidad rural.