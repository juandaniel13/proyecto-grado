\chapter{Resumen}

El sector agrícola  colombiano  enfrenta  múltiples  desafíos  derivados  de  la
limitada adopción tecnológica, la falta de conectividad en zonas  rurales  y  la
creciente vulnerabilidad ante fenómenos climáticos. Estas condiciones dificultan
el monitoreo oportuno de los  cultivos  y  la  toma  de  decisiones  informadas,
afectando  la  productividad  y  sostenibilidad  del  campo.   Frente   a   esta
problemática, se propone el desarrollo de una aplicación web para el análisis  y
simulación de redes  LPWAN  (Low  Power  Wide  Area  Network)  con  inteligencia
artificial, orientada a fortalecer la agricultura inteligente en Colombia.

La aplicación integrará agentes de inteligencia artificial capaces  de  analizar
contextos específicos —como tipo de cultivo, ubicación  geográfica,  condiciones
ambientales y presupuesto disponible— con el fin de  identificar  las  variables
agrícolas más relevantes y los sensores adecuados para su medición y  monitoreo.
Asimismo,  proporcionará  información  comparativa  sobre  el   rendimiento   de
distintas tecnologías LPWAN.

Posteriormente, el sistema evaluará  protocolos  como  LoRa,  Sigfox  y  NB-IoT,
determinando cuál presenta el mejor desempeño en  función  de  parámetros  tales
como  cobertura,  costo,  escalabilidad  y  eficiencia  energética.  Finalmente,
ofrecerá al usuario una representación visual de  la  arquitectura  de  red  que
mejor se adapte a los requerimientos técnicos y operativos de su proyecto.

El desarrollo  se  llevará  a  cabo  adoptando un enfoque híbrido que se compone
del modelo de prototipado y Kanban. La metodología  utilizada  se
base en el prototipado rápido e implementación del feedback  del  usuario  final
para cerrar la brecha entre el producto propuesto por el equipo de dessarrollo y
el esperado por el usuario final, esto con el fin de asegurar la satisfación del
usuairo final. Este proyecto busca contribuir al cumplimiento de  los  Objetivos
de Desarrollo  Sostenible  (ODS),  promoviendo  la  innovación  tecnológica,  la
sostenibilidad rural y la resiliencia climática del sector agrícola colombiano.

\section{Palabras Clave}

Inteligencia  artificial,  redes  LPWAN,   agricultura  4.0, agricultura de precisión, agentes de inteligencia artificial, sensórica, Kanban, modelo de prototipado.