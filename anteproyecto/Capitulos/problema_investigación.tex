       
\chapter{Problema de Investigación}




\section{Planteamiento del problema}

El sector agrícola colombiano es un pilar de la economía nacional, representando el 10.5\% del valor agregado bruto y registrando un crecimiento del 3.8\% en el segundo trimestre de 2025, respecto al mismo periodo un año antes \parencite{DANE2025}. A pesar de su relevancia, este sector enfrenta una limitación estructural en su modernización: la brecha de conectividad digital en zonas rurales \parencite{alvarez2022colombian}. Esta restricción dificulta la adopción de Agricultura 4.0, la cual depende de tecnologías como el Internet de las Cosas (IoT) y la Inteligencia Artificial (IA) para optimizar recursos y mejorar la productividad.

En este contexto de conectividad limitada, persiste una incertidumbre técnica relacionada con la selección e integración adecuada de tecnologías LPWAN —como LoRaWAN o NB-IoT— para su despliegue en escenarios agrícolas específicos. Esta decisión involucra compensaciones entre cobertura (especialmente sensible en regiones con infraestructura limitada), consumo energético y costo \parencite{Cognitive-LPWAN}. Actualmente, dicha selección se realiza de manera empírica debido a la ausencia de herramientas de apoyo, generando un alto riesgo de implementaciones subóptimas, ineficientes o económicamente inviables.

A ello se suma una desconexión entre las capacidades de las herramientas de simulación y las necesidades reales del entorno agrícola. Si bien existen simuladores como NS-3 o LoRaSim \parencite{almuhaya2022survey}, estos permiten modelar el comportamiento técnico de las redes LPWAN pero no traducen las necesidades del cultivo, terreno y condiciones ambientales en criterios de decisión tecnológica. Como resultado, se evidencia un vacío de investigación en la integración de modelos de desempeño de red con parámetros agronómicos, orientado a la generación de una recomendación tecnológica sistemática y fundamentada.

Las consecuencias de esta brecha son significativas: la falta de conectividad limita la capacidad de monitoreo y control de variables críticas del cultivo, perpetúa ineficiencias en el uso de recursos como agua y fertilizantes \parencite{alvarez2022colombian}, incrementa la desigualdad digital y afecta la competitividad del sector agrícola colombiano en mercados globales \parencite{florez2021agroindustria}.

Por lo tanto, surge la necesidad de desarrollar una herramienta inteligente que, integrando simulación de redes LPWAN e inteligencia artificial, sirva como asistente técnico para la selección de tecnologías de comunicación rurales, permitiendo predecir su desempeño según condiciones reales del terreno y los requerimientos del proyecto agrícola. Tal herramienta contribuiría a convertir una decisión compleja y especializada en una recomendación técnica sólida y accesible, respaldada en evidencia y simulación.


\section{Formulación del problema}




\subsection{Pregunta principal de investigación}


¿En qué medida una plataforma de software con agentes de inteligencia artificial que integran simulaciones de redes LPWAN y modelos de decisión agrícolas, mejora la precisión en las recomendaciones técnicas y reduce el tiempo de selección de tecnologías de conectividad para agricultura de precisión, en comparación con los métodos empíricos actualmente utilizados por ingenieros en Colombia?

\subsection{Sistematización del problema}


\begin{enumerate}[leftmargin=*, align=left]
    \item ¿Cuáles son los criterios técnicos determinantes para seleccionar entre las tecnologías LPWAN en el contexto agrícola colombiano?
    
    \item ¿Cómo se puede diseñar e implementar un sistema de inteligencia artificial que, integrado con un sistema de simulación de redes, traduzca los criterios identificados en una recomendación técnica cuantificable sobre la tecnología LPWAN óptima? 


    \item ¿En qué medida la recomendación de la plataforma genera una reducción significativa en los costos de implementación inicial y operativos, en comparación con una selección de tecnología realizada por expertos mediante métodos empíricos tradicionales?
 
\end{enumerate}