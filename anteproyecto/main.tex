\documentclass[12pt,letterpaper]{report}

% ====== Idioma y codificación ======
\usepackage[spanish,es-noshorthands]{babel}
\usepackage[T1]{fontenc}
\usepackage[utf8]{inputenc}

% ====== Tipografía y márgenes (APA 7) ======
\usepackage{newtxtext,newtxmath}
\usepackage{geometry}
\geometry{margin=2.54cm, headheight=16pt}
\usepackage{setspace}
\doublespacing 

% ====== Figuras, Tablas y Gráficos ======
\usepackage{graphicx}
\usepackage{float}       % Fix: Para usar [H]
\usepackage{booktabs}    % Fix: Para \toprule, \midrule
\usepackage{tabularx}    % Fix: Para tablas tabularx
\usepackage[hidelinks]{hyperref}

% ====== Bibliografía (APA 7) ======
\usepackage[backend=biber,style=apa]{biblatex}
\usepackage{enumitem}
\addbibresource{referencias.bib}

% ====== Encabezados ======
\usepackage{fancyhdr}
\pagestyle{fancy}
\fancyhf{}
\fancyhead[R]{\thepage}
\renewcommand{\headrulewidth}{0pt}

% ====== Comandos del proyecto ======
\newcommand{\tituloTesis}{Desarrollo de una aplicación web basada en agentes de inteligencia artificial para el análisis, soporte a la decisión y simulación visual de arquitecturas LPWAN en escenarios de agricultura de precisión en Colombia}
\newcommand{\nombreEstudiante}{Santiago Sánchez Moya}
\newcommand{\nombreEstudianteJ}{Juan Daniel Rodríguez Hurtado}
\newcommand{\nombreDirector}{Ing. Roberto Ferro Escobar, PhD}
\newcommand{\nombreCoDirector}{Ing. Carlos Andrés Martínez Alayón, MsC}
\newcommand{\nombreUni}{Universidad Distrital Francisco José de Caldas}
\newcommand{\nombreFacultad}{Facultad de Ingeniería}
\newcommand{\nombreProyecto}{Proyecto Curricular de Ingeniería de Sistemas}
\newcommand{\ciudad}{Bogotá D. C., Colombia}
\newcommand{\mesanio}{2026}

\begin{document}

% ====== Portada ======
\begin{titlepage}
  \centering
  \begin{minipage}{0.45\textwidth}
    \flushleft
    \includegraphics[height=2cm]{IMAGENES/LOGOLIDER.jpg}
  \end{minipage}
  \hfill
  \begin{minipage}{0.45\textwidth}
    \flushright
    \includegraphics[height=2cm]{IMAGENES/LOGOUD.png}
  \end{minipage}

  \vspace*{3cm}
  {\large \textbf{\tituloTesis} \par}
  \vfill
  {\normalsize \nombreEstudiante \\ \nombreEstudianteJ \par}
  \vfill
  {\normalsize
    Trabajo de grado para optar por el título de:\\
    \textbf{Ingeniero de Sistemas} \par
    \vspace{1cm}
    Director:\\ \nombreDirector \\
    Codirector:\\ \nombreCoDirector \par}
  \vfill
  {\normalsize
    \nombreUni \\
    \nombreFacultad \\
    \nombreProyecto \\
    \ciudad \\
    \mesanio \par}
\end{titlepage}

% ====== Índice ======
\pagenumbering{roman} % Páginas preliminares en romanos
\tableofcontents
\clearpage

% ====== Contenido ======
\pagenumbering{arabic} % Reinicia numeración a arábigos
\chapter{Resumen}

El sector agrícola colombiano enfrenta múltiples desafíos derivados de la limitada adopción tecnológica, la falta de conectividad en zonas rurales y la creciente vulnerabilidad ante fenómenos climáticos. Estas condiciones dificultan el monitoreo oportuno de los cultivos y la toma de decisiones informadas, afectando la productividad y sostenibilidad del campo. Frente a esta problemática, se propone el desarrollo de una aplicación web para el análisis y simulación de redes LPWAN (Low Power Wide Area Network) con inteligencia artificial, orientada a fortalecer la agricultura inteligente en Colombia.

La aplicación integrará agentes de IA capaces de analizar contextos específicos —como tipo de cultivo, ubicación, condiciones ambientales y presupuesto— para identificar las variables agrícolas más relevantes y realizar predicciones sobre el rendimiento de diferentes tecnologías LPWAN. Posteriormente, el sistema comparará protocolos como LoRa, Sigfox y NB-IoT, determinando cuál ofrece un mejor desempeño en función de parámetros como cobertura, costo, escalabilidad y eficiencia energética.

El desarrollo se llevará a cabo bajo la metodología ágil Scrumban, que combina la estructura iterativa de Scrum con la adaptabilidad de Kanban, facilitando un proceso de implementación flexible y colaborativo. Este proyecto busca contribuir al cumplimiento de los Objetivos de Desarrollo Sostenible (ODS), promoviendo la innovación tecnológica, la sostenibilidad rural y la resiliencia climática del sector agrícola colombiano.

\section{Palabras Clave}

inteligencia artificial; redes LPWAN; agricultura inteligente; análisis predictivo; sostenibilidad rural.
\chapter{Introducción}
La agricultura, pilar histórico  de  la  economía  y  sustento  de  millones  en
Colombia,  se  ve  crecientemente  amenazada  por  el  cambio  climático  y   la
variabilidad ambiental extrema, que generan pérdidas significativas y reducen la
competitividad del sector.  Para  mitigar  estos  riesgos,  la  adopción  de  la
agricultura de precisión, basada en redes de sensores y análisis  de  datos,  se
presenta  como  una  solución  indispensable.  Sin  embargo,  su  implementación
efectiva tropieza con una barrera crítica previa: la complejidad  técnica  y  la
incertidumbre en la toma de decisiones inicial para seleccionar y  desplegar  la
infraestructura de conectividad adecuada, como las redes  LPWAN  (LoRa,  Sigfox,
NB-IoT, entre otras), en diversos contextos agrícolas.\\


Ante  este  panorama,  la  incorporación  de  tecnologías   modernas   como   la
inteligencia artificial (AI) y las redes LPWAN (Low  Power  Wide  Area  Network)
ofrece nuevas  oportunidades  para  fortalecer  la  agricultura  inteligente  en
Colombia. En este contexto, se propone el desarrollo de una aplicación web  para
el análisis y simulación  de  arquitecturas  de  redes  LPWAN  con  inteligencia
artificial, orientada a mejorar la  toma  de  decisiones  en  el  despliegue  de
tecnologías de monitoreo agrícola. La aplicación  empleará  un  sistema  de  que
incluye un agente de IA, el cual a partir de un contexto determinado —como  tipo
de cultivo, ubicación geográfica, presupuesto  disponible  y  requerimientos  de
conectividad—,  identificará  las  variables  más  relevantes  y  realizarán  un
análisis comparativo entre las distintas tecnologías LPWAN, seleccionando la que
mejor se ajuste al escenario propuesto, proponiendo una aquitectura báse para su
despliegue y recomendando los sensores más adecuados para el tipo de cultivo que
se desea monitorear. Con base en dicho análisis, el sistema recomendará  la  red
más adecuada según criterios de rendimiento, costo, cobertura y escalabilidad.\\

En este contexto, el presente proyecto se desarrolla en el marco  del  semillero
SCISEN (Smart cities \& sensor network),  adscrito  al  grupo  de  investigación
LIDER (Laboratorio de Investigación y desarrollo en Electrónica y redes)  de  la
Universidad Distrital Francisco José de Caldas. Este grupo trabaja  en  diversas
líneas de investigación internas, entre las que se destacan  en  este  proyecto:
Agrointeligente (Smart Agro), internet de las cosas  (IoT)  y  el  desarrollo  y
programación de hardware, firmware y software  para  sistemas  de  comunicación.
Estas  líneas  se  encuentran  alineadas   con   los   ejes   de   investigación
institucionales en ciencias de la computación, desarrollo regional  sustentable,
infraestructura   y   tecnología,   redes    de    sensores    inalámbricos    y
telecomunicaciones. Asimismo, el  proyecto  está  directamente  relacionado  con
varios Objetivos de Desarrollo Sostenible (ODS) establecidos por la ONU: \\
\begin{itemize}
    \item  ODS 2. Hambre Cero: Este proyecto contribuye  a  una  agricultura  más
          resiliente y productiva al facilitar la toma de  decisiones  informada
          para el despliegue de tecnologías  de  agricultura  de  precisión.  Al
          ayudar a seleccionar la arquitectura  de  conectividad  (LPWAN)  más
          adecuda para cada  contexto,  la  herramienta  reduce  la  barrera  de
          entrada y el riesgo de inversión en sistemas de monitoreo y predicción
          agroclimática. Esto, a su vez, permite a los  productores  implementar
          soluciones más eficaces para anticipar fenómenos climáticos  adversos,
          optimizar recursos y reducir pérdidas, fortaleciendo así la  seguridad
          alimentaria  desde  una  perspectiva  de   planificación   tecnológica
          robusta.
    \item  ODS 9. Industria, Innovación e Infraestructura:  El  proyecto  es  una
          innovación  directa  en  el  ecosistema  tecnológico   agroindustrial.
          Desarrolla una herramienta digital avanzada (aplicación  web  con  IA)
          que resuelve un cuello de botella crítico en la cadena de valor de  la
          agricultura 4.0: la selección de arquitectura de  conectividad.  Al
          proporcionar un método accesible y basado en datos para diseñar  redes
          LPWAN  eficientes,  promueve  la   construcción   de  una arquitectura
          tecnológica más resiliente, accesible  y  adecuada  al  entorno  rural
          colombiano,   fomentando   directamente    la    innovación    y    la
          industrialización sostenible del sector.
    \item ODS 13. Acción por el Clima: La herramienta  contribuye  a  la  acción
          climática de manera indirecta pero  potente,  al  ser  un  facilitador
          clave para la agricultura climáticamente inteligente. Al optimizar  la
          elección de redes de sensores, maximiza la viabilidad y efectividad de
          los sistemas que monitorean variables climáticas y  ambientales.  Esto
          permite a los agricultores adaptar sus prácticas  con  base  en  datos
          precisos, mejorar la gestión de recursos como el agua y los insumos, y
          reducir  la  vulnerabilidad  de  los  cultivos,  promoviendo  así  una
          adaptación sistémica y basada en evidencia frente al cambio climático.
\end{itemize}

El desarrollo se llevará a cabo bajo la metodología ágil Scrumban,  que  combina
la flexibilidad de Kanban con la estructura iterativa de Scrum, favoreciendo  la
gestión adaptativa de tareas y la entrega  continua  de  valor.  Esta  propuesta
busca  contribuir  al  avance  de  la  agricultura  inteligente   en   Colombia,
facilitando  la  integración  de  herramientas  digitales  en   zonas   rurales,
optimizando el uso de recursos tecnológicos y promoviendo la sostenibilidad  del
sector.

Adicionalmente, se articula con el proyecto doctoral titulado “Estructuración de
un modelo para el análisis, simulación y  aplicación  de  tecnologías  LPWAN,  a
través de la integración  comunitaria  de  redes  IOT  al  agro  inteligente  en
sectores rurales colombianos", cuyo investigador principal  es  el  Ing.  Carlos
Andrés Martínez Alayón, bajo  la  dirección  del  docente  tutor  Roberto  Ferro
Escobar. Este proyecto ha sido institucionalizado sin recursos por el Consejo de
la Facultad de Ingeniería y está registrado en el  Sistema  de  Información  del
Centro de  Investigaciones  (SICIUD),  fortaleciendo  su  respaldo  académico  y
científico. Con las siguientes especificaciones:
\begin{itemize}
    \item Código SICIUD: 3370084523
    \item Estado: Vigente sin financiación
    \item Grupo  de  Investigación: Laboratorio   de
          Investigación y desarrollo
          en Electrónica y redes.
\end{itemize}
El desarrollo de este proyecto  contribuirá  con  los  siguientes  productos  de
investigación: La dirección de una tesis de pregrado, entrega de un  informe  de
investigación, el desarrollo de un producto de desarrollo web y un artículo para
la participación de una ponencia a un evento nacional o internacional.

\chapter{Problema de Investigación}




\section{Planteamiento del problema}

El sector agrícola colombiano es un pilar de la economía  nacional,  como  parte
del sector agropecuario,  registró  un  crecimiento  del  3.8\%  en  el  segundo
trimestre de 2025, respecto al mismo periodo en el 2024,  aportando  0,4  puntos
porcentuales al crecimiento total del PIB \parencite{DANEPIB2025II}. A pesar  de
su  relevancia,  este  sector  enfrenta  una  limitación   estructural   en   su
modernización:  la   brecha   de   conectividad   digital   en   zonas   rurales
\parencite{alvarez2022colombian}. Esta restricción dificulta la adopción  de  la
Agricultura 4.0, la cual depende de tecnologías como el Internet  de  las  Cosas
(IoT) y la Inteligencia Artificial (IA) para optimizar  recursos  y  mejorar  la
productividad.

En este contexto de conectividad limitada, persiste  una  incertidumbre  técnica
relacionada con la selección e integración adecuada de tecnologías  LPWAN  —como
LoRaWAN o NB-IoT— para su despliegue en escenarios agrícolas  específicos.  Esta
decisión involucra compensaciones entre  cobertura  (especialmente  sensible  en
regiones   con   infraestructura   limitada),   consumo   energético   y   costo
\parencite{Cognitive-LPWAN}. Actualmente, dicha selección se realiza  de  manera
empírica debido a la ausencia de herramientas de apoyo, generando un alto riesgo
de implementaciones subóptimas, ineficientes o económicamente inviables.

A ello se suma una desconexión entre las  capacidades  de  las  herramientas  de
simulación y las necesidades  reales  del  entorno  agrícola.  Si  bien  existen
simuladores como NS-3 o LoRaSim \parencite{almuhaya2022survey},  estos  permiten
modelar el comportamiento técnico de  las  redes  LPWAN  pero  no  traducen  las
necesidades del cultivo, terreno  y  condiciones  ambientales  en  criterios  de
decisión tecnológica. Como resultado, se evidencia un vacío de investigación  en
la integración de modelos  de  desempeño  de  red  con  parámetros  agronómicos,
orientado a  la  generación  de  una  recomendación  tecnológica  sistemática  y
fundamentada.

Las consecuencias de esta brecha son significativas: la  falta  de  conectividad
limita la capacidad de monitoreo y control de variables  críticas  del  cultivo,
perpetúa  ineficiencias  en  el  uso  de  recursos  como  agua  y  fertilizantes
\parencite{alvarez2022colombian}, incrementa la desigualdad digital y afecta  la
competitividad   del   sector   agrícola   colombiano   en   mercados   globales
\parencite{florez2021agroindustria}.

Por lo tanto, surge la necesidad de desarrollar una herramienta inteligente que,
integrando simulación de redes  LPWAN  e  inteligencia  artificial,  sirva  como
experto virtual para  la  selección  de  tecnologías  de  comunicación  rurales,
permitiendo predecir su desempeño según condiciones reales  del  terreno  y  los
requerimientos del proyecto agronómico. Tal herramienta contribuiría a convertir
una decisión compleja y especializada en  una  recomendación  técnica  sólida  y
accesible, respaldada en evidencia y simulación.


\section{Formulación del problema}



\subsection{Pregunta principal de investigación}

¿En qué medida una plataforma de software  basada  en  agentes  de  inteligencia
artificial, que integra simulaciones  de  redes  LPWAN  y  modelos  de  decisión
agronómicos, mejora la estimación del rendimiento de red (medido mediante Packet
Delivery Ratio en función de la  densidad  de  nodos)  y  reduce  el  tiempo  de
selección de tecnologías de  conectividad  para  agricultura  de  precisión,  en
comparación con métodos tradicionales de análisis técnico?

\subsection{Sistematización del problema}


\begin{enumerate}[leftmargin=*, align=left]
      \item ¿Cuáles son los criterios técnicos determinantes para seleccionar
            entre las tecnologías LPWAN en el contexto agrícola colombiano?

      \item ¿Cómo se puede diseñar e implementar un sistema de inteligencia
            artificial que, integrado con un sistema  de  simulación visual de  redes,
            traduzca los criterios identificados en  una  recomendación  técnica
            cuantificable sobre la tecnología LPWAN óptima?


      \item ¿En qué medida la recomendación de la plataforma genera una
            reducción significativa en los costos de  implementación  inicial  y
            operativos, en comparación con una selección de tecnología realizada
            por expertos mediante métodos tradicionales?

\end{enumerate} % Error: evité tildes en nombres de archivos
\chapter{Objetivos}

\section{Objetivo General}

Desarrollar una plataforma de  software  inteligente  que  integre  simulaciones
visuales de la arquitectura de redes LPWAN y modelos  de  decisión  agronómicos,
potenciada por inteligencia artificial, para hacer más efectiva y  eficiente  la
selección de tecnologías LPWAN en proyectos de agricultura de precisión en zonas
rurales de Colombia.

\section{Objetivos Específicos}

\begin{enumerate}[leftmargin=*, align=left]
    \item Identificar y documentar los requerimientos técnicos y
          agronómicos determinantes para la selección de tecnologías LPWAN en el
          contexto agrícola colombiano, a través de la revisión literaria  y  la
          consulta con expertos, estableciendo los requisitos funcionales  y  no
          funcionales de la plataforma.

    \item Diseñar una arquitectura de software que integre agentes de
          inteligencia artificial para la simulación y  toma  de  decisiones  de
          arquitecturas LPWAN.

    \item Implementar el sistema de decisión y simulación automatizada
          para selección de tecnologías LPWAN, bajo  un  esquema  de  desarrollo
          orientado a prototipos, asegurando su afinidad con los  requerimientos
          previamente definidos.

    \item Evaluar y Cuantificar el impacto en eficiencia operativa y económica
          frente a métodos tradicionales.

    \item Documentar detalladamente la arquitectura de software, el
          proceso de desarrollo orientado a prototipos y los  resultados  de  la
          validación operativa y económica para garantizar la  reproducibilidad,
          escalabilidad y transferibilidad de  la  solución  a  otros  contextos
          agrícolas.
\end{enumerate}


\chapter{Justificación}

El presente proyecto tiene  como  propósito  ofrecer  una  solución  tecnológica
innovadora mediante el desarrollo de una aplicación web que permita  analizar  y
simular  redes  LPWAN  (Low  Power  Wide  Area  Network)  aplicadas  a  entornos
agrícolas, utilizando inteligencia artificial (IA) generativa para  la  toma  de
decisiones  en  contextos  rurales.   Esta   propuesta   busca   contribuir   al
fortalecimiento del sector agrícola colombiano, considerado un pilar estratégico
de  la  economía  nacional  por  su  papel  en  la  generación  de  empleo,   el
abastecimiento alimentario y el desarrollo regional.

En un contexto donde los efectos del cambio climático y la variabilidad  de  las
condiciones ambientales afectan directamente la productividad del campo, se hace
necesario incorporar herramientas tecnológicas que faciliten  la  planificación,
monitoreo y predicción de  variables  agrícolas.  Mediante  el  uso  de  IA,  la
aplicación podrá analizar escenarios específicos —tipo de cultivo, localización,
presupuesto y requerimientos de conectividad— para determinar el protocolo LPWAN
más adecuado  entre  alternativas  como  LoRa,  Sigfox  y  NB-IoT,  considerando
criterios de cobertura, rendimiento, escalabilidad y costo.

\begin{itemize}
      \item \textbf{Justificación Investigativa:} Desde una perspectiva científica
            y tecnológica, este proyecto representa un aporte significativo a  las
            líneas de investigación en agricultura inteligente,  internet  de  las
            cosas (IoT) y comunicaciones de baja potencia, al integrar un  enfoque
            de análisis predictivo impulsado por IA generativa. El  desarrollo  se
            enmarca dentro de los  objetivos  del  grupo  de  investigación  LIDER
            (Laboratorio de Investigación y Desarrollo en Electrónica y Redes)  de
            la Universidad Distrital, a través del semillero SCISEN, articulándose
            con  las  áreas  institucionales  de  ciencias  de   la   computación,
            infraestructura  tecnológica  y   telecomunicaciones.   Asimismo,   la
            aplicación  de  la  metodología  híbrida (modelo de prototipado + Kanban)   permitirá   gestionar   el
            desarrollo del proyecto de manera iterativa y  flexible,  garantizando
            una  evolución  continua  del  producto  y  un  enfoque  orientado   a
            resultados.

      \item \textbf{Justificación Social:} La implementación de esta herramienta
            beneficiará directamente a las comunidades agrícolas, especialmente en
            zonas rurales con limitaciones de conectividad y acceso a  tecnología.
            Al ofrecer una plataforma accesible que asista a los actores  técnicos
            en la selección de la red de comunicación más eficiente,  se  promueve
            la inclusión digital y el empoderamiento tecnológico del sector rural.
            Esto no solo contribuirá a reducir las brechas digitales, sino también
            a mejorar la calidad de vida de los productores  mediante  el  uso  de
            datos inteligentes para optimizar sus operaciones agrícolas.
      \item \textbf{Justificación Económica:} El uso eficiente de las tecnologías
            LPWAN  en  la  agricultura  tiene  un  impacto  económico  directo  al
            optimizar los costos de implementación de redes y mejorar  la  gestión
            de los cultivos. La aplicación propuesta permitirá evaluar la relación
            costo-beneficio de cada tecnología en función del contexto productivo,
            lo  que  reducirá  pérdidas   económicas   derivadas   de   decisiones
            tecnológicas inadecuadas. Además, la posibilidad de simular escenarios
            antes de su implementación real representa un ahorro significativo  en
            infraestructura y mantenimiento, contribuyendo al  fortalecimiento  de
            la economía local y nacional.
      \item \textbf{Justificación Ambiental:} El proyecto aporta al desarrollo
            sostenible mediante el fomento de prácticas agrícolas más eficientes y
            responsables  con  el  medio  ambiente.  Al  proporcionar  información
            precisa sobre variables  y  condiciones  agrícolas,  los  agricultores
            podrán reducir el uso innecesario de insumos como agua,  fertilizantes
            y energía. De igual forma, el aprovechamiento de tecnologías  de  bajo
            consumo energético como las redes  LPWAN  contribuye  a  disminuir  la
            huella ecológica de las operaciones agrícolas, promoviendo  un  modelo
            de producción más sostenible y resiliente ante el cambio climático.
      \item \textbf{Justificación Académica:} El presente proyecto  se  desarrolla
            bajo la modalidad de grado de investigación, investigación - creación, innovación
            reglamentada en el artículo 7 del acuerdo No.02 (marzo 2023) de consejo de facultad de
            ingeniería. Este proyecto se llevará a cabo con en el fin de  que  cada  uno  de  sus
            autores opten por el título de Ingeniero de Sistemas.
\end{itemize}
\chapter{Marco Referencial}

\section{Marco Teórico}

\subsection{Agricultura 4.0}

\subsubsection{Contexto  y  relevancia  para  Colombia}   La   Agricultura   4.0
constituye la integración de tecnologías emergentes —tales como el  Internet  de
las Cosas (IoT), analítica avanzada  de  datos,  automatización  e  inteligencia
artificial— en los sistemas productivos agrícolas, con el propósito de optimizar
procedimientos, incrementar la productividad y fortalecer la  sostenibilidad  de
la  actividad  agrícola   \parencite[]{shafi2019precision}.   En   el   contexto
colombiano, este enfoque  adquiere  una  especial  importancia  debido  al  peso
estratégico del sector rural  y  a  la  diversidad  ecológica  y  climática  que
caracteriza los sistemas de producción del país.

No obstante, la literatura evidencia la existencia de rezagos  estructurales  en
materia de  adopción  tecnológica.  Únicamente  el  1.7\%  de  las  Unidades  de
Producción Agrícola (UPA) cuentan con conectividad a Internet y  solo  el  6.6\%
poseen activos de Tecnologías de la Información y las  Comunicaciones,  lo  cual
supone una limitación crítica para el despliegue de  infraestructuras  digitales
aplicables a esquemas de Agricultura 4.0 \parencite[]{alvarez2022colombian}.

Asimismo, un reporte del año 2021 de la corporación colombiana de  investigación
agropecuaria   (AGROSAVIA)   sostiene   que   la   transición   hacia   sistemas
agroindustriales inteligentes requiere  no  solo  inversión  en  infraestructura
tecnológica, sino también el fortalecimiento de capacidades  humanas,  así  como
procesos  efectivos  de  transferencia  de  conocimiento   técnico   hacia   los
productores rurales \parencite[]{florez2021agroindustria}.

De este modo, la Agricultura 4.0 en Colombia se configura no únicamente como una
transformación técnica, sino  como  un  proceso  socio-tecnológico  que  implica
cambios en la organización productiva, la  formación  de  capital  humano  y  la
gobernanza del conocimiento agrícola.

\subsubsection{Retos: brecha digital e ineficiencias} A pesar  de  las  ventajas
potenciales que supone la Agricultura 4.0, su implementación enfrenta  múltiples
desafíos. Uno de los más significativos es la denominada brecha  digital  rural,
entendida como la  desigualdad  en  el  acceso  a  tecnologías  digitales  entre
agricultores tecnificados y pequeños productores tradicionales. Se evidencia que
esta  brecha  no  solo  se  manifiesta  en  infraestructura,  sino  también   en
alfabetización  digital,  acceso  a  plataformas  inteligentes  y  capacidad  de
inversión en innovación tecnológica \parencite[]{quiroz2023agricultura}.

A ello  se  suma  la  baja  disponibilidad  de  conectividad  en  zonas  rurales
colombianas,  condición  imprescindible  para  soportar  sistemas  de  monitoreo
remoto,  transmisión  de  datos  sensorados  y  comunicación  distribuida  entre
dispositivos IoT \parencite[]{alvarez2022colombian}.

AGROSAVIA advierte además que la adopción acelerada de tecnologías  inteligentes
podría agravar desigualdades existentes  si  no  se  acompaña  de  políticas  de
democratización   tecnológica   y   apropiación    social    del    conocimiento
\parencite[]{florez2021agroindustria}.

Finalmente,  en  términos  operativos,  la  integración  de  modelos  de   datos
provenientes  de  sensores,  imágenes  satelitales  y  simulaciones  agronómicas
plantea grandes retos para la gestión eficiente de  información  y  la  toma  de
decisiones automatizada. Se identifican diez dominios en los que convergen estas
tecnologías, destacando la necesidad de arquitecturas interoperables para lograr
la    integración    funcional     de     múltiples     fuentes     de     datos
\parencite[]{quiroz2023agricultura}.

\subsubsection{El rol del IoT y la agricultura de precisión} La  Agricultura  de
Precisión se constituye como un fundamento metodológico de la  Agricultura  4.0.
Esta se define como una estrategia de gestión basada en la captura y análisis de
datos espacio-temporales, con el objetivo de optimizar decisiones  relativas  al
uso   de    insumos,    manejo    de    suelos    y    gestión    de    cultivos
\parencite[]{quiroz2023agricultura}.

En  este  marco,  el  IoT  emerge  como  un  componente  clave  y  se  presentan
arquitecturas IoT aplicables a la  agricultura  colombiana,  donde  se  integran
sensores ambientales, redes de comunicación de baja potencia  y  plataformas  de
visualización y análisis de datos \parencite[]{Cognitive-LPWAN}

Por  su  parte,  AGROSAVIA  enfatiza  que  la  incorporación  de   sistemas   de
sensorización y teledetección puede transformar los  procesos  tradicionales  de
cultivo en sistemas de control inteligente, combinando el conocimiento  empírico
local       con        modelos        predictivos        y        automatización
\parencite[]{florez2021agroindustria}.

En consecuencia, tanto la Agricultura de Precisión como las infraestructuras IoT
proveen las bases tecno-operativas para evolucionar hacia modelos agrícolas  más
robustos, sostenibles y basados en evidencia cuantificable. Este enfoque integra
los datos provenientes del entorno físico con modelos de decisión computacional,
reduciendo la incertidumbre operativa y permitiendo una mayor eficiencia  en  el
uso de recursos críticos como agua, fertilizantes y energía.


\subsection{Redes LPWAN: Tecnologías Habilitadoras de la Agricultura de
      Precisión}

\subsubsection{Fundamentos: Objetivos de diseño, toppologías, arquitecturas} Las
Redes de Área Amplia  de  Baja  Potencia  (LPWAN,  por  sus  siglas  en  inglés)
constituyen una categoría de tecnologías de  comunicación  inalámbrica  diseñada
para resolver el compromiso fundamental en la conectividad del Internet  de  las
Cosas (IoT): la necesidad de operar con bajo consumo de energía y largo  alcance
\parencite[]{diane2025systematic}. Este equilibrio es crucial para el despliegue
de  dispositivos  alimentados   por   batería   que   deben   transmitir   datos
intermitentemente    a    lo    largo    de    extensas    áreas     geográficas
\parencite[]{chen2018cognitive}.

Los objetivos de diseño  de  las  redes  LPWAN  están  orientados  a  la  máxima
eficiencia y efectividad: \begin{itemize}
      \item \textbf{Largo rango de transmisión:} Las tecnologías LPWAN están
            diseñadas para ofrecer comunicaciones de largo  alcance,  alcanzando
            varios kilómetros en zonas rurales y entre 2  y  5  km  en  entornos
            urbanos, especialmente aquellas basadas en  infraestructura  celular
            como NB-IoT. Muchas de estas tecnologías operan en  la  banda  Sub-1
            GHz, cuyas características permiten  una  mayor  penetración  de  la
            señal, menor  atenuación  y  menor  efecto  de  desvanecimiento  por
            multitrayectoria,  además  de  presentar  menos  interferencias   en
            comparación con la banda  de  2.4  GHz.  Tecnologías  como  LoRaWAN,
            SigFox, IEEE 802.11ah, IEEE 802.15.4g, Weightless y DASH7 aprovechan
            estas ventajas, mientras que RPMA opera  en  2.4  GHz.  El  tipo  de
            modulación es  un  factor  clave  en  el  alcance  de  comunicación,
            predominando la modulación de banda estrecha, como Ultra  Narrowband
            (UNB), que permite transmisiones de varios kilómetros con bajo nivel
            de ruido, y la modulación de  espectro  ensanchado,  que  mejora  la
            resistencia a interferencias mediante técnicas  como  DSSS,  FHSS  y
            CSS,    empleadas    por    tecnologías    como    LoRa    y    RPMA
            \parencite[]{diane2025systematic}.
      \item \textbf{Eficiencia Energética:} En las redes LPWAN, el consumo
            energético de los nodos depende principalmente de  la  topología  de
            red, los ciclos  de  operación  y  los  protocolos  de  comunicación
            empleados. Dado que los nodos suelen estar alimentados por  baterías
            y se despliegan a gran escala, es fundamental maximizar su vida útil
            para reducir los costos de mantenimiento. Por esta razón, las  LPWAN
            priorizan la topología en estrella  sobre  la  topología  en  malla,
            evitando  el  sobreconsumo  energético  de  los  nodos  intermedios.
            Además, el uso del duty cycle permite  disminuir  significativamente
            el consumo de energía al  mantener  los  nodos  en  estado  inactivo
            durante  ciertos  intervalos,  aunque  este  mecanismo  puede  estar
            limitado por regulaciones del espectro.  Finalmente,  el  empleo  de
            protocolos  de  acceso  al  medio  simplificados,  como   ALOHA   en
            tecnologías  como  LoRaWAN  y  SigFox,  contribuye  a   reducir   la
            complejidad de los dispositivos y su  consumo  energético,  mientras
            que NB-IoT utiliza mecanismos de acceso aleatorio propios  de  redes
            celulares \parencite[]{diane2025systematic}.
      \item \textbf{Escalabilidad:} La escalabilidad constituye un aspecto
            fundamental en las LPWAN, ya que numerosos escenarios de  aplicación
            demandan la capacidad de gestionar cientos de miles de  dispositivos
            de manera simultánea. Este concepto hace referencia a  la  habilidad
            de la red para conectar una gran cantidad de nodos  sin  afectar  la
            calidad ni la continuidad de los servicios ofrecidos.  No  obstante,
            la coexistencia de un elevado número de dispositivos puede  provocar
            interferencias, impactando de forma negativa el desempeño de la red.
            Para enfrentar estos desafíos, se implementan distintas  estrategias
            orientadas a mejorar la escalabilidad, tales como la comunicación  a
            través de múltiples canales, la instalación de varios gateways y los
            mecanismos de adaptación dinámica de la velocidad de transmisión  de
            datos \parencite[]{chilamkurthy2022low,}.

      \item \textbf{Bajo costo:} El creciente interés por las tecnologías LPWAN
            se debe, en gran medida, al bajo  costo  de  los  dispositivos.  Por
            ejemplo, equipos basados en LoRa o SigFox pueden adquirirse  por  un
            valor aproximado de 3 a  5  dólares  (año  2025).  Para  reducir  la
            inversión   de   capital,   estas   tecnologías   emplean   diversas
            estrategias, entre ellas el uso de transceptores con formas de  onda
            menos complejas, lo que permite disminuir el tamaño del hardware, la
            tasa máxima de transmisión y los  requerimientos  de  memoria.  Como
            resultado, se reduce la complejidad del diseño y, por ende, el costo
            de fabricación.\\

            Adicionalmente, una estación base puede gestionar decenas  de  miles
            de  dispositivos  finales  distribuidos  en  un   área   de   varios
            kilómetros, lo que contribuye a una disminución significativa de los
            costos operativos para los proveedores de red. Asimismo,  las  LPWAN
            operan tanto en bandas no licenciadas,  como  la  banda  ISM  o  los
            espacios  en  blanco  de  televisión,  como  en  bandas  licenciadas
            propiedad  de  los  operadores,  evitando  así  costos   adicionales
            asociados     al     uso      del      espectro      radioeléctricos
            \parencite[]{diane2025systematic}.
      \item \textbf{Calidad del servicio} La calidad de servicio (QoS) en
            tecnologías LPWAN debe adaptarse a aplicaciones  con  requerimientos
            diversos, desde aquellas tolerantes al  retardo,  como  la  medición
            inteligente, hasta sistemas  que  exigen  transmisión  inmediata  de
            alarmas.  La  QoS  no  depende  de  un  único  parámetro,  sino  del
            equilibrio entre rendimiento (throughput),  latencia,  variación  de
            latencia (jitter) y tasa de error.  Para  mejorarla,  la  literatura
            propone distintas estrategias, como la optimización de parámetros de
            radio (por ejemplo, Spreading Factor y  frecuencia  portadora)  para
            reducir colisiones y aumentar la tasa de entrega de datos, el uso de
            esquemas de corrección de  errores  y  redundancia  adaptativa  para
            incrementar la tasa de entrega de paquetes, modelos analíticos  para
            estimar la tasa de error de bit con mayor precisión y mecanismos  de
            enrutamiento  multi-salto  basados  en  aprendizaje   por   refuerzo
            (Q-learning) que consideran retardo, interferencia, uso de ancho  de
            banda y eficiencia energética. No obstante, el diseño de redes LPWAN
            implica compromisos entre variables como ancho de banda  y  tasa  de
            datos, así como decisiones sobre  espectro,  modulación,  acceso  al
            canal y duplexidad, cuyos impactos pueden  ser  altos,  moderados  o
            bajos    según     el     objetivo     de     desempeño     buscado.
            \parencite[]{chilamkurthy2022low,}.
      \item \textbf{Gestión de la interferencia:} La gestión de interferencias
            en redes LPWAN es un desafío  clave  debido  al  uso  frecuente  del
            espectro no licenciado ISM, que, aunque reduce costos, incrementa la
            congestión y disminuye la confiabilidad en la transmisión de  datos.
            La interferencia puede ser interna, cuando nodos  de  la  misma  red
            transmiten simultáneamente en  frecuencias  iguales  o  superpuestas
            (mitigada mediante técnicas de acceso al canal y diseño orientado  a
            la  escalabilidad),  o  externa,  provocada  por  otras  redes   que
            comparten el  mismo  espectro.  Para  enfrentar  este  problema,  se
            proponen  diversas  técnicas  de  optimización:  el  uso  de   radio
            cognitiva  para  detectar  interferencias  persistentes  y   cambiar
            dinámicamente  a  bandas  menos  congestionadas;  el   procesamiento
            espacial de señales para mejorar la  relación  señal-ruido  (SNR)  y
            eliminar interferencias prolongadas; el band steering, que  desplaza
            la comunicación entre bandas como 2.4 GHz y 5 GHz según el nivel  de
            congestión; el despliegue de múltiples estaciones base para aumentar
            la tasa de extracción de datos y fortalecer la señal en el receptor;
            y el uso de antenas  inteligentes,  que  concentran  la  energía  de
            transmisión hacia receptores específicos y controlan la potencia  de
            manera  eficiente,  mejorando   así   el   SNR   y   reduciendo   la
            interferencia. \parencite[]{chilamkurthy2022low,}.
\end{itemize}




\subsubsection{Análisis Comparativo de  Topologías  y  Arquitecturas  LPWAN}  El
funcionamiento de las tecnologías LPWAN se fundamenta en los objetivos de diseño
discutidos previamente y  en  las  diferentes  estrategias  existentes  para  su
implementación. En este apartado  se  analizan  las  distintas  arquitecturas  y
topologías LPWAN, así como los aspectos relacionados con su interoperabilidad.

\textbf{Topologías}: Los dos tipos principales  de  topología  presentes  en  el
espectro LPWAN son la topología en malla y la topología en estrella, siendo esta
última la más utilizada frente a las estructuras en malla,  debido  a  su  valor
agregado en  términos  de  eficiencia  energética  y  ampliación  del  rango  de
cobertura \parencite[]{chilamkurthy2022low}.

\begin{itemize}
      \item \textbf{Topología de estrella:} La topología de estrella consiste en

            una red de tipo punto a punto (P2P, por sus siglas en inglés), en la
            cual todos los nodos se conectan a un nodo central, denominado hub o
            gateway. Este hub o hub actúa como el único  medio  de  comunicación
            entre los nodos de la topología, enrutando  los  mensajes  hacia  un
            servidor central, donde se gestionan aspectos como  la  redundancia,
            la detección de fallos y la seguridad. Este enfoque  es  ampliamente
            utilizado en aplicaciones de monitoreo  y  en  entornos  peligrosos,
            donde  el  despliegue  de  cableado  representa   un   alto   riesgo
            \parencite[]{chilamkurthy2022low}.\\

            Entre las principales ventajas de una estructura basada en un  único
            hub se encuentran la alta velocidad de transmisión de  mensajes,  la
            escalabilidad (ya que es  posible  añadir  nuevos  nodos  de  manera
            sencilla mediante su conexión directa al hub) y el impacto  limitado
            ante fallos a nivel de nodo, puesto que la desconexión  de  un  nodo
            final no afecta al funcionamiento del resto de la red. No  obstante,
            esta  topología  presenta  desventajas   significativas,   como   la
            existencia de un único punto de fallo; si el gateway o  hub  central
            deja de funcionar, toda la red se  vuelve  inoperable  y  los  nodos
            finales dejan de ser accesibles \parencite[]{chilamkurthy2022low}.

      \item \textbf{Topología de malla:} Esta topología está compuesta por tres
            tipos  de  nodos:  nodos  sensores,  gateways   y   nodos   sensores
            enrutadores. En una topología de malla completamente conectada, cada
            nodo se comunica directamente con todos los demás nodos de  la  red.
            En contraste, en una topología de malla parcial, solo algunos  nodos
            mantienen múltiples conexiones,  mientras  que  otros  se  comunican
            únicamente con aquellos nodos con los que  intercambian  información
            de manera frecuente \parencite[]{chilamkurthy2022low}.\\

            Entre las ventajas de la topología de malla  se  destaca  su  diseño
            redundante, el cual permite la existencia de rutas alternativas para
            la transmisión de datos, mitigando el problema del  punto  único  de
            fallo presente en la topología en estrella. Asimismo, este  tipo  de
            topología  soporta  el  intercambio  de  datos   bajo   un   enfoque
            \textit{full-duplex}  (FD),  lo  que   contribuye   a   mejorar   la
            escalabilidad de la red. Sin embargo, también  presenta  desventajas
            importantes, como el aumento de la complejidad debido a la presencia
            de múltiples enlaces entre  nodos,  el  incremento  de  la  latencia
            ocasionado por  la  comunicación  multi-salto,  el  mayor  costo  de
            implementación y una reducción en la eficiencia energética  derivada
            de su diseño redundante \parencite[]{chilamkurthy2022low}.
\end{itemize}

\textbf{Arquitecturas LPWAN}:

La arquitectura básica  de  una  red  LPWAN  está  compuesta  esencialmente  por
dispositivos finales, estaciones de acceso o gateways, el núcleo de red (core  o
network server) y servidores de aplicación, incluyendo la conexión a internet  y
servicios   en   la   nube   para   respaldo   y    procesamiento    de    datos
\parencite[]{chilamkurthy2022low}.

\begin{itemize}

      \item \textbf{Componentes principales de la arquitectura LPWAN}

            \begin{itemize}

                  \item \textbf{Dispositivos finales  (End  Devices):}  Su  función  principal  es
                        recopilar información relevante según la aplicación. Los datos  capturados
                        pueden  enviarse  a  capas  superiores  de   la   arquitectura   para   su
                        procesamiento o para generar una respuesta  según  la  implementación.  La
                        información recolectada se transmite comúnmente a la  estación  de  acceso
                        mediante un canal de radio dedicado, y posteriormente al backend de la red
                        IoT \parencite[]{chilamkurthy2022low}.

                  \item \textbf{Estación de acceso  /  Gateway  /  Concentrador:}  Proporciona  el
                        enlace  radioeléctrico  para  la  administración  de  dispositivos  y   el
                        intercambio de tráfico.  Garantiza  la  integridad  del  enlace  de  radio
                        considerando métricas como la tasa de error  de  bit  (BER),  seguridad  y
                        calidad de servicio (QoS). La estación base se comunica con el  gateway  o
                        concentrador, el  cual  en  algunos  casos  es  denominado  núcleo  (core)
                        \parencite[]{chilamkurthy2022low}.

                  \item \textbf{Núcleo de red (Core):} Es responsable del control  y  enrutamiento
                        del tráfico de usuario. Actúa como enlace entre la estación de acceso y la
                        red  IoT,  además  de  funcionar  como  traductor  entre  los   protocolos
                        utilizados por la estación de acceso y aquellos  soportados  por  la  red.
                        Dependiendo de la tecnología, el concentrador puede ofrecer capacidades de
                        computación en el borde y almacenamiento de datos para reducir la carga en
                        la nube, especialmente en aplicaciones que requieren asistencia en  tiempo
                        real y baja latencia. También puede proporcionar tratamiento  prioritario,
                        control de  admisión  robusto  y  soporte  de  movilidad  en  determinadas
                        tecnologías LPWAN \parencite[]{chilamkurthy2022low}.

                  \item \textbf{Servidor LPWAN y servidores de aplicación:} El servidor  LPWAN  se
                        encarga del aprovisionamiento, registro y operación de las entidades de la
                        red. Puede compartir o ampliar funcionalidades esenciales del núcleo  como
                        el enrutamiento  de  tráfico,  seguridad  y  manejo  de  prioridades.  Los
                        servidores de aplicación y la nube gestionan la base de datos que almacena
                        los mensajes recibidos de los dispositivos conectados, y pueden analizar y
                        actuar sobre los datos mediante técnicas de análisis de grandes  volúmenes
                        de información \parencite[]{chilamkurthy2022low}.

            \end{itemize}

      \item \textbf{Variantes arquitectónicas}

            Las LPWAN pueden organizarse bajo diferentes enfoques
            arquitectónicos, dependiendo de la tecnología, el entorno de despliegue y los requerimientos
            de la aplicación:

            \begin{enumerate}

                  \item \textbf{Arquitectura híbrida con tecnologías de acceso externas:} En  este
                        esquema,  tecnologías  como  ZigBee  o  Wi-Fi  proporcionan   conectividad
                        primaria al dispositivo mediante su propia  arquitectura.  Posteriormente,
                        el gateway de dicha red preliminar se interconecta con el punto de  acceso
                        de  la  LPWAN,  generando  una  arquitectura  mixta.   Este   enfoque   es
                        particularmente     común     en      LPWAN      de      tipo      celular
                        \parencite[]{chilamkurthy2022low}.

                  \item \textbf{Arquitectura híbrida multi-LPWAN:} Permite  que  los  dispositivos
                        tengan acceso a múltiples redes LPWAN (por ejemplo, SigFox y  LoRa).  Cada
                        LPWAN recopila datos de los dispositivos dentro de su zona  de  cobertura,
                        con estaciones base instaladas según el área cubierta.  El  tráfico  puede
                        transmitirse hacia el núcleo y la nube a través del gateway LPWAN o de los
                        nodos. En estas arquitecturas mixtas,  las  entidades  del  núcleo  o  del
                        servidor  de  red  gestionan  funciones  como   autenticación,   registro,
                        asignación      de      recursos      y      control      del      tráfico
                        \parencite[]{chilamkurthy2022low}.

                  \item \textbf{Arquitecturas   LPWAN   cognitivas:}    Se    investigan    nuevas
                        arquitecturas basadas en inteligencia artificial y aprendizaje  automático
                        para  crear   soluciones   LPWAN   cognitivas.   Estas   buscan   soportar
                        comunicaciones  sofisticadas,  gestionar  aplicaciones  IoT   diversas   y
                        habilitar  redes  definidas  por  software.  Permiten  la  coexistencia  e
                        interoperabilidad de múltiples tecnologías  LPWAN,  facilitando  servicios
                        inteligentes más eficientes en aplicaciones  como  ciudades  inteligentes,
                        Green  IoT,  monitoreo  de  salud,  hogares  inteligentes   y   conducción
                        automatizada \parencite[]{chilamkurthy2022low}.

            \end{enumerate}

\end{itemize}


\subsubsection{Análisis Comparativo de Tecnologías LPWAN} Las tecnologías de Red
de Área Extensa de Baja  Potencia  (LPWAN)  se  dividen  principalmente  en  dos
categorías según el espectro de frecuencia  que  utilizan:  banda  con  licencia
(celular), gestionada por el consorcio 3GPP, y banda  sin  licencia  (ISM),  que
engloba a  tecnologías  propietarias  y  abiertas.  Este  análisis  compara  sus
características técnicas clave, rendimiento e idoneidad para distintos objetivos
de diseño IoT.

\begin{itemize}
      \item \textbf{Tecnologías en Banda con Licencia (3GPP):} Estas tecnologías

            operan en espectro licenciado (principalmente 700-900 MHz),  lo  que
            garantiza  una  comunicación  confiable  sin  interferencias,   pero
            conlleva costos de suscripción. Son ideales  para  aplicaciones  que
            requieren  alta  confiabilidad,  movilidad  y   cobertura   nacional
            \parencite[]{chilamkurthy2022low}.
            \begin{itemize}
                  \item \textbf{LTE-M (LTE Cat M1 / eMTC):} Basada en LTE,
                        ofrece la mayor tasa de datos (hasta  1  Mbps),  soporta
                        movilidad (handover) y voz (VoLTE). Es la  más  adecuada
                        para aplicaciones que requieren latencia baja (150 ms) y
                        una mayor cantidad de datos, como rastreo de  activos  y
                        wearables,   aunque   con    un    consumo    energético
                        moderado-alto.
                  \item \textbf{NB-IoT:} Optimizada para IoT, utiliza un ancho
                        de  banda  muy  estrecho  (200  kHz).  Destaca  por   su
                        excepcional  cobertura  (hasta  15  km,  presupuesto  de
                        enlace de 164 dB) y  gran  profundidad  de  penetración,
                        siendo óptima para  sensores  estáticos  en  ubicaciones
                        remotas o subterráneas (e.g.,  medidores  inteligentes).
                        Su latencia es mayor (hasta 10 s).
                  \item \textbf{EC-GSM-IoT:} Una evolución de GSM/GPRS. Ofrece
                        un   equilibrio   entre   cobertura,   capacidad    (50k
                        dispositivos/celda)    y    coste,    aprovechando    la
                        infraestructura  GSM  existente.  Es  una  opción   para
                        modernizar redes M2M tradicionales.
            \end{itemize}

      \item \textbf{Tecnologías en Banda Sin Licencia (Non-3GPP):} Operan en
            bandas ISM sub-GHz (e.g., 868 MHz, 915 MHz) o 2.4 GHz, sin costos de
            espectro pero sujetas a restricciones  de  duty  cycle  y  potencial
            interferencia.   Permiten   el   despliegue   de   redes    privadas
            \parencite[]{chilamkurthy2022low}.
            \begin{itemize}
                  \item \textbf{SigFox:} Utiliza Banda Ultra Estrecha (UNB). Se
                        caracteriza por su extremo bajo costo, mayor alcance (50
                        km rural) y minimización del consumo energético (6 nA en
                        sleep). Es adecuada  para  aplicaciones  de  uplink  muy
                        esporádico con mensajes diminutos (12  bytes),  pero  su
                        escalabilidad de carga es limitada (140 mensajes/día)  y
                        la capacidad de downlink es muy reducida.
                  \item \textbf{LoRaWAN:} Utiliza modulación CSS de espectro
                        ensanchado.   Ofrece   un    buen    equilibrio    entre
                        \textbf{alcance} (15 km  rural),  consumo  energético  y
                        flexibilidad (diferentes clases de dispositivos  A/B/C).
                        Su  arquitectura  de  red  abierta  y  la  capacidad  de
                        desplegar gateways privados la hacen  muy  popular  para
                        redes IoT corporativas y de  ciudades  inteligentes.  La
                        tasa de datos es baja (0.3-50 kbps) y depende del factor
                        de ensanchamiento (SF).
                  \item \textbf{RPMA (Ingenu):} Opera en la banda de 2.4 GHz, lo

                        que  le  otorga  un  gran  ancho   de   banda   y   alta
                        escalabilidad estructural. Ofrece la mayor tasa de datos
                        entre las tecnologías sin licencia  (624  kbps  uplink).
                        Sin embargo, su alcance es limitado (10 km rural) debido
                        a  la  mayor  atenuación  de  la  frecuencia   y   sufre
                        interferencia de otras tecnologías (Wi-Fi, Bluetooth).
                  \item \textbf{Telensa:} Tecnología UNB especializada en
                        aplicaciones   de   control   como   alumbrado   público
                        inteligente.    Ofrece    comunicación     bidireccional
                        full-duplex y una vida útil de batería de ~8 años,  pero
                        con un alcance modesto (4  km  rural)  y  baja  tasa  de
                        datos.
                  \item \textbf{Weightless:} Conjunto de estándares abiertos.
                        Weightless-P es el más completo, ofreciendo comunicación
                        bidireccional  confiable  con   acknowledgments,   buena
                        gestión de interferencias y movilidad, comparable a  una
                        versión ligera de LTE para IoT.
                  \item \textbf{DASH7 (D7AP):} Protocolo derivado de RFID
                        activo. Se destaca por soportar movilidad,  comunicación
                        asíncrona  y  baja  latencia,  ideal  para  activos   en
                        movimiento como logística. Su  principio  de  diseño  es
                        BLAST    (Bursty,    Light,    Asynchronous,    Stealth,
                        Transitional).
                  \item \textbf{NB-Fi:} Enfocada en lograr la máxima
                        cobertura y penetración (hasta  30  km,  presupuesto  de
                        enlace de 174 dB) en banda sin licencia, con una tasa de
                        datos mínima (11 bps).  Es  adecuada  para  aplicaciones
                        donde  el  único  requisito  es  recibir  una  señal  de
                        sensores muy remotos.
            \end{itemize}

\end{itemize}

\textbf{Comparativa Sintética por  Objetivos  de  Diseño:}  La  elección  de  la
tecnología óptima implica negociar entre distintos  objetivos,  ya  que  ninguna
domina en todos los aspectos.

\begin{itemize}
      \item \textbf{Eficiencia Energética y Vida Útil:} Las tecnologías
            sin licencia (especialmente SigFox, LoRaWAN, D7AP) son  generalmente
            superiores. SigFox es líder absoluto en consumo en modo  sleep.  Las
            tecnologías celulares (NB-IoT, LTE-M) implementan  modos  de  ahorro
            (PSM, eDRX) para alcanzar vidas de ~10 años.
      \item \textbf{Costo Total:} Las tecnologías sin licencia tienen
            ventaja  al  eliminar  tarifas  de  suscripción  y  permitir   redes
            privadas. SigFox y NB-Fi ofrecen costos operativos  muy  bajos.  Las
            tecnologías  con  licencia  implican  un  costo   recurrente,   pero
            aprovechan infraestructura existente.
      \item \textbf{Cobertura y Penetración:} NB-IoT lidera en la banda
            con licencia. En la banda sin licencia, SigFox y NB-Fi  ofrecen  los
            mayores alcances, mientras que D7AP y Weightless-P tienen un alcance
            más corto.
      \item \textbf{Escalabilidad:}
            \begin{itemize}
                  \item \textit{Escalabilidad Estructural (dispositivos por
                              celda):} Superior en tecnologías sin licencia
                        como  D7AP,  Weightless-P,  LoRaWAN  y   SigFox   (hasta
                        millones). Las tecnologías celulares manejan decenas  de
                        miles (NB-IoT: ~50k, LTE-M: ~80k-1M).
                  \item \textit{Escalabilidad de Carga (mensajes por
                              dispositivo):} Superior en tecnologías con
                        licencia, al no tener restricciones de duty  cycle.  Las
                        tecnologías sin licencia están limitadas por  regulación
                        (e.g., 140 mensajes/día en SigFox, duty cycle del 1\% en
                        EU para LoRa).
            \end{itemize}
      \item \textbf{Manejo de Interferencias:} Las tecnologías con
            licencia están libres de interferencias accidentales. Entre las  sin
            licencia, LoRaWAN (CSS) y Weightless-P (saltos de frecuencia +  FEC)
            son muy robustas. Las que usan UNB (SigFox, NB-Fi) también minimizan
            el riesgo.
      \item \textbf{Calidad de Servicio (QoS), Movilidad y Latencia:} Las
            tecnologías con licencia (especialmente LTE-M y NB-IoT)  ofrecen  la
            mejor QoS garantizada, soporte nativo para handover y las  latencias
            más bajas. Entre las sin licencia, Weightless-P y D7AP son  las  que
            mejor   soportan   movilidad   y   comunicaciones    bidireccionales
            confiables.
\end{itemize}

\textbf{Discusión y Aplicabilidad:} No existe una tecnología LPWAN única  óptima
para todos los casos. La selección debe basarse en los requisitos específicos de
la aplicación: \begin{itemize}
      \item \textbf{Aplicaciones de Monitoreo Masivo y Estático} (e.g.,
            medidores,  agricultura):  NB-IoT  (por   cobertura/penetración)   o
            SigFox/LoRaWAN (por coste y energía).
      \item \textbf{Aplicaciones con Movilidad o Baja Latencia} (e.g.,
            logística,   wearables):   LTE-M   (mejor    opción    celular)    o
            D7AP/Weightless-P (en redes privadas).
      \item \textbf{Redes Privadas y Control Industrial} (e.g., ciudades
            inteligentes,  fabricas):  LoRaWAN  (flexibilidad  y  ecosistema)  o
            Weightless-P (QoS y confiabilidad).
      \item \textbf{Aplicaciones con Datos más Voluminosos o Voz} (e.g.,
            vigilancia, teleasistencia): LTE-M es la opción predominante.
\end{itemize} La tendencia futura apunta hacia la convergencia y coexistencia de
múltiples tecnologías en arquitecturas híbridas, donde una plataforma de gestión
unificada pueda seleccionar dinámicamente la mejor conexión disponible para cada
dispositivo y aplicación.



\subsubsection{Aplicación en el Monitoreo Agrícola} Las  LPWAN  proporcionan  la
\textbf{base tecnológica} sobre la cual se sustentan los sistemas de Agricultura
de Precisión. Su  aplicación  en  el  monitoreo  agrícola  es  crucial  por  las
siguientes razones, que justifican la necesidad  de  seleccionar  la  tecnología
adecuada:

\begin{itemize}
      \item \textbf{Viabilidad en Campos Extensos:} La capacidad de largo
            alcance permite conectar sensores distribuidos en  grandes  parcelas
            de cultivo donde otras tecnologías son  inviables  u  extremadamente
            caras de implementar. Esto es fundamental para obtener datos de alta
            resolución espacial \parencite[]{chen2018cognitive}.
      \item \textbf{Sostenibilidad Operativa:} El bajo consumo energético
            garantiza que los sensores permanezcan operativos  durante  periodos
            prolongados  (años)  sin  intervención  humana   para   recargar   o
            reemplazar baterías, reduciendo  significativamente  los  costos  de
            mantenimiento y operación.
      \item \textbf{Insumo para la Inteligencia Artificial (IA):} Las LPWAN son
            el canal de comunicación que traslada los datos  críticos  de  campo
            (clima, suelo, estado hídrico) hacia  las  plataformas  de  IA,  las
            cuales  procesan  esta  información  para   generar   prescripciones
            agronómicas \parencite[]{chen2018cognitive}. La selección errónea de
            la LPWAN puede resultar en fallas de conectividad, comprometiendo la
            calidad  de  los  datos  y,  por   ende,   la   precisión   de   las
            recomendaciones del sistema inteligente.
\end{itemize}

\subsection{Sensórica y Variables Agronómicas}

La agricultura de precisión se fundamenta en la  captura  sistemática  de  datos
agronómicos mediante sensores, los cuales permiten  monitorear  las  condiciones
del cultivo y el ambiente en tiempo real o casi real. Esta capacidad de medición
precisa y continua resulta esencial para la toma de  decisiones  informadas,  la
optimización de insumos y la mitigación de riesgos asociados a  la  variabilidad
climática. En el contexto del  despliegue  de  redes  LPWAN  para  el  monitoreo
agrícola, la selección  de  la  sensórica  adecuada  y  la  comprensión  de  las
variables que  miden  constituyen  un  paso  crítico  previo  al  diseño  de  la
arquitectura de comunicación.

\subsubsection{Sensórica en  la  agricultura  de  presición}  Los  sensores  son
dispositivos que convierten señales del  mundo  físico  (por  ejemplo,  humedad,
temperatura, composición química) en datos digitales, actuando como la  interfaz
primaria entre el campo y los sistemas de información (Applying IoT Sensors  and
Big Data to Improve  Precision  Crop.pdf).  Su  integración  en  componentes  de
maquinaria, suelo, plantas o animales proporciona  información  vital  sobre  el
estado del sistema agroproductivo \parencite[]{alahmad2023applying}.

Para el desarrollo de sistemas IoT agrícolas (Ag-IoT), la selección  del  sensor
apropiado debe considerar factores como bajo consumo de energía,  compatibilidad
en la transferencia de información,  precisión,  sensibilidad,  repetibilidad  y
durabilidad \parencite[]{alahmad2023applying}. Existe  una  amplia  variedad  de
sensores clasificables según el parámetro físico que miden:

\begin{itemize}
      \item \textbf{Sensores Químicos:} Miden propiedades como el pH del suelo y
            agua, conductividad eléctrica, salinidad, y concentraciones de gases
            (CO2, O2, CH4) y nutrientes (nitratos). Se dividen principalmente en
            fotodetectores y electroquímicos \parencite[]{alahmad2023applying}.
      \item \textbf{Sensores Ópticos:} Utilizan la reflectancia de la luz en
            diferentes longitudes de onda para determinar materia  orgánica  del
            suelo, humedad, color, contenido de  clorofila  en  plantas,  estrés
            hídrico y  detección  de  enfermedades  foliares.  Tecnologías  como
            cámaras multiespectrales, hiperespectrales y el Índice de Vegetación
            de Diferencia Normalizada (NDVI) son clave para la  teledetección  y
            estimación de rendimiento \parencite[]{alahmad2023applying}.
      \item \textbf{Sensores Electromecánicos y de Humedad:} Incluyen sensores
            de humedad del suelo (que miden la constante dieléctrica),  sensores
            de presión, acelerómetros y celdas de carga. Son fundamentales  para
            medir la compactación del suelo, el crecimiento de frutos, el viento
            y el peso continuo de las plantas \parencite[]{alahmad2023applying}.

      \item \textbf{Sensores Acústicos:} Detectan cambios en las frecuencias
            sonoras,  útiles  para  identificar  plagas  (como  barrenadores  de
            madera) mediante el sonido que generan al alimentarse o  moverse,  y
            para    estimar    la    altura    del     dosel     del     cultivo
            \parencite[]{alahmad2023applying}.
      \item \textbf{Sensores Térmicos:} Monitorean la temperatura de hojas,
            suelo y ambiente. Los datos de temperatura foliar se  utilizan  para
            predecir la producción, estimar la evapotranspiración y programar el
            riego \parencite[]{alahmad2023applying}.
\end{itemize}

\subsubsection{ Variables Agronómicas Críticas para el Monitoreo} Las  variables
agronómicas monitoreables definen el estado de salud del cultivo  y  del  suelo,
guiando las  intervenciones  de  manejo.  La  cantidad  y  frecuencia  de  datos
requeridos dependen del cultivo específico, las condiciones  ambientales  y  los
objetivos del productor  \parencite[]{alahmad2023applying}.  Las  variables  más
relevantes, vinculadas a los sensores antes descritos, incluyen:

\begin{itemize}
      \item \textbf{Variables Edáficas (del Suelo):}
            \begin{itemize}
                  \item \textbf{Humedad del Suelo: }Parámetro fundamental para
                        la programación eficiente del riego. Se mide  comúnmente
                        con sensores  capacitivos  o  de  resistencia  eléctrica
                        \parencite[]{alahmad2023applying}.
                  \item \textbf{Nutrientes (N, P, K) y pH:} Determinan la
                        fertilidad del suelo y la  necesidad  de  fertilización.
                        Sensores electroquímicos y ópticos permiten estimaciones
                        en el sitio, aunque su medición en tiempo real sigue  siendo
                        un           desafío            en            desarrollo
                        \parencite[]{alahmad2023applying}.
                  \item \textbf{Temperatura del Suelo: }Afecta la germinación de
                        semillas y la actividad microbiana. Se mide con sensores
                        térmicos \parencite[]{alahmad2023applying}.
                  \item \textbf{Conductividad Eléctrica (CE):} Indica la
                        salinidad del suelo y la concentración de iones. Se mide
                        con                 sensores                  eléctricos
                        \parencite[]{alahmad2023applying}.
            \end{itemize}
      \item \textbf{Variables Ambientales y Climáticas:}
            \begin{itemize}
                  \item \textbf{Temperatura y Humedad del Aire:} Condicionan la
                        evapotranspiración, la aparición de  enfermedades  y  el
                        estrés  térmico  de  los  cultivos.  Se  monitorean  con
                        sensores  térmicos   y   de   humedad   (capacitivos   o
                        resistivos) \parencite[]{alahmad2023applying}.
                  \item \textbf{Precipitación y Viento: }Afectan la programación

                        de riego, la aplicación de agroquímicos y los riesgos de
                        erosión.  Se  miden  con  pluviómetros   y   anemómetros
                        (sensores         mecánicos         o         acústicos)
                        \parencite[]{alahmad2023applying}.
                  \item \textbf{Radiación Solar:} Determina la fotosíntesis y el
                        crecimiento.  Se  mide  con  sensores  de  radiación   o
                        piranómetros \parencite[]{alahmad2023applying}.

            \end{itemize}
\end{itemize}



\subsection{Simulación de Redes LPWAN}

\subsubsection{Estado del Arte: Herramientas de Simulación y sus Métricas de
      Desempeño}

El siguiente análisis se basa en el artículo: \textit{A Survey on LoRaWAN
      Technology: Recent Trends,  Opportunities,  Simulation  Tools  and  Future
      Directions} \parencite[]{almuhaya2022survey}. La simulación de redes LPWAN
es crucial para evaluar el rendimiento  de  la  red  a
gran escala. A continuación, se detallan las herramientas de simulación tratadas
en el artículo y sus métricas de desempeño.

\begin{table}[H]
      \centering
      \small % Reducimos un poco el tamaño de letra para que respire la tabla
      \begin{tabularx}{\textwidth}{@{} p{2.2cm} >{\raggedright\arraybackslash}X >{\raggedright\arraybackslash}X >{\raggedright\arraybackslash}X @{}}
            \toprule
            \textbf{Herramienta} & \textbf{Descripción y Plataforma}         & \textbf{Características Clave}                                                                                  & \textbf{Métricas de Desempeño}                \\ \midrule

            LoRaSim              & Simulador de eventos discretos en Python. & Simula un único gateway para varios dispositivos finales (EDs). Incorpora efecto de captura y asignación de SF. & Packet Reception Ratio (PRR) y escalabilidad. \\ \addlinespace

            LoRaWANSim           & Módulo para ns-3 basado en C++.           & Soporta comunicación multicanal, multi-gateway y bidireccional. Incluye Clases A y C.                           & Consumo de energía, PDR y latencia.           \\ \addlinespace

            FAD                  & Simulador de eventos discretos en Python. & Enfocado en gran escala. Introduce un modelo de canal realista y considera el efecto de captura.                & PDR y consumo de energía.                     \\ \addlinespace

            Simu-LoRa            & Implementado en MATLAB.                   & Diseñado para evaluar la capa física (PHY) y diferentes factores de dispersión (SF).                            & Eficiencia energética y PDR.                  \\ \bottomrule
      \end{tabularx}
      \caption{Comparativa de herramientas de simulación para redes LoRaWAN.}
      \label{tab:lorawan_tools}
\end{table}



\subsection{Agentes de Inteligencia Artificial}

\subsubsection{Inteligencia Artificial Generativa}

La inteligencia artificial generativa corresponde a una clase de sistemas de  IA
diseñados para producir nuevo contenido, como texto, código  o  representaciones
simbólicas a partir de patrones aprendidos en grandes volúmenes de  datos.  Este
tipo de sistemas se basa principalmente en modelos de aprendizaje  profundo,  en
particular modelos de lenguaje de gran  escala  (Large  Language  Models,  LLM),
entrenados   sobre   corpus   extensos    y    heterogéneos    de    información
\parencite[]{feuerriegel2024generative}.

La  característica  distintiva  de  la  IA  generativa  es   su   capacidad   de
generalización, permitiéndole responder a tareas no vistas previamente  mediante
razonamiento contextual y generación probabilística de salidas  plausibles,  más
que mediante reglas determinísticas \parencite[]{feuerriegel2024generative}.  En
este sentido, los sistemas generativos no “recuperan” respuestas  exactas,  sino
que construyen resultados en función del contexto de entrada y del  conocimiento
internalizado durante el entrenamiento.

Asimismo,  estos  modelos  presentan  limitaciones   inherentes,   como   sesgos
aprendidos de los datos, ausencia de comprensión semántica en sentido  humano  y
la  imposibilidad  de  garantizar  la  corrección  factual   de   sus   salidas.
\parencite[]{feuerriegel2024generative} Por esta razón, los sistemas basados  en
IA generativa deben integrarse dentro de arquitecturas más  amplias,  donde  sus
resultados  puedan  ser  evaluados,  validados  o   complementados   por   otros
componentes del sistema.

\subsubsection{Agentes de Inteligencia Artificial}

Los agentes de inteligencia artificial representan una evolución  conceptual  de
los sistemas de IA  tradicionales,  al  incorporar  explícitamente  nociones  de
autonomía, percepción, acción y orientación a objetivos. Un agente  autónomo  es
una entidad computacional capaz de operar de manera  independiente  en  entornos
dinámicos, gestionando su estado interno y tomando decisiones  sin  intervención
humana directa \parencite[]{he2025llm}.

En el documento de \textit{LLM-Based Multi-Agent Systems for Software
      Engineering} define como atributos esenciales de un agente:

\begin{itemize}
      \item Autonomía,
      \item Percepción del entorno,
      \item Capacidad de razonamiento orientado a objetivos,
      \item Interacción social con otros agentes o humanos, y
      \item Capacidad de aprendizaje y adaptación.
\end{itemize}

En el caso particular de los agentes basados en LLM, estos integran un modelo de
lenguaje como núcleo cognitivo del  agente,  complementado  con  componentes  de
memoria, percepción y ejecución  de  acciones. De  modo  que  estos  agentes  son
sistemas que utilizan el LLM  para  razonar  sobre  observaciones,  objetivos  y
retroalimentación,  permitiéndoles  planificar  y  actuar  de  forma   iterativa
\parencite[]{he2025llm}.

No obstante, pese a sus capacidades,  los  agentes  basados  en  LLM  no  poseen
entendimiento  real  del  entorno,  sino   que   operan   mediante   inferencias
estadísticas, lo que implica riesgos cuando se les otorga  mayor  autonomía  sin
mecanismos de control adecuados \parencite[]{he2025llm}.

\subsubsection{Sistemas Multiagente}

Un sistema multiagente se define  como  un  marco  computacional  compuesto  por
múltiples agentes inteligentes que interactúan entre sí con el fin  de  resolver
problemas complejos que no pueden ser  abordados  eficientemente  por  un  único
agente. Estos sistemas  se  caracterizan  por  la  cooperación,  coordinación  y
especialización funcional de sus agentes \parencite[]{he2025llm}.

Con la integración de modelos de lenguaje, han surgido los sistemas  multiagente
basados en LLM, los cuales se componen de dos elementos principales:

\begin{itemize}
      \item Una plataforma de orquestación, encargada de coordinar las
            interacciones.
      \item Un conjunto de agentes especializados, cada uno con roles y
            responsabilidades diferenciadas.
\end{itemize}

Estos sistemas permiten dividir tareas  complejas  en  subtareas,  asignarlas  a
agentes con diferentes roles y combinar los resultados  mediante  mecanismos  de
comunicación estructurada. Sin embargo, el estudio resalta que la  investigación
en este campo aún es incipiente y que existen desafíos abiertos relacionados con
la escalabilidad, la coherencia global del sistema y la evaluación del desempeño
colectivo \parencite[]{he2025llm}.

\subsubsection{Gobernanza de Agentes de Inteligencia Artificial}

La gobernanza de agentes de  IA  surge  como  un  campo  emergente  orientado  a
gestionar los riesgos, impactos y responsabilidades asociados con el  despliegue
de agentes autónomos. El documento Agent Governance: A  Field  Guide  define  la
gobernanza de agentes como el conjunto de mecanismos técnicos, organizacionales,
legales y sociales destinados a asegurar que los agentes operen de forma segura,
transparente      y      alineada      con      los      objetivos       humanos
\parencite[]{kraprayoon2025ai}.

El texto enfatiza que los agentes presentan desafíos particulares frente a otros
sistemas de IA debido a su capacidad de tomar múltiples decisiones encadenadas a
lo largo del tiempo, interactuar con herramientas externas y  actuar  con  menor
supervisión humana directa. Esto dificulta la atribución de responsabilidades  y
aumenta la complejidad del control del sistema \parencite[]{kraprayoon2025ai}.

Como  respuesta,  el  documento  propone  una  taxonomía  de  intervenciones  de
gobernanza, que incluye categorías como:

\begin{itemize}
      \item Alineación,
      \item Control,
      \item Visibilidad,
      \item Seguridad y robustez, y
      \item Integración social
\end{itemize} orientadas a  mitigar  riesgos  y  facilitar  la  supervisión  del
comportamiento de los agentes \parencite[]{kraprayoon2025ai}.

\subsection{Agilismo, Kanban y Modelo de Prototipo}

Las metodologías ágiles representan un conjunto de prácticas y marcos de trabajo
cuyo objetivo principal es la entrega rápida y continua  de  valor  al  cliente,
fomentando la adaptabilidad y la respuesta al cambio. El desarrollo de  software
ágil y su implementación se centran en el estudio de cómo estos enfoques  pueden
afectar el trabajo de un equipo de desarrollo. Uno  de  los  marcos  ágiles  más
populares es Kanban, el cual puede ser usado en múltiples contextos más allá del
desarrollo de software \parencite[]{grotenfelt2021agile}.

De  manera  general,  el  enfoque  ágil  se  caracteriza  por  ser  iterativo  e
incremental, priorizando entregas parciales funcionales en  lugar  de  un  único
producto final al término  del  proyecto.  Asimismo,  promueve  la  colaboración
constante  entre  equipos  multidisciplinarios,  la  participación  activa   del
cliente, la adaptabilidad frente  a  cambios  en  los  requisitos  y  la  mejora
continua    de    los    procesos    \parencite[]{hossain2023software}.    Estas
características permiten que los equipos  respondan  con  mayor  flexibilidad  a
entornos dinámicos y a necesidades cambiantes.

\subsubsection{Framework de Kanban}

Kanban se centra en el flujo del trabajo, la  visualización  del  proceso  y  la
limitación del trabajo  en  curso  (WIP).  El  nombre  proviene  del  japonés  y
significa “señal visual” \parencite[]{grotenfelt2021agile}.

En este marco, las tareas se representan mediante tarjetas que  se  desplazan  a
través de distintas columnas en un tablero Kanban,  reflejando  las  etapas  del
proceso de desarrollo. Una de sus prácticas fundamentales es limitar el  trabajo
en progreso (WIP) para evitar la sobrecarga del equipo y mejorar  la  eficiencia
del flujo. El ingreso de nuevas  tareas  al  sistema  depende  de  la  capacidad
disponible,   lo   que   favorece   una   entrega    continua    y    sostenible
\parencite[]{hossain2023software}.  Kanban  resulta  especialmente  adecuado  en
contextos donde existe un flujo constante de solicitudes o cambios,  permitiendo
gestionar el trabajo de manera dinámica.

\subsubsection{Procesos centrales de Kanban}

\begin{itemize}
      \item Visualización del flujo de trabajo mediante el tablero Kanban.
      \item Limitación del WIP para evitar sobrecarga y mejorar la eficiencia
            del flujo.
      \item Gestión continua del flujo de trabajo basada en la capacidad disponible.
      \item Hacer explícitas las políticas del proceso
      \item Implementar ciclos de feedback.
      \item Mejora colaborativa.
\end{itemize} \parencite{hossain2023software}

\subsubsection{Características del Enfoque Ágil}

Entre las principales características del enfoque ágil se destacan:

\begin{itemize}
      \item Desarrollo iterativo e incremental.
      \item Colaboración constante entre equipos multidisciplinarios.
      \item Enfoque centrado en el cliente y en la entrega temprana de valor.
      \item Flexibilidad y adaptabilidad ante cambios en requisitos.
      \item Entrega continua de incrementos funcionales.
      \item Equipos autoorganizados y empoderados.
      \item Transparencia en el progreso y en la toma de decisiones.
      \item Mejora continua mediante retrospectivas y ciclos de
            retroalimentación.
      \item Uso de herramientas modernas como integración continua y pruebas
            automatizadas.
      \item Énfasis en software funcional sobre documentación extensa.
\end{itemize} \parencite[]{grotenfelt2021agile}

\subsubsection{Ventajas del Modelo Ágil}

Las metodologías ágiles presentan diversas ventajas, entre ellas:

\begin{itemize}
      \item Alta flexibilidad frente a cambios.
      \item Mayor alineación con las necesidades del cliente.
      \item Entregas frecuentes de valor.
      \item Mitigación temprana de riesgos mediante iteraciones cortas.
      \item Mejora en la calidad del software gracias a prácticas continuas de
            prueba.
      \item Mayor colaboración y comunicación dentro del equipo.
      \item Transparencia en el estado del proyecto.
      \item Reducción de sobrecarga documental.
\end{itemize} \parencite[]{hossain2023software}

\subsubsection{Desventajas del Modelo Ágil}

No obstante, el modelo ágil también presenta limitaciones:

\begin{itemize}
      \item Menor predictibilidad en tiempos y costos.
      \item Complejidad en proyectos de gran escala.
      \item Requiere equipos con experiencia en prácticas ágiles.
      \item Posible aumento del alcance (scope creep) si no se gestiona
            adecuadamente.
      \item Dificultad para realizar estimaciones precisas al inicio del
            proyecto.
      \item Dependencia de la participación constante del cliente.
      \item Posibles desafíos en documentación formal para entornos regulados.
\end{itemize} \parencite[]{hossain2023software}

\subsubsection{Métricas Ágiles}

Las métricas permiten inspeccionar  el  progreso  y  detectar  oportunidades  de
mejora. En entornos ágiles, la medición del desempeño se orienta tanto al avance
del   producto    como    a    la    eficiencia    del    flujo    de    trabajo.
\subsubsection{Métricas de Flujo y Eficiencia}

\begin{itemize}
      \item Tiempo de ciclo (cycle time): mide el tiempo desde que una tarea
            entra en estado “en progreso” hasta que se completa.
      \item Tiempo de espera (lead time): mide el tiempo total desde que se
            solicita una tarea hasta su entrega final.
\end{itemize} \parencite[]{polk2011agile,}


\subsubsection{Modelo de Prototipado (Prototyping Model)}

El Modelo de Prototipado es un enfoque de desarrollo de software que  se  centra
en la creación rápida de un modelo funcional o  prototipo  con  el  objetivo  de
recopilar retroalimentación temprana de los usuarios y  refinar  los  requisitos
antes de desarrollar el producto  final.  Este  modelo  se  basa  en  la  mejora
iterativa del prototipo a partir  de  la  interacción  con  los  usuarios  hasta
alcanzar      la      funcionalidad       y       el       diseño       deseados
\parencite[]{hossain2023software}.

El objetivo principal del modelo es garantizar que el software final  se  alinee
estrechamente con las  necesidades  y  expectativas  de  los  usuarios.  Resulta
especialmente útil cuando los  requisitos  son  poco  claros,  están  sujetos  a
cambios  o  cuando  se  requiere  una  validación  temprana  por  parte  de  los
interesados. Además, favorece una mejor comunicación entre los stakeholders y el
equipo de desarrollo, contribuyendo a la construcción de un producto más exitoso
y orientado al usuario \parencite[]{hossain2023software}.

\subsubsection{Características del Modelo de Prototipado}

El modelo de prototipado se distingue por las siguientes características:

\begin{itemize}
      \item \textbf{Desarrollo Iterativo:} Se basa en ciclos repetitivos de
            creación  de  prototipos,  recopilación   de   retroalimentación   y
            refinamiento    continuo.    Cada    iteración    permite    mejorar
            progresivamente  el  sistema  hasta  ajustarlo  a   los   requisitos
            esperados \parencite[]{hossain2023software}.
      \item \textbf{Desarrollo Rápido de Prototipos:} Se enfatiza la
            construcción rápida de modelos funcionales o maquetas,  generalmente
            centradas en aspectos específicos de la interfaz o funcionalidad.
      \item \textbf{Enfoque Centrado en el Usuario:} Los usuarios participan
            activamente    evaluando    el    prototipo     y     proporcionando
            retroalimentación, asegurando que  el  producto  final  refleje  sus
            expectativas \parencite[]{hossain2023software}.
      \item \textbf{Flexibilidad en los Requisitos:} Es adecuado para entornos
            donde  los  requisitos   evolucionan,   permitiendo   modificaciones
            continuas a partir de la retroalimentación obtenida.
      \item \textbf{Mejora de la Comunicación:} La representación visual e
            interactiva   del   sistema   facilita    la    comprensión    entre
            desarrolladores   y   stakeholders,   reduciendo   ambigüedades    y
            malentendidos \parencite[]{hossain2023software}.
\end{itemize}

\subsubsection{Ventajas del Modelo de Prototipado}

Entre las principales ventajas del Modelo de Prototipado se encuentran:

\begin{itemize}
      \item \textbf{Desarrollo Centrado en el Usuario:} Incrementa la
            probabilidad de satisfacción del usuario al involucrarlo activamente
            en el proceso \parencite[]{hossain2023software}.
      \item \textbf{Clarificación de Requisitos:} Permite identificar y resolver

            ambigüedades tempranamente, generando especificaciones más precisas.

      \item \textbf{Reducción de Riesgos y Errores Costosos:} La detección
            temprana de fallos  de  diseño  o  usabilidad  disminuye  retrabajos
            posteriores \parencite[]{hossain2023software}.
      \item \textbf{Aceleración del Desarrollo:} Las iteraciones rápidas
            posibilitan ajustes  ágiles  y  avances  simultáneos  en  diferentes
            componentes.
      \item \textbf{Mejora en la Comunicación:} El prototipo actúa como
            herramienta tangible de interacción entre los actores  del  proyecto
            \parencite[]{hossain2023software}.
\end{itemize}

\subsubsection{Desventajas del Modelo de Prototipado}

A pesar de sus beneficios, el modelo presenta ciertas limitaciones:

\begin{itemize}
      \item \textbf{Escalabilidad Limitada:} Puede resultar difícil de gestionar

            en proyectos grandes y complejos \parencite[]{hossain2023software}.
      \item \textbf{Costos Potencialmente Elevados:} La construcción de
            múltiples prototipos  y  ciclos  iterativos  puede  incrementar  los
            recursos necesarios.
      \item \textbf{Funcionalidades Incompletas:} Los prototipos pueden no
            reflejar completamente la complejidad del sistema final.
      \item \textbf{Riesgo de Retrasos:} Las iteraciones sucesivas pueden
            extender el cronograma si no se controlan adecuadamente.
      \item \textbf{Posible Mala Interpretación:} Algunos usuarios pueden
            confundir el prototipo con el producto final, generando expectativas
            irreales \parencite[]{hossain2023software}.
\end{itemize}

En conjunto, el Modelo de Prototipado constituye un enfoque iterativo,  flexible
y altamente centrado en el usuario, adecuado para contextos donde la  definición
inicial de requisitos es incierta y donde la validación temprana resulta crítica
para el éxito del proyecto \parencite[]{hossain2023software}.

\section{Marco Conceptual}

\subsection{Agricultura  4.0}  Integración  de  tecnologías   emergentes   (IoT,
analítica de  datos,  automatización  e  inteligencia  artificial)  en  sistemas
productivos   agrícolas   para   optimizar   procedimientos,   incrementar    la
productividad y fortalecer la sostenibilidad. En Colombia, este proceso  implica
no solo una transformación técnica, sino  también  cambios  en  la  organización
productiva, la formación de capital humano  y  la  gobernanza  del  conocimiento
agrícola \parencite[]{shafi2019precision}.

\subsection{Brecha  digital  rural}  Desigualdad  en  el  acceso  a  tecnologías
digitales entre agricultores tecnificados y pequeños productores  tradicionales,
manifestada no solo en infraestructura, sino también en alfabetización  digital,
acceso a  plataformas  inteligentes  y  capacidad  de  inversión  en  innovación
tecnológica \parencite[]{quiroz2023agricultura, alvarez2022colombian}.

\subsection{Agricultura de Precisión} Estrategia de gestión basada en la captura
y análisis de datos espacio-temporales, con el objetivo de optimizar  decisiones
relativas al uso de insumos, manejo de suelos y gestión de cultivos.  Constituye
un      fundamento      metodológico      de      la       Agricultura       4.0
\parencite[]{quiroz2023agricultura}.

\subsection{Internet de las Cosas (IoT) en agricultura} Componente clave  de  la
Agricultura de Precisión que integra sensores ambientales, redes de comunicación
de baja potencia y plataformas de visualización y análisis de datos, permitiendo
transformar  procesos  tradicionales  en   sistemas   de   control   inteligente
\parencite[]{Cognitive-LPWAN, florez2021agroindustria}.

\subsection{Redes  de  Área  Amplia  de  Baja  Potencia  (LPWAN)}  Categoría  de
tecnologías de comunicación inalámbrica diseñadas para  resolver  el  compromiso
entre bajo consumo de energía y largo alcance en conectividad IoT. Sus objetivos
de  diseño  incluyen  largo  rango  de   transmisión,   eficiencia   energética,
escalabilidad, bajo costo, calidad  de  servicio  y  gestión  de  interferencias
\parencite[]{diane2025systematic, chen2018cognitive}.

\subsection{Topologías de red LPWAN} Estructuras de conexión en redes LPWAN:  la
topología en estrella conecta todos los nodos a un gateway  central,  ofreciendo
simplicidad y eficiencia energética  pero  con  un  punto  único  de  fallo;  la
topología en  malla  permite  rutas  alternativas  mediante  nodos  enrutadores,
aumentando  la  redundancia  pero  con  mayor  latencia  y  consumo   energético
\parencite[]{chilamkurthy2022low}.

\subsection{Arquitectura LPWAN} Estructura funcional compuesta por  dispositivos
finales (nodos sensores), gateways, núcleo de red y  servidores  de  aplicación.
Puede ser básica  (estrella),  híbrida  con  tecnologías  externas,  multi-LPWAN
(múltiples  tecnologías  coexistiendo)   o   cognitiva   (con   capacidades   de
inteligencia        artificial        para        optimización         dinámica)
\parencite[]{chilamkurthy2022low}.

\subsection{Tecnologías LPWAN} Clasificadas según  el  espectro:  en  banda  con
licencia (3GPP) como LTE-M, NB-IoT y EC-GSM-IoT, que ofrecen alta  confiabilidad
y cobertura con costos de suscripción; en banda sin licencia (ISM) como  SigFox,
LoRaWAN, RPMA, Telensa, Weightless, DASH7 y NB-Fi, que permiten  redes  privadas
con menor costo pero sujetas a interferencias  y  restricciones  de  duty  cycle
\parencite[]{chilamkurthy2022low}.

\subsection{Sensórica agrícola}  Dispositivos  que  convierten  señales  físicas
(humedad,  temperatura,  composición  química,  etc.)  en  datos  digitales.  Se
clasifican  en  sensores  químicos,  ópticos,  electromecánicos,   acústicos   y
térmicos, y su selección debe considerar bajo consumo, precisión  y  durabilidad
para aplicaciones IoT en agricultura \parencite[]{alahmad2023applying}.

\subsection{Variables agronómicas} Parámetros medibles que definen el estado del
cultivo y el entorno. Incluyen  variables  edáficas  (humedad,  nutrientes,  pH,
temperatura, conductividad eléctrica), ambientales (temperatura  y  humedad  del
aire, precipitación, viento, radiación solar)  y  del  cultivo  (salud  vegetal,
estrés,      enfermedades,      estado      fenológico      y       rendimiento)
\parencite[]{alahmad2023applying, pyingkodi2022sensor}.

\subsection{Simulación de redes LPWAN} Herramientas computacionales para evaluar
el rendimiento  de  redes  a  gran  escala,  como  LoRaSim,  LoRaWANSim,  FAD  y
Simu-LoRa. Permiten medir métricas como Packet Delivery Ratio (PDR), consumo  de
energía, latencia y escalabilidad,  facilitando  el  diseño  y  optimización  de
despliegues IoT \parencite[]{almuhaya2022survey}.

\subsection{Inteligencia Artificial Generativa} Clase de sistemas de IA  basados
en modelos de aprendizaje profundo (especialmente Large Language Models) capaces
de  producir  nuevo  contenido  (texto,  código,  representaciones   simbólicas)
mediante  razonamiento  contextual  y  generación  probabilística,  aunque   con
limitaciones     como     sesgos     y     falta     de     garantía     factual
\parencite[]{feuerriegel2024generative}.

\subsection{Agente de Inteligencia Artificial}  Entidad  computacional  autónoma
que percibe su entorno, razona  orientada  a  objetivos,  interactúa  con  otros
agentes o humanos, y aprende y se adapta. En agentes basados en LLM,  el  modelo
de lenguaje actúa como núcleo cognitivo, complementado con memoria y capacidades
de acción \parencite[]{he2025llm}.

\subsection{Sistema Multiagente} Marco  computacional  compuesto  por  múltiples
agentes inteligentes que interactúan (cooperan, se coordinan y se  especializan)
para resolver problemas complejos. En  sistemas  basados  en  LLM,  incluye  una
plataforma de  orquestación  y  agentes  especializados  que  dividen  tareas  y
combinan resultados \parencite[]{he2025llm}.

\subsection{Gobernanza de  Agentes  de  IA}  Conjunto  de  mecanismos  técnicos,
organizacionales, legales y sociales  destinados  a  asegurar  que  los  agentes
autónomos operen de forma segura, transparente y alineada con objetivos humanos.
Incluye categorías como alineación, control, visibilidad, seguridad y  robustez,
e integración social \parencite[]{kraprayoon2025ai}.

\subsection{Metodologías ágiles} Conjunto  de  prácticas  y  marcos  de  trabajo
caracterizados por  el  desarrollo  iterativo  e  incremental,  la  colaboración
constante, la adaptabilidad al cambio y la entrega continua de valor al cliente.
Promueven   equipos   autoorganizados   y   transparencia   en    el    progreso
\parencite[]{hossain2023software}.

\subsection{Kanban} Marco ágil centrado en la visualización del flujo de trabajo
mediante un  tablero  con  columnas  que  representan  etapas  del  proceso,  la
limitación del trabajo en progreso (WIP) y la  gestión  continua  basada  en  la
capacidad disponible, mejorando la eficiencia del flujo y  permitiendo  entregas
sostenibles \parencite[]{grotenfelt2021agile, hossain2023software}.

\subsection{Modelo  de  Prototipado}  Enfoque  de  desarrollo  de  software  que
consiste en la creación rápida de modelos funcionales  o  maquetas  (prototipos)
para recopilar retroalimentación temprana de los usuarios y  refinar  requisitos
de forma iterativa. Es especialmente útil cuando los requisitos son inciertos  o
cambiantes,    y    favorece    la    comunicación    con    los    stakeholders
\parencite[]{hossain2023software}.
\chapter{Metodología}

El desarrollo del proyecto se fundamenta  en  una  metodología  ágil  basada  en
Kanban, combinada con un modelo de prototipado iterativo. Este  enfoque  permite
gestionar  el  flujo  de  trabajo  de  manera   continua,   facilitar   entregas
incrementales y adaptar el sistema de acuerdo con la retroalimentación  obtenida
durante el proceso de desarrollo. \\


\section{Gestión del Proyecto mediante Kanban} El proyecto se gestiona  mediante
un tablero Kanban, en el cual las actividades se  organizan  en  función  de  su
estado de avance. El uso de Kanban permite: 
\begin{itemize}
    \item Visualizar el progreso del proyecto.
    \item Priorizar actividades de manera dinámica.
    \item Facilitar entregas parciales y funcionales.
    \item Controlar el flujo de trabajo durante todo el desarrollo.
\end{itemize}

\noindent Por otro lado, el tablero propuesto para el proyecto se estructura de la siguiente manera:

\begin{itemize}
    \item Backlog
    \item Análisis
    \item Diseño
    \item Desarrollo
    \item Pruebas
    \item Finalizado
\end{itemize}

Se establecen límites de trabajo en progreso (WIP) con el fin de evitar sobrecarga y mantener eficiencia en la ejecución. La gestión del trabajo se realiza mediante priorización dinámica del backlog, permitiendo adaptar el desarrollo según hallazgos técnicos y validaciones parciales.

Kanban es utilizado exclusivamente como mecanismo de gestión del flujo de tareas, mientras que la estructura técnica del desarrollo se definió mediante el modelo de prototipado.

\section{Planificación inicial}

La fase de planificación se desarrolla previo al inicio del ciclo  de  vida  del
software y tiene como propósito establecer las bases generales del proyecto.  En
esta etapa se definen:

\begin{itemize}
    \item El alcance inicial del sistema.
    \item Los objetivos del proyecto.
    \item Los requisitos funcionales y no funcionales.
    \item Los actores involucrados.
    \item Los lineamientos metodológicos y técnicos.
\end{itemize}

Los resultados de esta fase alimentan el tablero Kanban y sirven como referencia
durante el desarrollo, permitiendo ajustes cuando el proyecto lo requiera.

\section{Ciclo de Vida del Desarrollo con Prototipado}

El ciclo de vida del sistema se desarrolla  de  manera  iterativa  a  través  de
prototipos, siguiendo las fases que se describen a continuación:

\subsection{Iteración 1 - Prototipo Inicial}

Incluye las siguientes etapas:

\begin{itemize}
    \item Análisis
    \begin{itemize}
        \item Refinamiento de historias de usuario prioritarias.
    \end{itemize}
\end{itemize}

\begin{itemize}
    \item Diseño
    \begin{itemize}
        \item Diseño de interfaz base.
        \item Diseño técnico preliminar.
    \end{itemize}
\end{itemize}

\begin{itemize}
    \item Implementación
    \begin{itemize}
        \item Desarrollo de la estructura funcional de la Web App.
        \item Configuración de base de datos.
        \item Implementación de lógica básica del sistema.
    \end{itemize}
\end{itemize}

\begin{itemize}
    \item Pruebas
    \begin{itemize}
        \item Pruebas funcionales iniciales.
        \item Validación de comportamiento general.
    \end{itemize}
\end{itemize}

El resultado de esta iteración es un prototipo funcional base.

\subsection{Iteración 2 - Prototipo Refinado}

\begin{itemize}
    \item Análisis
    \begin{itemize}
        \item Ajustes derivados del prototipo anterior.
        \item Definición de integración de inteligencia artificial.
    \end{itemize}
\end{itemize}

\begin{itemize}
    \item Diseño
    \begin{itemize}
        \item Diseño de integración del modelo de IA.
        \item Refinamiento arquitectónico.
    \end{itemize}
\end{itemize}

\begin{itemize}
    \item Implementación
    \begin{itemize}
        \item Integración del modelo de IA.
        \item Optimización del backend.
        \item Ajustes de interfaz.
    \end{itemize}
\end{itemize}

\begin{itemize}
    \item Pruebas
    \begin{itemize}
        \item Pruebas de integración.
        \item Validación de resultados predictivos.
        \item Corrección de defectos.
    \end{itemize}
\end{itemize}


El resultado es el prototipo final validado.

\subsection{Fase Final - Despliegue y Cierre}

Incluye:

\begin{itemize}
    \item Configuración del entorno productivo.
    \item Validación del pipeline de integración continua.
    \item Pruebas finales.
    \item Análisis de resultados.
    \item Documentación final.
\end{itemize}

Durante las iteraciones se realizaron despliegues incrementales en entornos de prueba, mientras que en esta fase se consolida la versión estable del sistema.
% IMPORTANTE: Renombra tu archivo "Alcances y Limitaciones.tex" a "alcances.tex"
\chapter{Alcances y Limitaciones}

\section{Alcances del proyecto}

\subsection{Desarrollo de una  Web  App  Funcional}  El  proyecto  contempla  la
construcción de una plataforma web plenamente operativa que permita gestionar la
información agrícola, visualizar datos relevantes y  acceder  a  funcionalidades
clave definidas  en  los  requisitos  del  sistema.
\subsection{Integración  de Modelos de IA para  Predicción}  El  sistema  incluirá  model
de inteligencia artificial generativa capaces de usar datos históricos, documentación relevante y
herramientas (tools), con  el  fin
de predecir variables agrícolas específicas. La IA  será  un  componente  activo
dentro de la Web App, permitiendo generar  análisis  y  pronósticos  dentro  del
flujo de la aplicación.
\subsection{Proceso Iterativo de Prototipos} El  alcance
incluye un ciclo iterativo de creación, evaluación y mejora de prototipos (fase
7.3). Cada iteración permitirá refinar tanto la interfaz como el desempeño
funcional y predictivo del sistema hasta obtener  un  prototipo  final  robusto.
\subsection{Arquitectura Base Sólida y Escalable} El proyecto abarca el diseño y
construcción de una arquitectura inicial estable  (fase  7.1),  que  sirva  como
plataforma para las iteraciones posteriores.  La  arquitectura  será  escalable,
modular  y   preparada   para   recibir   futuras   expansiones   del   sistema.
\subsection{Validación Funcional y Técnica} El proyecto comprende la  validación
del comportamiento de la Web App mediante pruebas de funcionamiento, análisis de
desempeño y revisión del comportamiento de los modelos de IA. Esto  asegura  que
el   sistema   cumpla   con   los    criterios    de    calidad    establecidos.
\subsection{Implementación  Final   del   Sistema}   El   alcance   incluye   la
consolidación de todas las funcionalidades definitivas, la integración global de
los componentes, pruebas finales y la entrega de una versión estable, lista para
operar. \subsection{Generación de Informes y Documentación Técnica}  Se  incluye
la elaboración de documentación técnica y metodológica  relevante,  informes  de
análisis, justificaciones de  decisiones  de  diseño  y documentación
en actas estructuradas del feedback del usuario final, con el fin  de  garantizar  claridad,  trazabilidad  y  soporte  para
futuras mejoras.


\section{Limitaciones del proyecto}

\subsection{No se realizará integración con sensores físicos ni  con  redes  WSN
    reales.} El sistema operará exclusivamente con datos de  entrada  proporcionados
por el usuario o mediante bases de conocimiento de referencia. \subsection{No se implementará
    una red LPWAN operativa.} El análisis comparativo será conceptual  y  basado  en
literatura técnica,  métricas  estimadas  y  modelos  analíticos  derivados  del
contexto  ingresado.  \subsection{El  sistema  no  ejecutará   simulaciones   de
    propagación, consumo energético o comportamiento físico de dispositivos IoT.} La
aplicación  procesará  información  contextual  para   ofrecer   recomendaciones
fundamentadas,  mas  no   simulaciones   numéricas   avanzadas.   \subsection{La
    inteligencia artificial no generará predicciones cuantitativas de comportamiento
    climático  o  agrícola.}  Su  función   se   limitará   a   la   identificación,
interpretación  y  estructuración  de  variables  relevantes  para  la  toma  de
decisiones tecnológicas. \subsection{Los resultados dependen  de  la  calidad  y
    precisión del contexto ingresado por el usuario.} Ambigüedades,  inconsistencias
o información insuficiente pueden afectar la pertinencia de las recomendaciones.
\subsection{Uso de planes gratuitos y APIs con acceso limitado} El desarrollo  y
despliegue del sistema se realizará utilizando  planes  gratuitos  de  servicios
externos,  tales  como  APIs  de  inteligencia  artificial  y   plataformas   de
despliegue,  lo  que  puede  imponer  restricciones  en  rendimiento,  escalabilidad   y
disponibilidad debido a las limitaciones presupuestales.

\section{Productos y Resultados de Investigación}

A continuación se presentan los siguientes productos principales que resultarán del proyecto de investigación.

\begin{table}[H]
    \centering
    \small % Tamaño de fuente ligeramente menor para mejorar el ajuste
    \begin{tabularx}{\textwidth}{@{} >{\raggedright\arraybackslash}p{4cm} >{\raggedright\arraybackslash}X c @{}}
        \toprule
        \textbf{Producto}                                   & \textbf{Descripción}                                                                                                                                                                                                & \textbf{Cantidad} \\
        \midrule
        Documento final de investigación (Trabajo de grado) & Informe que consolida todo el proceso investigativo: marco teórico, metodología, resultados, conclusiones, recomendaciones y trabajo futuro. Constituye el requisito para optar al título de Ingeniero de Sistemas. & 1                 \\
        \addlinespace
        Plataforma web                                     & Plataforma funcional para el análisis y simulación visual de arquitecturas de redes LPWAN con agentes de IA generativa, orientada a la toma de decisiones en agricultura de precisión.                              & 1                 \\
        \addlinespace
        Póster científico                                   & Material gráfico de divulgación que sintetiza los objetivos, metodología, resultados y conclusiones del proyecto para eventos académicos.                                                                           & 1                 \\
        \addlinespace
        Evento científico con componente de apropiación     & Participación en congresos o jornadas académicas para socializar los resultados y promover la apropiación social del conocimiento.                                                                                  & 1                 \\
        \bottomrule
    \end{tabularx}
    \caption{Productos y resultados de investigación esperados. Elaboración propia.}
    \label{tab:productos_completos}
\end{table}
% Requiere los paquetes: booktabs, tabularx, array
% Puedes agregarlos en el preámbulo con:
% \usepackage{booktabs, tabularx, array}

\chapter{Presupuesto}

Los siguientes valores se representan en pesos colombianos (COP) para el año 2026.

\begin{table}[H]
    \centering
    \begin{tabularx}{\textwidth}{>{\raggedright\arraybackslash}X c c c c}
        \toprule
        \textbf{Personal}         & \textbf{Horas/Semaman}                    & \textbf{Valor Hora} & \textbf{Valor Mes} & \textbf{Valor Semestre} \\
        \midrule
        Estudiante Investigador 1 & 20                                        & \$12.400            & \$1.000.000        & \$6.000.000             \\
        \addlinespace
        Estudiante Investigador 2 & 20                                        & \$12.500            & \$1.000.000        & \$6.000.000             \\
        \addlinespace
        Codirector (MsC)          & 4                                         & \$100.000           & \$1.600.000        & \$9.600.000             \\
        \addlinespace
        Director (PhD)            & 4                                         & \$500.000           & \$8.000.000        & \$48.000.000            \\
        \midrule
        \textbf{TOTAL RRHH}       & \multicolumn{4}{r}{\textbf{\$69.600.000}}                                                                      \\
        \bottomrule
    \end{tabularx}
    \caption{Presupuesto detallado de Recursos Humanos. Elaboración propia.}
    \label{tab:presupuesto_humano}
\end{table}

% --- TABLA 2: RECURSOS FÍSICOS ---
\begin{table}[H]
    \centering
    \begin{tabularx}{\textwidth}{>{\raggedright\arraybackslash}X c c c}
        \toprule
        \textbf{Elemento (Grupo LIDER)} & \textbf{Valor Unidad}                    & \textbf{Cantidad} & \textbf{Valor Semestre} \\
        \midrule
        Uso de computadores             & \$200.000                                & 2                 & \$2.400.000             \\
        \addlinespace
        Servicio de conectividad        & \$125.000                                & 2                 & \$1.500.000             \\
        \midrule
        \textbf{TOTAL FÍSICO}           & \multicolumn{3}{r}{\textbf{\$3.900.000}}                                               \\
        \bottomrule
    \end{tabularx}
    \caption{Presupuesto de Recursos Físicos e Infraestructura. Elaboración propia.}
    \label{tab:presupuesto_fisico}
\end{table}

% --- TABLA 3: SOFTWARE Y CLOUD (ESTRATEGIA FREE TIER) ---
\begin{table}[H]
    \centering
    \begin{tabularx}{\textwidth}{>{\raggedright\arraybackslash}X r}
        \toprule
        \textbf{Recurso Digital}                         & \textbf{Costo (COP)} \\
        \midrule
        Servicios de cómputo en la nube                  & \$0                  \\
        \addlinespace
        Servicios de almacenamiento en la nube           & \$0                  \\
        \addlinespace
        API de LLM (Modelos de Lenguaje)                 & \$0                  \\
        \addlinespace
        Servicio de almacenamiento de contenido estático & \$0                  \\
        \midrule
        \multicolumn{2}{l}{\textbf{TOTAL SOFTWARE}}                             \\
        \multicolumn{2}{r}{\textbf{\$0}}                                        \\
        \bottomrule
    \end{tabularx}
    \caption{Recursos de Software y Servicios Cloud bajo modalidad de capa gratuita. Elaboración propia.}
    \label{tab:presupuesto_software}
\end{table}
% --- TABLA 5: RESUMEN TOTAL ---
\begin{table}[H]
    \centering
    \begin{tabularx}{0.6\textwidth}{>{\raggedright\arraybackslash}X r}
        \toprule
        \textbf{Rubro}          & \textbf{Valor Total}  \\
        \midrule
        Recurso Humano          & \$69.600.000          \\
        Recurso Físico          & \$3.900.000           \\
        Software y Cloud        & \$0                   \\
        \midrule
        \textbf{TOTAL PROYECTO} & \textbf{\$73.500.000} \\
        \bottomrule
    \end{tabularx}
    \caption{Consolidado general del presupuesto del proyecto. Elaboración propia.}
    \label{tab:presupuesto_total}
\end{table}

\printbibliography[title=Bibliografía, heading=bibintoc]

\end{document}